% Options for packages loaded elsewhere
\PassOptionsToPackage{unicode}{hyperref}
\PassOptionsToPackage{hyphens}{url}
%
\documentclass[
]{article}
\title{TD1 - Analyse en Composantes Principales sur les données de
fertilité en Europe}
\author{Courtenay Rebecca \& Ducros Chloé \& Lasson Marie}
\date{}

\usepackage{amsmath,amssymb}
\usepackage{lmodern}
\usepackage{iftex}
\ifPDFTeX
  \usepackage[T1]{fontenc}
  \usepackage[utf8]{inputenc}
  \usepackage{textcomp} % provide euro and other symbols
\else % if luatex or xetex
  \usepackage{unicode-math}
  \defaultfontfeatures{Scale=MatchLowercase}
  \defaultfontfeatures[\rmfamily]{Ligatures=TeX,Scale=1}
\fi
% Use upquote if available, for straight quotes in verbatim environments
\IfFileExists{upquote.sty}{\usepackage{upquote}}{}
\IfFileExists{microtype.sty}{% use microtype if available
  \usepackage[]{microtype}
  \UseMicrotypeSet[protrusion]{basicmath} % disable protrusion for tt fonts
}{}
\makeatletter
\@ifundefined{KOMAClassName}{% if non-KOMA class
  \IfFileExists{parskip.sty}{%
    \usepackage{parskip}
  }{% else
    \setlength{\parindent}{0pt}
    \setlength{\parskip}{6pt plus 2pt minus 1pt}}
}{% if KOMA class
  \KOMAoptions{parskip=half}}
\makeatother
\usepackage{xcolor}
\IfFileExists{xurl.sty}{\usepackage{xurl}}{} % add URL line breaks if available
\IfFileExists{bookmark.sty}{\usepackage{bookmark}}{\usepackage{hyperref}}
\hypersetup{
  pdftitle={TD1 - Analyse en Composantes Principales sur les données de fertilité en Europe},
  pdfauthor={Courtenay Rebecca \& Ducros Chloé \& Lasson Marie},
  hidelinks,
  pdfcreator={LaTeX via pandoc}}
\urlstyle{same} % disable monospaced font for URLs
\usepackage[margin=1in]{geometry}
\usepackage{color}
\usepackage{fancyvrb}
\newcommand{\VerbBar}{|}
\newcommand{\VERB}{\Verb[commandchars=\\\{\}]}
\DefineVerbatimEnvironment{Highlighting}{Verbatim}{commandchars=\\\{\}}
% Add ',fontsize=\small' for more characters per line
\usepackage{framed}
\definecolor{shadecolor}{RGB}{248,248,248}
\newenvironment{Shaded}{\begin{snugshade}}{\end{snugshade}}
\newcommand{\AlertTok}[1]{\textcolor[rgb]{0.94,0.16,0.16}{#1}}
\newcommand{\AnnotationTok}[1]{\textcolor[rgb]{0.56,0.35,0.01}{\textbf{\textit{#1}}}}
\newcommand{\AttributeTok}[1]{\textcolor[rgb]{0.77,0.63,0.00}{#1}}
\newcommand{\BaseNTok}[1]{\textcolor[rgb]{0.00,0.00,0.81}{#1}}
\newcommand{\BuiltInTok}[1]{#1}
\newcommand{\CharTok}[1]{\textcolor[rgb]{0.31,0.60,0.02}{#1}}
\newcommand{\CommentTok}[1]{\textcolor[rgb]{0.56,0.35,0.01}{\textit{#1}}}
\newcommand{\CommentVarTok}[1]{\textcolor[rgb]{0.56,0.35,0.01}{\textbf{\textit{#1}}}}
\newcommand{\ConstantTok}[1]{\textcolor[rgb]{0.00,0.00,0.00}{#1}}
\newcommand{\ControlFlowTok}[1]{\textcolor[rgb]{0.13,0.29,0.53}{\textbf{#1}}}
\newcommand{\DataTypeTok}[1]{\textcolor[rgb]{0.13,0.29,0.53}{#1}}
\newcommand{\DecValTok}[1]{\textcolor[rgb]{0.00,0.00,0.81}{#1}}
\newcommand{\DocumentationTok}[1]{\textcolor[rgb]{0.56,0.35,0.01}{\textbf{\textit{#1}}}}
\newcommand{\ErrorTok}[1]{\textcolor[rgb]{0.64,0.00,0.00}{\textbf{#1}}}
\newcommand{\ExtensionTok}[1]{#1}
\newcommand{\FloatTok}[1]{\textcolor[rgb]{0.00,0.00,0.81}{#1}}
\newcommand{\FunctionTok}[1]{\textcolor[rgb]{0.00,0.00,0.00}{#1}}
\newcommand{\ImportTok}[1]{#1}
\newcommand{\InformationTok}[1]{\textcolor[rgb]{0.56,0.35,0.01}{\textbf{\textit{#1}}}}
\newcommand{\KeywordTok}[1]{\textcolor[rgb]{0.13,0.29,0.53}{\textbf{#1}}}
\newcommand{\NormalTok}[1]{#1}
\newcommand{\OperatorTok}[1]{\textcolor[rgb]{0.81,0.36,0.00}{\textbf{#1}}}
\newcommand{\OtherTok}[1]{\textcolor[rgb]{0.56,0.35,0.01}{#1}}
\newcommand{\PreprocessorTok}[1]{\textcolor[rgb]{0.56,0.35,0.01}{\textit{#1}}}
\newcommand{\RegionMarkerTok}[1]{#1}
\newcommand{\SpecialCharTok}[1]{\textcolor[rgb]{0.00,0.00,0.00}{#1}}
\newcommand{\SpecialStringTok}[1]{\textcolor[rgb]{0.31,0.60,0.02}{#1}}
\newcommand{\StringTok}[1]{\textcolor[rgb]{0.31,0.60,0.02}{#1}}
\newcommand{\VariableTok}[1]{\textcolor[rgb]{0.00,0.00,0.00}{#1}}
\newcommand{\VerbatimStringTok}[1]{\textcolor[rgb]{0.31,0.60,0.02}{#1}}
\newcommand{\WarningTok}[1]{\textcolor[rgb]{0.56,0.35,0.01}{\textbf{\textit{#1}}}}
\usepackage{graphicx}
\makeatletter
\def\maxwidth{\ifdim\Gin@nat@width>\linewidth\linewidth\else\Gin@nat@width\fi}
\def\maxheight{\ifdim\Gin@nat@height>\textheight\textheight\else\Gin@nat@height\fi}
\makeatother
% Scale images if necessary, so that they will not overflow the page
% margins by default, and it is still possible to overwrite the defaults
% using explicit options in \includegraphics[width, height, ...]{}
\setkeys{Gin}{width=\maxwidth,height=\maxheight,keepaspectratio}
% Set default figure placement to htbp
\makeatletter
\def\fps@figure{htbp}
\makeatother
\setlength{\emergencystretch}{3em} % prevent overfull lines
\providecommand{\tightlist}{%
  \setlength{\itemsep}{0pt}\setlength{\parskip}{0pt}}
\setcounter{secnumdepth}{5}
\ifLuaTeX
  \usepackage{selnolig}  % disable illegal ligatures
\fi

\begin{document}
\maketitle

{
\setcounter{tocdepth}{2}
\tableofcontents
}
\hypertarget{pruxe9sentation}{%
\section{Présentation}\label{pruxe9sentation}}

On s'intéresse à la fertilité des femmes en Europe en 2012. Pour cela,
on a construit un tableau de données avec 39 Pays d'Europe en lignes et
en colonnes les tranches d'âge 15-19, 20-24, 25-29, 30-34, 35-39, 40 et
plus. Dans une case du tableau on a la fertilité moyenne des femmes d'un
pays pour une tranche d'âge. La fertilité correspond ici au nombre moyen
d'enfants nés vivants pour 1000 femmes. Pour répondre à l'ensemble des
questions, il vous faudra mettre en oeuvre une ACP à partir des données
fournies dans le fichier ``AnaDo\_JeuDonnees\_FertiliteEurope.csv''

Il vous est demandé de fournir un rapport répondant aux questions
posées. Dans ce rapport, chaque étape de votre code devra être indiquée
: chargement de la librairie de l'ACP, chargement des données, affichage
des graphes et des données permettant de répondre aux questions, etc.

Pour fournir le rapport, vous pouvez au choix : compléter ce fichier et
le compiler pour créer le PDF avec les réponses ; ou bien créer votre
propre rapport répondant aux questions. Dans les deux cas, les étapes
vous permettant de répondre aux questions devront apparaître (chargement
des données, affichage des graphes, \ldots)

\hypertarget{questions}{%
\section{Questions}\label{questions}}

\begin{enumerate}
\def\labelenumi{\arabic{enumi})}
\item
  Importer les données et en faire l'ACP. Dans ce jeu de données, que
  représentent les individus et que représentent les variables ?
\item
  En analysant les deux graphes générés par l'ACP de la question 1),
  indiquer :

  \begin{enumerate}
  \def\labelenumii{\alph{enumii}.}
  \item
    Quelle est la variable la plus liée à la dimension 2 ?
  \item
    Quel pays a le taux de fertilité le plus élevé pour la classe 20-24
    ans ?
  \item
    Que peut-on dire du taux de fertilité à tout âge des Danoises et des
    Hollandaises ?
  \item
    Que peut-on dire du taux de fertilité des Irlandaises
    comparativement aux autres pays d'Europe ?
  \item
    Citer quatre pays ayant un taux de fertilité très proche pour toutes
    les classes d'âge.
  \item
    Le taux de fertilité (pour toutes les classes d'âge) des Croates
    est-il proche de 0 ou proche du taux de fertilité moyen des pays
    d'Europe ?
  \item
    Que peut-on dire sur la fertilité des femmes de plus de 35 ans dans
    les pays où la fertilité des adolescentes est faible ?
  \end{enumerate}
\item
  En utilisant la fonction \texttt{summary}, indiquer le pourcentage
  d'inertie associé au plan principal. Ensuite, indiquer quel est le
  pays ayant le plus contribué à la création de l'axe 1, puis quels sont
  les trois pays ayant le plus contribué à la création de l'axe 2. Bien
  montrer la table des résultats ayant permis de répondre à cette
  question.
\item
  En utilisant la fonction \texttt{plot}, tracer le graphe des individus
  et des variables sur les dimensions 3 et 4.
\item
  En utilisant la fonction \texttt{dimdesc}, indiquer quelles sont les
  quatre variables ayant un taux de corrélation (positif ou négatif) de
  plus de 0.8 avec la première dimension.
\item
  Quels sont les 6 individus ayant le plus contribué à la construction
  du plan (sur les dimensions 1-2) ? Bonus : tracer un graphique pour
  répondre à cette question.
\item
  Tracer le cercle des corrélations sur les dimensions 1 et 2 en y
  affichant les deux variables ayant le plus contribué.
\item
  Faire l'ACP avec la variable supplémentaire ``Région'' puis tracer le
  graphe des individus sur les dimensions 1 et 2 avec le coloriage en
  fonction de la région. Sur ce graphe, ne sélectionner que les indivus
  qui ont une qualité de projection supérieure à 0.95. Comparer la
  région Europe de l'Est à la région Europe du Nord.
\end{enumerate}

\hypertarget{template}{%
\section{Template}\label{template}}

\hypertarget{chargement-de-la-librairie}{%
\subsection{Chargement de la
librairie}\label{chargement-de-la-librairie}}

\begin{Shaded}
\begin{Highlighting}[]
\FunctionTok{library}\NormalTok{(FactoMineR)}
\end{Highlighting}
\end{Shaded}

\hypertarget{importation-des-donnuxe9es-et-affichage-des-graphes}{%
\subsection{1.Importation des données et affichage des
graphes}\label{importation-des-donnuxe9es-et-affichage-des-graphes}}

On importe dans un premier temps les données de fertilité en Europe.

\begin{Shaded}
\begin{Highlighting}[]
\NormalTok{donnees }\OtherTok{=} \FunctionTok{read.csv}\NormalTok{(}\AttributeTok{file =} \StringTok{"AnaDo\_JeuDonnees\_FertiliteEurope.csv"}\NormalTok{, }\AttributeTok{sep =} \StringTok{";"}\NormalTok{, }\AttributeTok{dec =} \StringTok{"."}\NormalTok{, }
                   \AttributeTok{header=}\ConstantTok{TRUE}\NormalTok{, }\AttributeTok{row.names =} \DecValTok{1}\NormalTok{)}
\FunctionTok{head}\NormalTok{(donnees)}
\end{Highlighting}
\end{Shaded}

\begin{verbatim}
##                    X15.19 X20.24 X25.29 X30.34 X35.39 X40.et..       Région
## Albanie              20.8  107.1  126.7   72.3   24.4      5.5   Europe sud
## Allemagne             8.0   36.4   78.9   94.6   50.9      9.3 Europe Ouest
## Autriche              8.6   44.5   89.2   94.6   46.8      9.6 Europe Ouest
## Belgique              9.0   52.6  127.0  116.6   48.3      9.6 Europe Ouest
## Biélorussie          21.4   90.4  106.7   67.6   26.0      4.4   Europe est
## Bosnie-Herzégovine   11.0   52.5   91.8   69.7   25.7      4.7   Europe sud
\end{verbatim}

Les individus concernés sont les femmes. Elles sont classées selon les
pays et les régions d'Europe dans lesquelles elles se trouvent mais
aussi dans la tranche d'âge. Ainsi, pour chaque tranche d'âge nous avons
la moyenne de la fertilité des femmes dans leur pays respectif. On peut
voir qu'il y a de deux variables qualitatives et 6 variables
quantitatives. Les 2 variables quantitatives sont :

Pays

Région

Les 4 variables qualitatives sont :

15-19

20-24

25-29

30-34

35-39

40+

On fait l'ACP de ces données.

\begin{Shaded}
\begin{Highlighting}[]
\CommentTok{\#Il faut enlever les valeurs quantitatives }
\NormalTok{p }\OtherTok{\textless{}{-}} \FunctionTok{ncol}\NormalTok{(donnees)}
\NormalTok{n }\OtherTok{\textless{}{-}} \FunctionTok{nrow}\NormalTok{(donnees)}

\NormalTok{nouveauDonnees }\OtherTok{\textless{}{-}}\NormalTok{ donnees[,}\FunctionTok{c}\NormalTok{(}\DecValTok{2}\SpecialCharTok{:}\NormalTok{p}\DecValTok{{-}1}\NormalTok{)]}
\NormalTok{acp }\OtherTok{\textless{}{-}} \FunctionTok{PCA}\NormalTok{(nouveauDonnees);acp}
\end{Highlighting}
\end{Shaded}

\includegraphics{Template_TD1_ACP_files/figure-latex/unnamed-chunk-3-1.pdf}
\includegraphics{Template_TD1_ACP_files/figure-latex/unnamed-chunk-3-2.pdf}

\begin{verbatim}
## **Results for the Principal Component Analysis (PCA)**
## The analysis was performed on 39 individuals, described by 6 variables
## *The results are available in the following objects:
## 
##    name               description                          
## 1  "$eig"             "eigenvalues"                        
## 2  "$var"             "results for the variables"          
## 3  "$var$coord"       "coord. for the variables"           
## 4  "$var$cor"         "correlations variables - dimensions"
## 5  "$var$cos2"        "cos2 for the variables"             
## 6  "$var$contrib"     "contributions of the variables"     
## 7  "$ind"             "results for the individuals"        
## 8  "$ind$coord"       "coord. for the individuals"         
## 9  "$ind$cos2"        "cos2 for the individuals"           
## 10 "$ind$contrib"     "contributions of the individuals"   
## 11 "$call"            "summary statistics"                 
## 12 "$call$centre"     "mean of the variables"              
## 13 "$call$ecart.type" "standard error of the variables"    
## 14 "$call$row.w"      "weights for the individuals"        
## 15 "$call$col.w"      "weights for the variables"
\end{verbatim}

\hypertarget{interpruxe9tation-de-lacp}{%
\subsection{2.Interprétation de l'ACP}\label{interpruxe9tation-de-lacp}}

\begin{verbatim}
a. 
\end{verbatim}

La variable la plus liée à la dimension 2 est : 25-29.

\begin{verbatim}
b. 
\end{verbatim}

Le taux de fertilité le plus élevé pour la classe 20-24 semble être
l'Albanie.

\begin{verbatim}
c.
\end{verbatim}

Le taux de fertilité à tout âge des Danoises et des Hollandaises
semblent être, en moyenne, la même. En effet, les deux points sont
proches l'un de l'autre. Il est donc probable que les variables des
individus soient très similaires.

\begin{verbatim}
d.
\end{verbatim}

D'après le graphique des variables, on remarque que l'Irlande est plus
impacté par la dim 1 alors qu'il ne l'est pas du tout par la dim 2. Il
est donc corrélé par l'axe 1. Nous remarquons également que plus nous
allons à droite plus nous avons une représentation des femmes âgées
tandis que si nous allons vers le côté gauche, nous avons une
représentation des femmes jeunes. Donc l'Irlande, qui a son point sur la
dimension 1 du côté droit, a un taux de fertilité qui est généralement
plus fréquent chez les femmes ayant minimum 30 ans.

\begin{verbatim}
e.
\end{verbatim}

Estonie, Croatie, République-Tchèque et Lettonie.

\begin{verbatim}
f.
\end{verbatim}

Le taux de fertilité des Croates est proche du taux de fertilitié moyen
des pays d'Europe car les valeurs sont centrées sur une valeur moyenne.

\begin{verbatim}
g.
\end{verbatim}

La fertilité des femmes de plus de 35 ans est forte dans les pays où la
fertilité des adolescentes est faible. En effet, la classe 15-19 a une
corrélation négative alors que les classes 35+ ont une corrélation
positive: ces classes sont très corrélées mais dans un sens inverse. Ils
sont donc opposés.

\hypertarget{pourcentage-dinertie-du-plan-principal}{%
\subsection{3.Pourcentage d'inertie du plan
principal}\label{pourcentage-dinertie-du-plan-principal}}

\begin{Shaded}
\begin{Highlighting}[]
\FunctionTok{summary}\NormalTok{(acp, }\AttributeTok{nbelements =} \ConstantTok{Inf}\NormalTok{)}
\end{Highlighting}
\end{Shaded}

\begin{verbatim}
## 
## Call:
## PCA(X = nouveauDonnees) 
## 
## 
## Eigenvalues
##                        Dim.1   Dim.2   Dim.3   Dim.4   Dim.5   Dim.6
## Variance               3.702   1.360   0.677   0.206   0.046   0.009
## % of var.             61.704  22.665  11.287   3.427   0.772   0.145
## Cumulative % of var.  61.704  84.369  95.656  99.084  99.855 100.000
## 
## Individuals
##                        Dist    Dim.1    ctr   cos2    Dim.2    ctr   cos2  
## Albanie            |  3.883 | -2.879  5.742  0.550 |  2.435 11.182  0.393 |
## Allemagne          |  1.704 |  0.843  0.492  0.244 | -1.355  3.459  0.632 |
## Autriche           |  1.007 |  0.567  0.222  0.317 | -0.647  0.789  0.413 |
## Belgique           |  2.141 |  0.937  0.608  0.192 |  1.730  5.643  0.653 |
## Biélorussie        |  2.990 | -2.762  5.283  0.853 |  0.964  1.751  0.104 |
## Bosnie-Herzégovine |  2.073 | -1.425  1.406  0.472 | -0.670  0.846  0.104 |
## Bulgarie           |  3.870 | -3.359  7.816  0.754 | -0.338  0.216  0.008 |
## Croatie            |  0.651 | -0.244  0.041  0.140 |  0.043  0.004  0.004 |
## Danemark           |  2.352 |  1.989  2.740  0.715 |  0.774  1.131  0.108 |
## Espagne            |  3.293 |  1.818  2.290  0.305 | -2.637 13.110  0.641 |
## Estonie            |  0.537 | -0.086  0.005  0.026 |  0.086  0.014  0.026 |
## Finlande           |  1.729 |  1.569  1.704  0.823 |  0.697  0.916  0.163 |
## France             |  2.894 |  1.726  2.063  0.356 |  2.293  9.911  0.628 |
## Grèce              |  1.950 |  0.998  0.690  0.262 | -1.617  4.932  0.688 |
## Hongrie            |  1.674 | -0.630  0.275  0.142 | -1.478  4.121  0.780 |
## Irlande            |  5.081 |  4.133 11.828  0.662 | -0.049  0.005  0.000 |
## Islande            |  2.724 |  2.096  3.043  0.592 |  1.686  5.362  0.383 |
## Italie             |  2.809 |  2.064  2.950  0.540 | -1.698  5.438  0.366 |
## Lettonie           |  0.842 | -0.422  0.124  0.252 | -0.339  0.216  0.162 |
## Lituanie           |  1.588 | -0.778  0.419  0.240 |  0.896  1.515  0.319 |
## Luxembourg         |  1.943 |  1.754  2.130  0.815 | -0.754  1.072  0.151 |
## Macédoine          |  2.135 | -2.031  2.857  0.905 |  0.449  0.380  0.044 |
## Malte              |  1.280 | -0.216  0.032  0.029 | -0.924  1.611  0.521 |
## Moldavie           |  3.699 | -3.501  8.490  0.896 | -1.006  1.906  0.074 |
## Monténégro         |  1.546 | -0.523  0.190  0.115 |  1.299  3.180  0.706 |
## Norvège            |  2.085 |  1.647  1.879  0.624 |  1.134  2.424  0.296 |
## Pays-Bas           |  2.635 |  2.142  3.178  0.661 |  0.655  0.808  0.062 |
## Pologne            |  1.395 | -1.089  0.821  0.610 | -0.520  0.509  0.139 |
## Portugal           |  1.906 |  0.157  0.017  0.007 | -1.876  6.634  0.968 |
## Rép. Tchèque       |  1.199 |  0.003  0.000  0.000 | -0.454  0.388  0.143 |
## Roumanie           |  3.403 | -3.052  6.452  0.805 | -0.455  0.391  0.018 |
## Royaume-Uni        |  2.181 |  1.251  1.083  0.329 |  0.921  1.599  0.178 |
## Russie             |  2.727 | -2.403  3.998  0.776 |  0.836  1.317  0.094 |
## Serbie             |  1.450 | -1.416  1.390  0.954 |  0.246  0.114  0.029 |
## Slovaquie          |  1.763 | -1.398  1.353  0.628 | -0.945  1.683  0.287 |
## Slovénie           |  1.611 |  0.780  0.421  0.234 |  0.453  0.387  0.079 |
## Suède              |  3.014 |  2.739  5.195  0.826 |  1.228  2.844  0.166 |
## Suisse             |  2.529 |  2.246  3.493  0.789 | -1.078  2.190  0.182 |
## Ukraine            |  3.376 | -3.242  7.277  0.922 |  0.013  0.000  0.000 |
##                     Dim.3    ctr   cos2  
## Albanie             0.292  0.322  0.006 |
## Allemagne          -0.516  1.007  0.092 |
## Autriche           -0.473  0.846  0.220 |
## Belgique           -0.750  2.128  0.123 |
## Biélorussie         0.085  0.028  0.001 |
## Bosnie-Herzégovine -1.257  5.983  0.368 |
## Bulgarie            1.212  5.560  0.098 |
## Croatie            -0.578  1.265  0.789 |
## Danemark           -0.919  3.197  0.153 |
## Espagne             0.671  1.704  0.042 |
## Estonie             0.382  0.553  0.507 |
## Finlande            0.046  0.008  0.001 |
## France              0.098  0.036  0.001 |
## Grèce              -0.291  0.322  0.022 |
## Hongrie            -0.127  0.061  0.006 |
## Irlande             2.948 32.905  0.337 |
## Islande             0.429  0.696  0.025 |
## Italie              0.552  1.152  0.039 |
## Lettonie            0.248  0.233  0.087 |
## Lituanie           -0.904  3.092  0.324 |
## Luxembourg          0.098  0.037  0.003 |
## Macédoine          -0.438  0.725  0.042 |
## Malte              -0.413  0.647  0.104 |
## Moldavie            0.293  0.325  0.006 |
## Monténégro          0.122  0.056  0.006 |
## Norvège            -0.532  1.071  0.065 |
## Pays-Bas           -1.096  4.550  0.173 |
## Pologne            -0.578  1.266  0.172 |
## Portugal           -0.267  0.270  0.020 |
## Rép. Tchèque       -1.047  4.154  0.763 |
## Roumanie            1.004  3.814  0.087 |
## Royaume-Uni         1.507  8.601  0.477 |
## Russie              0.962  3.501  0.124 |
## Serbie              0.101  0.039  0.005 |
## Slovaquie          -0.258  0.253  0.021 |
## Slovénie           -1.330  6.696  0.681 |
## Suède               0.195  0.143  0.004 |
## Suisse             -0.276  0.289  0.012 |
## Ukraine             0.807  2.466  0.057 |
## 
## Variables
##                       Dim.1    ctr   cos2    Dim.2    ctr   cos2    Dim.3
## X15.19             | -0.831 18.668  0.691 | -0.024  0.042  0.001 |  0.455
## X20.24             | -0.728 14.319  0.530 |  0.533 20.899  0.284 |  0.358
## X25.29             |  0.041  0.046  0.002 |  0.979 70.413  0.958 | -0.168
## X30.34             |  0.908 22.280  0.825 |  0.342  8.591  0.117 | -0.060
## X35.39             |  0.942 23.981  0.888 |  0.026  0.049  0.001 |  0.322
## X40.et..           |  0.876 20.705  0.767 | -0.008  0.005  0.000 |  0.454
##                       ctr   cos2  
## X15.19             30.562  0.207 |
## X20.24             18.973  0.128 |
## X25.29              4.154  0.028 |
## X30.34              0.539  0.004 |
## X35.39             15.304  0.104 |
## X40.et..           30.468  0.206 |
\end{verbatim}

Le pourcentage d'inertie associé au plan principal est : 84.36911\%.

\begin{Shaded}
\begin{Highlighting}[]
\NormalTok{acp}\SpecialCharTok{$}\NormalTok{eig}
\end{Highlighting}
\end{Shaded}

\begin{verbatim}
##         eigenvalue percentage of variance cumulative percentage of variance
## comp 1 3.702253939             61.7042323                          61.70423
## comp 2 1.359892659             22.6648777                          84.36911
## comp 3 0.677217935             11.2869656                          95.65608
## comp 4 0.205646734              3.4274456                          99.08352
## comp 5 0.046294862              0.7715810                          99.85510
## comp 6 0.008693871              0.1448978                         100.00000
\end{verbatim}

Axe1 : Irlande (11.828) Axe2 : Espagne (13.110), Albanie (11.182),
France (9.911)

\begin{Shaded}
\begin{Highlighting}[]
\NormalTok{acp}\SpecialCharTok{$}\NormalTok{ind}\SpecialCharTok{$}\NormalTok{contrib}
\end{Highlighting}
\end{Shaded}

\begin{verbatim}
##                           Dim.1        Dim.2        Dim.3        Dim.4
## Albanie            5.742210e+00 1.118242e+01  0.321820976 9.330586e+00
## Allemagne          4.916964e-01 3.459449e+00  1.007235535 1.020979e-01
## Autriche           2.222922e-01 7.888806e-01  0.846050562 5.752118e-01
## Belgique           6.079461e-01 5.642806e+00  2.127579526 1.221312e+00
## Biélorussie        5.283230e+00 1.750919e+00  0.027573307 3.667731e+00
## Bosnie-Herzégovine 1.405588e+00 8.457253e-01  5.983087742 2.772611e+00
## Bulgarie           7.816444e+00 2.155357e-01  5.559918286 2.603839e+01
## Croatie            4.110323e-02 3.539326e-03  1.265155050 3.137510e-01
## Danemark           2.740358e+00 1.130847e+00  3.197048603 1.620986e+00
## Espagne            2.289882e+00 1.311049e+01  1.703669119 1.643068e+00
## Estonie            5.135330e-03 1.391644e-02  0.553332766 1.971515e-01
## Finlande           1.704490e+00 9.162448e-01  0.008033388 4.775301e-01
## France             2.063421e+00 9.910815e+00  0.036205853 5.813942e-01
## Grèce              6.902389e-01 4.932169e+00  0.321658872 4.672534e-01
## Hongrie            2.751937e-01 4.121389e+00  0.060726566 2.517531e+00
## Irlande            1.182844e+01 4.617105e-03 32.905006761 1.968052e-01
## Islande            3.043349e+00 5.361999e+00  0.696113428 9.985601e-04
## Italie             2.949766e+00 5.438278e+00  1.152388008 4.015745e+00
## Lettonie           1.235036e-01 2.164148e-01  0.233248029 1.828058e+00
## Lituanie           4.193486e-01 1.514723e+00  3.091989591 1.607663e-01
## Luxembourg         2.130304e+00 1.072335e+00  0.036733589 3.774505e-01
## Macédoine          2.856731e+00 3.802827e-01  0.725144278 1.330208e-01
## Malte              3.241266e-02 1.611012e+00  0.647298885 6.872285e+00
## Moldavie           8.489856e+00 1.906405e+00  0.324769009 3.632156e+00
## Monténégro         1.895961e-01 3.179623e+00  0.056297785 4.401554e+00
## Norvège            1.878574e+00 2.423888e+00  1.071281202 3.515673e-01
## Pays-Bas           3.178394e+00 8.082201e-01  4.549517907 6.048351e+00
## Pologne            8.213031e-01 5.089875e-01  1.265575104 6.492874e-01
## Portugal           1.706368e-02 6.634490e+00  0.270251525 2.046935e-01
## Rép. Tchèque       6.287958e-06 3.878109e-01  4.154112145 1.147847e+00
## Roumanie           6.452351e+00 3.905264e-01  3.814179852 1.280720e+01
## Royaume-Uni        1.083363e+00 1.598936e+00  8.601102886 7.410663e-01
## Russie             3.997939e+00 1.317363e+00  3.500521675 2.810967e-01
## Serbie             1.389523e+00 1.142734e-01  0.038507919 9.983192e-02
## Slovaquie          1.352885e+00 1.682571e+00  0.252890968 2.407451e+00
## Slovénie           4.208368e-01 3.874167e-01  6.695950317 1.623547e-02
## Suède              5.194648e+00 2.844148e+00  0.143306326 2.905404e-01
## Suisse             3.493297e+00 2.190199e+00  0.288906672 2.046965e-01
## Ukraine            7.277283e+00 3.347046e-04  2.465809987 1.604696e+00
##                           Dim.5
## Albanie            8.683907e-01
## Allemagne          2.946085e+00
## Autriche           1.945865e-01
## Belgique           2.847246e+00
## Biélorussie        4.057238e+00
## Bosnie-Herzégovine 9.238590e-01
## Bulgarie           8.719383e-01
## Croatie            1.511772e-01
## Danemark           1.649746e-02
## Espagne            2.804208e-02
## Estonie            5.825488e+00
## Finlande           7.228573e-02
## France             4.096185e+00
## Grèce              2.358690e+00
## Hongrie            1.235717e-01
## Irlande            1.756255e-01
## Islande            9.523625e-03
## Italie             5.966154e+00
## Lettonie           1.098573e+01
## Lituanie           1.421425e+01
## Luxembourg         2.544759e+00
## Macédoine          1.589868e+00
## Malte              1.120095e-01
## Moldavie           1.992433e+00
## Monténégro         2.362191e+00
## Norvège            1.455343e+00
## Pays-Bas           1.205145e+01
## Pologne            5.282008e+00
## Portugal           7.752790e-05
## Rép. Tchèque       1.575511e+00
## Roumanie           1.034046e+00
## Royaume-Uni        8.095398e-01
## Russie             1.014979e+00
## Serbie             6.569832e-01
## Slovaquie          2.684412e-02
## Slovénie           2.500553e-01
## Suède              4.075717e-01
## Suisse             4.330877e+00
## Ukraine            5.770889e+00
\end{verbatim}

\hypertarget{graphe-sur-les-dimensions-3-et-4}{%
\subsection{4.Graphe sur les dimensions 3 et
4}\label{graphe-sur-les-dimensions-3-et-4}}

\begin{Shaded}
\begin{Highlighting}[]
\FunctionTok{plot}\NormalTok{(acp, }\AttributeTok{axes =} \FunctionTok{c}\NormalTok{(}\DecValTok{3}\NormalTok{,}\DecValTok{4}\NormalTok{))}
\end{Highlighting}
\end{Shaded}

\includegraphics{Template_TD1_ACP_files/figure-latex/unnamed-chunk-7-1.pdf}

\begin{Shaded}
\begin{Highlighting}[]
\FunctionTok{plot}\NormalTok{(acp, }\AttributeTok{axes =} \FunctionTok{c}\NormalTok{(}\DecValTok{3}\NormalTok{,}\DecValTok{4}\NormalTok{), }\AttributeTok{choix=}\StringTok{\textquotesingle{}var\textquotesingle{}}\NormalTok{)}
\end{Highlighting}
\end{Shaded}

\includegraphics{Template_TD1_ACP_files/figure-latex/unnamed-chunk-7-2.pdf}

\hypertarget{variables-corruxe9luxe9es-uxe0-la-premiuxe8re-dimension}{%
\subsection{5.Variables corrélées à la première
dimension}\label{variables-corruxe9luxe9es-uxe0-la-premiuxe8re-dimension}}

\begin{Shaded}
\begin{Highlighting}[]
\FunctionTok{dimdesc}\NormalTok{(acp, }\AttributeTok{axes =} \DecValTok{1}\NormalTok{)}
\end{Highlighting}
\end{Shaded}

\begin{verbatim}
## $Dim.1
## $quanti
##          correlation      p.value
## X35.39     0.9422466 3.648887e-19
## X30.34     0.9082239 1.433901e-15
## X40.et..   0.8755389 3.021708e-13
## X20.24    -0.7280993 1.496018e-07
## X15.19    -0.8313477 5.634938e-11
## 
## attr(,"class")
## [1] "condes" "list"  
## 
## $call
## $call$num.var
## [1] 1
## 
## $call$proba
## [1] 0.05
## 
## $call$weights
##  [1] 1 1 1 1 1 1 1 1 1 1 1 1 1 1 1 1 1 1 1 1 1 1 1 1 1 1 1 1 1 1 1 1 1 1 1 1 1 1
## [39] 1
## 
## $call$X
##                           Dim.1 X15.19 X20.24 X25.29 X30.34 X35.39 X40.et..
## Albanie            -2.879419548   20.8  107.1  126.7   72.3   24.4      5.5
## Allemagne           0.842585388    8.0   36.4   78.9   94.6   50.9      9.3
## Autriche            0.566536039    8.6   44.5   89.2   94.6   46.8      9.6
## Belgique            0.936910183    9.0   52.6  127.0  116.6   48.3      9.6
## Biélorussie        -2.761946002   21.4   90.4  106.7   67.6   26.0      4.4
## Bosnie-Herzégovine -1.424604825   11.0   52.5   91.8   69.7   25.7      4.7
## Bulgarie           -3.359464201   42.1   71.9   90.9   67.1   27.2      4.7
## Croatie            -0.243614629   11.4   54.6   99.1   91.4   39.9      7.8
## Danemark            1.989156842    4.4   37.3  112.9  125.0   55.5     10.8
## Espagne             1.818326879    8.9   29.2   57.6   91.2   62.4     14.9
## Estonie            -0.086109202   16.1   57.9   99.6   87.0   46.6     10.9
## Finlande            1.568782205    7.3   52.0  107.5  112.3   58.0     12.8
## France              1.726073787    9.5   59.3  132.1  127.0   58.3     13.4
## Grèce               0.998309325    8.5   32.6   76.5   91.6   47.8     11.7
## Hongrie            -0.630354215   19.2   41.9   77.9   81.5   39.3      7.7
## Irlande             4.132654402   12.1   51.1   87.8  130.3   97.5     23.8
## Islande             2.096241442    8.1   58.3  120.5  126.0   65.0     14.4
## Italie              2.063759847    6.4   33.4   73.2   95.3   61.1     16.2
## Lettonie           -0.422284634   15.4   57.1   93.8   78.7   40.3     10.8
## Lituanie           -0.778131530   14.0   55.4  117.3   86.5   34.1      6.6
## Luxembourg          1.753824803    7.0   41.3   84.3  110.3   57.3     13.2
## Macédoine          -2.030954064   19.0   72.1  103.0   76.5   27.8      4.4
## Malte              -0.216332974   17.8   39.4   87.1   92.8   42.9      7.1
## Moldavie           -3.501189071   25.7   80.0   78.4   46.1   19.9      3.8
## Monténégro         -0.523214971   13.3   74.6  114.9   89.8   39.4     10.1
## Norvège             1.646946571    6.1   48.9  115.1  122.2   57.0     10.7
## Pays-Bas            2.142245544    4.5   35.3  108.4  136.7   55.8      9.6
## Pologne            -1.088973075   14.7   53.1   93.7   73.0   32.7      6.9
## Portugal            0.156964591   12.0   37.5   71.4   82.8   43.4      9.5
## Rép. Tchèque        0.003013146   11.0   42.1   93.4   98.9   38.0      6.7
## Roumanie           -3.052280092   37.3   70.8   89.3   65.5   27.8      5.3
## Royaume-Uni         1.250697949   18.1   65.4  105.1  114.6   65.2     14.8
## Russie             -2.402611253   26.9   89.3  103.7   71.9   33.6      7.0
## Serbie             -1.416440335   20.6   68.4   99.0   83.2   34.5      6.9
## Slovaquie          -1.397641621   21.3   51.1   85.2   77.9   32.1      6.0
## Slovénie            0.779511048    4.5   43.8  108.2  106.8   44.4      7.7
## Suède               2.738693839    5.5   47.7  114.3  134.1   67.6     14.9
## Suisse              2.245862331    3.5   31.7   81.9  111.6   63.0     12.5
## Ukraine            -3.241529918   28.0   90.6   90.2   58.9   25.2      5.1
\end{verbatim}

Les quatre variables ayant un taux de corrélation de plus de 0.8 à la
première dimension sont :

15-19

30-34

35-39

40 et..

\hypertarget{individus-contribuant-le-plus-uxe0-la-cruxe9ation-du-plan}{%
\subsection{6.Individus contribuant le plus à la création du
plan}\label{individus-contribuant-le-plus-uxe0-la-cruxe9ation-du-plan}}

\begin{Shaded}
\begin{Highlighting}[]
\NormalTok{acp2 }\OtherTok{\textless{}{-}} \FunctionTok{PCA}\NormalTok{(nouveauDonnees,}\AttributeTok{ncp=}\DecValTok{2}\NormalTok{,}\AttributeTok{graph=}\ConstantTok{FALSE}\NormalTok{)}
\FunctionTok{plot}\NormalTok{(acp2,}\AttributeTok{invisible=}\FunctionTok{c}\NormalTok{(}\StringTok{\textquotesingle{}ind.sup\textquotesingle{}}\NormalTok{),}\AttributeTok{select=}\StringTok{\textquotesingle{}contrib  6\textquotesingle{}}\NormalTok{,}\AttributeTok{habillage=}\StringTok{\textquotesingle{}contrib\textquotesingle{}}\NormalTok{,}
     \AttributeTok{title=}\StringTok{"Six individus ayant contribué à la création du plan"}\NormalTok{)}
\end{Highlighting}
\end{Shaded}

\includegraphics{Template_TD1_ACP_files/figure-latex/unnamed-chunk-9-1.pdf}

Les six individus ayant le plus contribué à la création du plan sont:
Irlande, Espagne, Moldavie, Bulgarie, Ukraine et Alabnie.

\hypertarget{cercle-des-corruxe9lations}{%
\subsection{7.Cercle des
corrélations}\label{cercle-des-corruxe9lations}}

\begin{Shaded}
\begin{Highlighting}[]
\FunctionTok{plot}\NormalTok{(acp2,}\AttributeTok{choix=}\StringTok{\textquotesingle{}var\textquotesingle{}}\NormalTok{,}\AttributeTok{select=}\StringTok{\textquotesingle{}contrib  2\textquotesingle{}}\NormalTok{,}\AttributeTok{unselect=}\DecValTok{0}\NormalTok{)}
\end{Highlighting}
\end{Shaded}

\includegraphics{Template_TD1_ACP_files/figure-latex/unnamed-chunk-10-1.pdf}
Les deux tranches d'âge qui ont le plus contribué sont 25-29 ans et
30-34 ans.

\hypertarget{graphique-final}{%
\subsection{8.Graphique final}\label{graphique-final}}

Faire l'ACP avec la variable supplémentaire ``Région'' puis tracer le
graphe des individus sur les dimensions 1 et 2 avec le coloriage en
fonction de la région. Sur ce graphe, ne sélectionner que les indivus
qui ont une qualité de projection supérieure à 0.95.

\begin{Shaded}
\begin{Highlighting}[]
\NormalTok{acp3}\OtherTok{\textless{}{-}}\FunctionTok{PCA}\NormalTok{(donnees,}\AttributeTok{ncp=}\DecValTok{2}\NormalTok{,}\AttributeTok{quali.sup=}\FunctionTok{c}\NormalTok{(}\DecValTok{7}\NormalTok{),}\AttributeTok{graph=}\ConstantTok{FALSE}\NormalTok{)}
\end{Highlighting}
\end{Shaded}

\begin{Shaded}
\begin{Highlighting}[]
\FunctionTok{plot}\NormalTok{(acp3,}\AttributeTok{invisible=}\FunctionTok{c}\NormalTok{(}\StringTok{\textquotesingle{}quali\textquotesingle{}}\NormalTok{,}\StringTok{\textquotesingle{}ind.sup\textquotesingle{}}\NormalTok{),}\AttributeTok{habillage=}\DecValTok{7}\NormalTok{,}
     \AttributeTok{title=}\StringTok{"Graphe des individus de l\textquotesingle{}ACP"}\NormalTok{,}\AttributeTok{select=}\StringTok{\textquotesingle{}cos2  0.95\textquotesingle{}}\NormalTok{)}
\end{Highlighting}
\end{Shaded}

\includegraphics{Template_TD1_ACP_files/figure-latex/unnamed-chunk-12-1.pdf}

\begin{Shaded}
\begin{Highlighting}[]
\FunctionTok{plot}\NormalTok{(acp3,}\AttributeTok{invisible=}\FunctionTok{c}\NormalTok{(}\StringTok{\textquotesingle{}ind\textquotesingle{}}\NormalTok{,}\StringTok{\textquotesingle{}ind.sup\textquotesingle{}}\NormalTok{),}\AttributeTok{select=}\StringTok{\textquotesingle{}contrib  39\textquotesingle{}}\NormalTok{,}
     \AttributeTok{habillage=}\DecValTok{7}\NormalTok{,}\AttributeTok{title=}\StringTok{"Graphe des individus de l\textquotesingle{}ACP"}\NormalTok{)}
\end{Highlighting}
\end{Shaded}

\includegraphics{Template_TD1_ACP_files/figure-latex/unnamed-chunk-13-1.pdf}
L'Europe de l'Est et du Nord sont diamétralement opposées, d'après le
graphique. Les femmes de l'Europe de l'Est ont un plus grand taux de
fécondité lorsqu'elles sont jeunes. Contrairement aux femmes de l'Europe
du Nord.

\end{document}
