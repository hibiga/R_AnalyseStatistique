% Options for packages loaded elsewhere
\PassOptionsToPackage{unicode}{hyperref}
\PassOptionsToPackage{hyphens}{url}
%
\documentclass[
]{article}
\title{TD2 - SVD pour la compression et la réduction de dimension}
\author{Courtenay Rebecca \& Ducros Chloé \& Lasson Marie}
\date{}

\usepackage{amsmath,amssymb}
\usepackage{lmodern}
\usepackage{iftex}
\ifPDFTeX
  \usepackage[T1]{fontenc}
  \usepackage[utf8]{inputenc}
  \usepackage{textcomp} % provide euro and other symbols
\else % if luatex or xetex
  \usepackage{unicode-math}
  \defaultfontfeatures{Scale=MatchLowercase}
  \defaultfontfeatures[\rmfamily]{Ligatures=TeX,Scale=1}
\fi
% Use upquote if available, for straight quotes in verbatim environments
\IfFileExists{upquote.sty}{\usepackage{upquote}}{}
\IfFileExists{microtype.sty}{% use microtype if available
  \usepackage[]{microtype}
  \UseMicrotypeSet[protrusion]{basicmath} % disable protrusion for tt fonts
}{}
\makeatletter
\@ifundefined{KOMAClassName}{% if non-KOMA class
  \IfFileExists{parskip.sty}{%
    \usepackage{parskip}
  }{% else
    \setlength{\parindent}{0pt}
    \setlength{\parskip}{6pt plus 2pt minus 1pt}}
}{% if KOMA class
  \KOMAoptions{parskip=half}}
\makeatother
\usepackage{xcolor}
\IfFileExists{xurl.sty}{\usepackage{xurl}}{} % add URL line breaks if available
\IfFileExists{bookmark.sty}{\usepackage{bookmark}}{\usepackage{hyperref}}
\hypersetup{
  pdftitle={TD2 - SVD pour la compression et la réduction de dimension},
  pdfauthor={Courtenay Rebecca \& Ducros Chloé \& Lasson Marie},
  hidelinks,
  pdfcreator={LaTeX via pandoc}}
\urlstyle{same} % disable monospaced font for URLs
\usepackage[margin=1in]{geometry}
\usepackage{color}
\usepackage{fancyvrb}
\newcommand{\VerbBar}{|}
\newcommand{\VERB}{\Verb[commandchars=\\\{\}]}
\DefineVerbatimEnvironment{Highlighting}{Verbatim}{commandchars=\\\{\}}
% Add ',fontsize=\small' for more characters per line
\usepackage{framed}
\definecolor{shadecolor}{RGB}{248,248,248}
\newenvironment{Shaded}{\begin{snugshade}}{\end{snugshade}}
\newcommand{\AlertTok}[1]{\textcolor[rgb]{0.94,0.16,0.16}{#1}}
\newcommand{\AnnotationTok}[1]{\textcolor[rgb]{0.56,0.35,0.01}{\textbf{\textit{#1}}}}
\newcommand{\AttributeTok}[1]{\textcolor[rgb]{0.77,0.63,0.00}{#1}}
\newcommand{\BaseNTok}[1]{\textcolor[rgb]{0.00,0.00,0.81}{#1}}
\newcommand{\BuiltInTok}[1]{#1}
\newcommand{\CharTok}[1]{\textcolor[rgb]{0.31,0.60,0.02}{#1}}
\newcommand{\CommentTok}[1]{\textcolor[rgb]{0.56,0.35,0.01}{\textit{#1}}}
\newcommand{\CommentVarTok}[1]{\textcolor[rgb]{0.56,0.35,0.01}{\textbf{\textit{#1}}}}
\newcommand{\ConstantTok}[1]{\textcolor[rgb]{0.00,0.00,0.00}{#1}}
\newcommand{\ControlFlowTok}[1]{\textcolor[rgb]{0.13,0.29,0.53}{\textbf{#1}}}
\newcommand{\DataTypeTok}[1]{\textcolor[rgb]{0.13,0.29,0.53}{#1}}
\newcommand{\DecValTok}[1]{\textcolor[rgb]{0.00,0.00,0.81}{#1}}
\newcommand{\DocumentationTok}[1]{\textcolor[rgb]{0.56,0.35,0.01}{\textbf{\textit{#1}}}}
\newcommand{\ErrorTok}[1]{\textcolor[rgb]{0.64,0.00,0.00}{\textbf{#1}}}
\newcommand{\ExtensionTok}[1]{#1}
\newcommand{\FloatTok}[1]{\textcolor[rgb]{0.00,0.00,0.81}{#1}}
\newcommand{\FunctionTok}[1]{\textcolor[rgb]{0.00,0.00,0.00}{#1}}
\newcommand{\ImportTok}[1]{#1}
\newcommand{\InformationTok}[1]{\textcolor[rgb]{0.56,0.35,0.01}{\textbf{\textit{#1}}}}
\newcommand{\KeywordTok}[1]{\textcolor[rgb]{0.13,0.29,0.53}{\textbf{#1}}}
\newcommand{\NormalTok}[1]{#1}
\newcommand{\OperatorTok}[1]{\textcolor[rgb]{0.81,0.36,0.00}{\textbf{#1}}}
\newcommand{\OtherTok}[1]{\textcolor[rgb]{0.56,0.35,0.01}{#1}}
\newcommand{\PreprocessorTok}[1]{\textcolor[rgb]{0.56,0.35,0.01}{\textit{#1}}}
\newcommand{\RegionMarkerTok}[1]{#1}
\newcommand{\SpecialCharTok}[1]{\textcolor[rgb]{0.00,0.00,0.00}{#1}}
\newcommand{\SpecialStringTok}[1]{\textcolor[rgb]{0.31,0.60,0.02}{#1}}
\newcommand{\StringTok}[1]{\textcolor[rgb]{0.31,0.60,0.02}{#1}}
\newcommand{\VariableTok}[1]{\textcolor[rgb]{0.00,0.00,0.00}{#1}}
\newcommand{\VerbatimStringTok}[1]{\textcolor[rgb]{0.31,0.60,0.02}{#1}}
\newcommand{\WarningTok}[1]{\textcolor[rgb]{0.56,0.35,0.01}{\textbf{\textit{#1}}}}
\usepackage{graphicx}
\makeatletter
\def\maxwidth{\ifdim\Gin@nat@width>\linewidth\linewidth\else\Gin@nat@width\fi}
\def\maxheight{\ifdim\Gin@nat@height>\textheight\textheight\else\Gin@nat@height\fi}
\makeatother
% Scale images if necessary, so that they will not overflow the page
% margins by default, and it is still possible to overwrite the defaults
% using explicit options in \includegraphics[width, height, ...]{}
\setkeys{Gin}{width=\maxwidth,height=\maxheight,keepaspectratio}
% Set default figure placement to htbp
\makeatletter
\def\fps@figure{htbp}
\makeatother
\setlength{\emergencystretch}{3em} % prevent overfull lines
\providecommand{\tightlist}{%
  \setlength{\itemsep}{0pt}\setlength{\parskip}{0pt}}
\setcounter{secnumdepth}{5}
\ifLuaTeX
  \usepackage{selnolig}  % disable illegal ligatures
\fi

\begin{document}
\maketitle

{
\setcounter{tocdepth}{2}
\tableofcontents
}
\hypertarget{introduction}{%
\section{Introduction}\label{introduction}}

Dans ce TD, on procède à une compression d'image en utilisant la SVD.
Vous retrouverez toutes les fonctions dont vous aurez besoin dans la
partie 2.11 du cours (page 14 du PDF).

Voici en plus quelques autres fonctions utiles.

\hypertarget{chargement-de-la-librairie}{%
\subsection{Chargement de la
librairie}\label{chargement-de-la-librairie}}

\begin{Shaded}
\begin{Highlighting}[]
\FunctionTok{library}\NormalTok{(jpeg)}
\end{Highlighting}
\end{Shaded}

\hypertarget{tracer-des-graphes}{%
\subsection{Tracer des graphes}\label{tracer-des-graphes}}

Afficher les 8 premières valeurs singulières dans un graphe:

\begin{Shaded}
\begin{Highlighting}[]
\NormalTok{singular\_values }\OtherTok{\textless{}{-}} \FunctionTok{c}\NormalTok{(}\DecValTok{610}\NormalTok{,}\DecValTok{120}\NormalTok{,}\DecValTok{90}\NormalTok{,}\DecValTok{68}\NormalTok{,}\DecValTok{55}\NormalTok{,}\DecValTok{47}\NormalTok{,}\DecValTok{39}\NormalTok{,}\DecValTok{34}\NormalTok{,}\DecValTok{32}\NormalTok{,}\DecValTok{28}\NormalTok{)}
\FunctionTok{plot}\NormalTok{(singular\_values[}\DecValTok{1}\SpecialCharTok{:}\DecValTok{8}\NormalTok{], }\AttributeTok{main=}\StringTok{"8 first Singular values"}\NormalTok{)}
\end{Highlighting}
\end{Shaded}

\includegraphics{Sujet_TD2_ACP_files/figure-latex/unnamed-chunk-2-1.pdf}

A partir de valeurs singulières, tracer le pourcentage cumulatif de leur
impact :

\begin{Shaded}
\begin{Highlighting}[]
\NormalTok{singular\_values }\OtherTok{\textless{}{-}} \FunctionTok{c}\NormalTok{(}\DecValTok{610}\NormalTok{,}\DecValTok{120}\NormalTok{,}\DecValTok{90}\NormalTok{,}\DecValTok{68}\NormalTok{,}\DecValTok{55}\NormalTok{,}\DecValTok{47}\NormalTok{,}\DecValTok{39}\NormalTok{,}\DecValTok{34}\NormalTok{,}\DecValTok{32}\NormalTok{,}\DecValTok{28}\NormalTok{)}
\FunctionTok{plot}\NormalTok{(}\FunctionTok{cumsum}\NormalTok{(singular\_values) }\SpecialCharTok{/} \FunctionTok{sum}\NormalTok{(singular\_values), }
     \AttributeTok{main=}\StringTok{"Cumulative Percent of Singular values"}\NormalTok{)}
\end{Highlighting}
\end{Shaded}

\includegraphics{Sujet_TD2_ACP_files/figure-latex/unnamed-chunk-3-1.pdf}

Pour tracer des graphes ou des images sur \emph{n} lignes et \emph{p}
colonnes on utilise la fonction \texttt{par(mfrow\ =\ c(n,p))} suivie de
tous les appels à \texttt{plot}.

\begin{Shaded}
\begin{Highlighting}[]
\FunctionTok{par}\NormalTok{(}\AttributeTok{mfrow=}\FunctionTok{c}\NormalTok{(}\DecValTok{1}\NormalTok{,}\DecValTok{2}\NormalTok{))}
\FunctionTok{plot.image}\NormalTok{(PIC)}
\FunctionTok{plot.image}\NormalTok{(refuge\_ossau)}
\end{Highlighting}
\end{Shaded}

\includegraphics{Sujet_TD2_ACP_files/figure-latex/unnamed-chunk-6-1.pdf}

\hypertarget{fonction-svd}{%
\subsection{Fonction SVD}\label{fonction-svd}}

La fonction \texttt{svd} de R retourne un triplé \emph{(d,u,v)} avec
\emph{d} le vecteur des valeurs singulières, et \emph{u} et \emph{v} les
deux matrices orthogonales de gauche et de droite. Pour accéder par
exemple au vecteur des valeurs singulières, il faut donc utiliser
\texttt{X.svd\$d} en supposant qu'on a stocké le résultat de la svd dans
la variable \texttt{X.svd}.

\hypertarget{questions}{%
\section{Questions}\label{questions}}

L'ensemble des questions vaut 18 points, en plus des 2 points bonus de
la question 7. La qualité du rapport sera notée sur 4 points: vous aurez
ces points par exemple en labellisant vos graphes et leurs axes, en
faisant un affichage correct etc. Je vous invite donc à chercher sur
Internet les différents paramètres possibles pour la fonction
\texttt{plot}.

\hypertarget{question-1}{%
\subsection{Question 1}\label{question-1}}

Importer l'image du pic du midi d'Ossau, en faire la SVD, puis tracer
côte à côte l'image originale et l'image issue de la compression avec 40
composantes.

\hypertarget{svd}{%
\subsubsection{SVD}\label{svd}}

\begin{Shaded}
\begin{Highlighting}[]
\NormalTok{r }\OtherTok{\textless{}{-}}\NormalTok{ PIC[, , }\DecValTok{1}\NormalTok{] }\CommentTok{\#rouge}
\NormalTok{g }\OtherTok{\textless{}{-}}\NormalTok{ PIC[, , }\DecValTok{2}\NormalTok{] }\CommentTok{\#vert}
\NormalTok{b }\OtherTok{\textless{}{-}}\NormalTok{ PIC[, , }\DecValTok{3}\NormalTok{] }\CommentTok{\#bleu}
\CommentTok{\#SVD pour chaque bande de couleurs}
\NormalTok{PIC\_R\_SVD }\OtherTok{\textless{}{-}} \FunctionTok{svd}\NormalTok{(r)}
\NormalTok{PIC\_G\_SVD }\OtherTok{\textless{}{-}} \FunctionTok{svd}\NormalTok{(g)}
\NormalTok{PIC\_B\_SVD }\OtherTok{\textless{}{-}} \FunctionTok{svd}\NormalTok{(b)}
\NormalTok{SVD }\OtherTok{\textless{}{-}} \FunctionTok{list}\NormalTok{(PIC\_R\_SVD, PIC\_G\_SVD, PIC\_B\_SVD)}
\end{Highlighting}
\end{Shaded}

\hypertarget{image-et-compression}{%
\subsubsection{Image et compression}\label{image-et-compression}}

\begin{Shaded}
\begin{Highlighting}[]
\NormalTok{PLOT\_IMG }\OtherTok{\textless{}{-}} \ControlFlowTok{function}\NormalTok{(pic, }\AttributeTok{main =} \StringTok{""}\NormalTok{) \{}
\NormalTok{  h }\OtherTok{\textless{}{-}} \FunctionTok{dim}\NormalTok{(pic)[}\DecValTok{1}\NormalTok{] }
\NormalTok{  w }\OtherTok{\textless{}{-}} \FunctionTok{dim}\NormalTok{(pic)[}\DecValTok{2}\NormalTok{]}
  \FunctionTok{plot}\NormalTok{(}\AttributeTok{x =} \FunctionTok{c}\NormalTok{(}\DecValTok{0}\NormalTok{, h), }\AttributeTok{y =} \FunctionTok{c}\NormalTok{(}\DecValTok{0}\NormalTok{, w), }\AttributeTok{type =} \StringTok{"n"}\NormalTok{, }\AttributeTok{main =}\NormalTok{ main)}
  \FunctionTok{rasterImage}\NormalTok{(pic, }\DecValTok{0}\NormalTok{, }\DecValTok{0}\NormalTok{, h, w)}
\NormalTok{\}}


\NormalTok{VP }\OtherTok{\textless{}{-}} \ControlFlowTok{function}\NormalTok{(i) \{}
  \FunctionTok{list}\NormalTok{(}\AttributeTok{d =}\NormalTok{ i}\SpecialCharTok{$}\NormalTok{d[}\DecValTok{1}\SpecialCharTok{:}\DecValTok{40}\NormalTok{], }\CommentTok{\#racine carrée de valeurs propres [valeurs singulieres]}
       \AttributeTok{u =}\NormalTok{ i}\SpecialCharTok{$}\NormalTok{u[, }\DecValTok{1}\SpecialCharTok{:}\DecValTok{40}\NormalTok{], }\CommentTok{\#vecteurs propres de droite}
       \AttributeTok{v =}\NormalTok{ i}\SpecialCharTok{$}\NormalTok{v[, }\DecValTok{1}\SpecialCharTok{:}\DecValTok{40}\NormalTok{]) }\CommentTok{\#vecteurs propres de gauche}
\NormalTok{\}}

\NormalTok{COMPRESS\_IMG }\OtherTok{\textless{}{-}} \ControlFlowTok{function}\NormalTok{(i) \{}
\NormalTok{    img.compressed }\OtherTok{\textless{}{-}}\NormalTok{ i}\SpecialCharTok{$}\NormalTok{u }\SpecialCharTok{\%*\%} \FunctionTok{diag}\NormalTok{(i}\SpecialCharTok{$}\NormalTok{d) }\SpecialCharTok{\%*\%} \FunctionTok{t}\NormalTok{(i}\SpecialCharTok{$}\NormalTok{v)}
\NormalTok{\}}

\NormalTok{IMG }\OtherTok{\textless{}{-}} \ControlFlowTok{function}\NormalTok{(svd) \{}
\NormalTok{  dim\_mini }\OtherTok{\textless{}{-}} \FunctionTok{lapply}\NormalTok{(svd, VP)}
\NormalTok{  img }\OtherTok{\textless{}{-}} \FunctionTok{sapply}\NormalTok{(dim\_mini, COMPRESS\_IMG, }\AttributeTok{simplify =} \StringTok{\textquotesingle{}array\textquotesingle{}}\NormalTok{)}
\NormalTok{  img[img }\SpecialCharTok{\textless{}} \DecValTok{0}\NormalTok{] }\OtherTok{\textless{}{-}} \DecValTok{0}
\NormalTok{  img[img }\SpecialCharTok{\textgreater{}} \DecValTok{1}\NormalTok{] }\OtherTok{\textless{}{-}} \DecValTok{1}
  \FunctionTok{return}\NormalTok{(}\FunctionTok{list}\NormalTok{(}\AttributeTok{img =}\NormalTok{ img, }\AttributeTok{svd.reduced =}\NormalTok{ dim\_mini))}
\NormalTok{\}}
\end{Highlighting}
\end{Shaded}

\hypertarget{graphique-des-deux-images}{%
\subsubsection{Graphique des deux
images}\label{graphique-des-deux-images}}

\begin{Shaded}
\begin{Highlighting}[]
\FunctionTok{par}\NormalTok{(}\AttributeTok{mfrow =} \FunctionTok{c}\NormalTok{(}\DecValTok{1}\NormalTok{, }\DecValTok{2}\NormalTok{))}
\FunctionTok{PLOT\_IMG}\NormalTok{(PIC, }\StringTok{"Image d\textquotesingle{}origine"}\NormalTok{)}
\FunctionTok{PLOT\_IMG}\NormalTok{(}\FunctionTok{IMG}\NormalTok{(SVD)}\SpecialCharTok{$}\NormalTok{img,}
\FunctionTok{paste}\NormalTok{(}\StringTok{"SVD avec 40 composantes"}\NormalTok{))}
\end{Highlighting}
\end{Shaded}

\includegraphics{Sujet_TD2_ACP_files/figure-latex/unnamed-chunk-9-1.pdf}

\hypertarget{question-2}{%
\subsection{Question 2}\label{question-2}}

Tracer le graphe du pourcentage cumulatif des valeurs singulières pour
la bande rouge. Avec environ combien de composantes transmet-on
(environ) 80\% de l'information de l'image initiale sur la bande rouge ?

\begin{Shaded}
\begin{Highlighting}[]
\FunctionTok{plot}\NormalTok{(}\FunctionTok{cumsum}\NormalTok{(PIC\_R\_SVD}\SpecialCharTok{$}\NormalTok{d) }\SpecialCharTok{/} \FunctionTok{sum}\NormalTok{(PIC\_R\_SVD}\SpecialCharTok{$}\NormalTok{d), }
     \AttributeTok{main=}\StringTok{"Pourcentage cumulatif des valeurs singulières pour la bande rouge"}\NormalTok{, }
     \AttributeTok{ylab=}\StringTok{"Pourcentage"}\NormalTok{, }\AttributeTok{xlab=}\StringTok{"Nombres de composantes"}\NormalTok{, }\AttributeTok{pch=}\DecValTok{1}\NormalTok{)}
\FunctionTok{abline}\NormalTok{(}\AttributeTok{h=}\FloatTok{0.80}\NormalTok{, }\AttributeTok{col=}\StringTok{"red"}\NormalTok{)}
\FunctionTok{abline}\NormalTok{(}\AttributeTok{v=}\DecValTok{300}\NormalTok{, }\AttributeTok{col=}\StringTok{"red"}\NormalTok{)}
\end{Highlighting}
\end{Shaded}

\includegraphics{Sujet_TD2_ACP_files/figure-latex/unnamed-chunk-10-1.pdf}
Il faut environ 300 composantes pour transmettre (environ) 80\% de
l'information de l'image initiale sur la bande rouge.

\hypertarget{question-3}{%
\subsection{Question 3}\label{question-3}}

Pour la bande verte, combien de pourcentage de l'information est
représenté exactement avec la première composante (i.e.~quel est le
poids de la première valeur singulière par rapport au total) ? Avec la
deuxième ?

\begin{Shaded}
\begin{Highlighting}[]
\FunctionTok{plot}\NormalTok{((PIC\_G\_SVD}\SpecialCharTok{$}\NormalTok{d)[}\DecValTok{1}\SpecialCharTok{:}\DecValTok{2}\NormalTok{] }\SpecialCharTok{/} \FunctionTok{sum}\NormalTok{(PIC\_G\_SVD}\SpecialCharTok{$}\NormalTok{d), }
     \AttributeTok{main=}\StringTok{"Pourcentage cumulatif des valeurs singulières 1 et 2"}\NormalTok{, }
     \AttributeTok{ylab=}\StringTok{"Pourcentage cumulé"}\NormalTok{, }\AttributeTok{xlab=}\StringTok{"Composantes"}\NormalTok{, }\AttributeTok{pch=}\DecValTok{19}\NormalTok{)}
\end{Highlighting}
\end{Shaded}

\includegraphics{Sujet_TD2_ACP_files/figure-latex/unnamed-chunk-11-1.pdf}
Pour la 1er valeur singulière, son poids par rapport au total est de
25\%. Pour la deuxieme valeur singulière, son poids est d'environ 2\%.

\hypertarget{question-4}{%
\subsection{Question 4}\label{question-4}}

Afficher sur un graphique côte à côte (les trois sur la même ligne), les
valeurs des 50 premières valeurs singulières pour chacune des trois
bandes (RGB). Donner un titre à chaque graphe et labelliser les axes.

\begin{Shaded}
\begin{Highlighting}[]
\FunctionTok{plot}\NormalTok{(PIC\_G\_SVD}\SpecialCharTok{$}\NormalTok{d[}\DecValTok{1}\SpecialCharTok{:}\DecValTok{50}\NormalTok{],}\AttributeTok{pch=}\DecValTok{2}\NormalTok{, }\AttributeTok{col=}\StringTok{"chartreuse4"}\NormalTok{,}\AttributeTok{ylab=}\StringTok{"Valeurs"}\NormalTok{,}\AttributeTok{xlab=}\StringTok{"Valeurs singulières"}\NormalTok{,}
     \AttributeTok{ylim=}\FunctionTok{c}\NormalTok{(}\DecValTok{0}\NormalTok{,}\DecValTok{1000}\NormalTok{),}\AttributeTok{main=}\StringTok{"50 premieres valeurs pour les 3 bandes"}\NormalTok{)}
\FunctionTok{par}\NormalTok{(}\AttributeTok{new =}\NormalTok{ T)  }
\FunctionTok{plot}\NormalTok{(PIC\_R\_SVD}\SpecialCharTok{$}\NormalTok{d[}\DecValTok{1}\SpecialCharTok{:}\DecValTok{50}\NormalTok{],}\AttributeTok{pch=}\DecValTok{3}\NormalTok{, }\AttributeTok{col=}\StringTok{"brown3"}\NormalTok{,}\AttributeTok{ylab=}\StringTok{""}\NormalTok{,}\AttributeTok{xlab=}\StringTok{""}\NormalTok{,}\AttributeTok{ylim=}\FunctionTok{c}\NormalTok{(}\DecValTok{0}\NormalTok{,}\DecValTok{1000}\NormalTok{))}
\FunctionTok{par}\NormalTok{(}\AttributeTok{new =}\NormalTok{ T)  }
\FunctionTok{plot}\NormalTok{(PIC\_B\_SVD}\SpecialCharTok{$}\NormalTok{d[}\DecValTok{1}\SpecialCharTok{:}\DecValTok{50}\NormalTok{],}\AttributeTok{pch=}\DecValTok{4}\NormalTok{, }\AttributeTok{col=}\StringTok{"darkblue"}\NormalTok{,}\AttributeTok{ylab=}\StringTok{""}\NormalTok{,}\AttributeTok{xlab=}\StringTok{""}\NormalTok{,}\AttributeTok{ylim=}\FunctionTok{c}\NormalTok{(}\DecValTok{0}\NormalTok{,}\DecValTok{1000}\NormalTok{))}
\FunctionTok{legend}\NormalTok{(}\AttributeTok{x=}\StringTok{"topright"}\NormalTok{,}
       \AttributeTok{legend=}\FunctionTok{c}\NormalTok{(}\StringTok{"Valeurs {-} bande verte"}\NormalTok{,}\StringTok{"Valeurs {-} bande rouge"}\NormalTok{,}\StringTok{"Valeurs {-} bande bleue"}\NormalTok{),}
       \AttributeTok{pch=}\FunctionTok{c}\NormalTok{(}\DecValTok{2}\NormalTok{,}\DecValTok{3}\NormalTok{,}\DecValTok{4}\NormalTok{),}\AttributeTok{col=}\FunctionTok{c}\NormalTok{(}\StringTok{"chartreuse4"}\NormalTok{,}\StringTok{"brown3"}\NormalTok{,}\StringTok{"darkblue"}\NormalTok{),}
       \AttributeTok{text.col=}\FunctionTok{c}\NormalTok{(}\StringTok{"chartreuse4"}\NormalTok{,}\StringTok{"brown3"}\NormalTok{,}\StringTok{"darkblue"}\NormalTok{))}
\end{Highlighting}
\end{Shaded}

\includegraphics{Sujet_TD2_ACP_files/figure-latex/unnamed-chunk-12-1.pdf}

\hypertarget{question-5}{%
\subsection{Question 5}\label{question-5}}

Afficher sur un graphique côte à côte (les trois sur la même ligne), le
pourcentage cumulatif des 100 premières valeurs singulières pour chacune
des trois bandes (RGB). Donner un titre à chaque graphe et labelliser
les axes.

\begin{Shaded}
\begin{Highlighting}[]
\FunctionTok{plot}\NormalTok{(}\FunctionTok{cumsum}\NormalTok{(PIC\_G\_SVD}\SpecialCharTok{$}\NormalTok{d)[}\DecValTok{1}\SpecialCharTok{:}\DecValTok{50}\NormalTok{] }\SpecialCharTok{/} \FunctionTok{sum}\NormalTok{(PIC\_G\_SVD}\SpecialCharTok{$}\NormalTok{d), }\AttributeTok{pch=}\DecValTok{2}\NormalTok{, }\AttributeTok{col=}\StringTok{"chartreuse4"}\NormalTok{,}
     \AttributeTok{ylab=}\StringTok{"Pourcentage cumulé"}\NormalTok{,}\AttributeTok{xlab=}\StringTok{"Valeurs singulières"}\NormalTok{,}\AttributeTok{ylim=}\FunctionTok{c}\NormalTok{(}\DecValTok{0}\NormalTok{,}\FloatTok{0.6}\NormalTok{),}
     \AttributeTok{main=}\StringTok{"Pourcentage cumulé des 50 premieres valeurs pour les 3 bandes"}\NormalTok{)}
\FunctionTok{par}\NormalTok{(}\AttributeTok{new =}\NormalTok{ T)  }
\FunctionTok{plot}\NormalTok{(}\FunctionTok{cumsum}\NormalTok{(PIC\_R\_SVD}\SpecialCharTok{$}\NormalTok{d)[}\DecValTok{1}\SpecialCharTok{:}\DecValTok{50}\NormalTok{] }\SpecialCharTok{/} \FunctionTok{sum}\NormalTok{(PIC\_R\_SVD}\SpecialCharTok{$}\NormalTok{d),}\AttributeTok{pch=}\DecValTok{3}\NormalTok{, }\AttributeTok{col=}\StringTok{"brown3"}\NormalTok{,}
     \AttributeTok{ylab=}\StringTok{""}\NormalTok{,}\AttributeTok{xlab=}\StringTok{""}\NormalTok{,}\AttributeTok{ylim=}\FunctionTok{c}\NormalTok{(}\DecValTok{0}\NormalTok{,}\FloatTok{0.6}\NormalTok{))}
\FunctionTok{par}\NormalTok{(}\AttributeTok{new =}\NormalTok{ T)  }
\FunctionTok{plot}\NormalTok{(}\FunctionTok{cumsum}\NormalTok{(PIC\_B\_SVD}\SpecialCharTok{$}\NormalTok{d)[}\DecValTok{1}\SpecialCharTok{:}\DecValTok{50}\NormalTok{] }\SpecialCharTok{/} \FunctionTok{sum}\NormalTok{(PIC\_B\_SVD}\SpecialCharTok{$}\NormalTok{d),}\AttributeTok{pch=}\DecValTok{4}\NormalTok{, }\AttributeTok{col=}\StringTok{"darkblue"}\NormalTok{,}
     \AttributeTok{ylab=}\StringTok{""}\NormalTok{,}\AttributeTok{xlab=}\StringTok{""}\NormalTok{,}\AttributeTok{ylim=}\FunctionTok{c}\NormalTok{(}\DecValTok{0}\NormalTok{,}\FloatTok{0.6}\NormalTok{))}
\FunctionTok{legend}\NormalTok{(}\AttributeTok{x=}\StringTok{"bottomright"}\NormalTok{,}
       \AttributeTok{legend=}\FunctionTok{c}\NormalTok{(}\StringTok{"Pourcentage cumulé {-} bande verte"}\NormalTok{,}
                \StringTok{"Pourcentage cumulé {-} bande rouge"}\NormalTok{,}
                \StringTok{"Pourcentage cumulé {-} bande bleue"}\NormalTok{),}
       \AttributeTok{pch=}\FunctionTok{c}\NormalTok{(}\DecValTok{2}\NormalTok{,}\DecValTok{3}\NormalTok{,}\DecValTok{4}\NormalTok{),}\AttributeTok{col=}\FunctionTok{c}\NormalTok{(}\StringTok{"chartreuse4"}\NormalTok{,}\StringTok{"brown3"}\NormalTok{,}\StringTok{"darkblue"}\NormalTok{),}
       \AttributeTok{text.col=}\FunctionTok{c}\NormalTok{(}\StringTok{"chartreuse4"}\NormalTok{,}\StringTok{"brown3"}\NormalTok{,}\StringTok{"darkblue"}\NormalTok{))}
\end{Highlighting}
\end{Shaded}

\includegraphics{Sujet_TD2_ACP_files/figure-latex/unnamed-chunk-13-1.pdf}

\hypertarget{question-6}{%
\subsection{Question 6}\label{question-6}}

Tracer un graphe sur lequel apparaît le poids de l'image (en MB) en
fonction du nombre de composantes sélectionnées. Le tracer avec un pas
de 20, pour les 400 premières composantes.

\begin{Shaded}
\begin{Highlighting}[]
\NormalTok{vecteur }\OtherTok{\textless{}{-}} \FunctionTok{c}\NormalTok{()}

\NormalTok{IMG\_BIS }\OtherTok{\textless{}{-}} \ControlFlowTok{function}\NormalTok{(svd, nb.comp) \{}
\NormalTok{  dim\_mini }\OtherTok{\textless{}{-}} \FunctionTok{lapply}\NormalTok{(svd, }\ControlFlowTok{function}\NormalTok{(i) }\FunctionTok{list}\NormalTok{(}\AttributeTok{d =}\NormalTok{ i}\SpecialCharTok{$}\NormalTok{d[}\DecValTok{1}\SpecialCharTok{:}\NormalTok{nb.comp],}
                                           \AttributeTok{u =}\NormalTok{ i}\SpecialCharTok{$}\NormalTok{u[, }\DecValTok{1}\SpecialCharTok{:}\NormalTok{nb.comp],}
                                           \AttributeTok{v =}\NormalTok{ i}\SpecialCharTok{$}\NormalTok{v[, }\DecValTok{1}\SpecialCharTok{:}\NormalTok{nb.comp]))}
\NormalTok{  img }\OtherTok{\textless{}{-}} \FunctionTok{sapply}\NormalTok{(dim\_mini, COMPRESS\_IMG, }\AttributeTok{simplify =} \StringTok{\textquotesingle{}array\textquotesingle{}}\NormalTok{)}
\NormalTok{  img[img }\SpecialCharTok{\textless{}} \DecValTok{0}\NormalTok{] }\OtherTok{\textless{}{-}} \DecValTok{0}
\NormalTok{  img[img }\SpecialCharTok{\textgreater{}} \DecValTok{1}\NormalTok{] }\OtherTok{\textless{}{-}} \DecValTok{1}
  \FunctionTok{return}\NormalTok{(}\FunctionTok{list}\NormalTok{(}\AttributeTok{img =}\NormalTok{ img, }\AttributeTok{svd.reduced =}\NormalTok{ dim\_mini))}
\NormalTok{\}}

\ControlFlowTok{for}\NormalTok{ (index }\ControlFlowTok{in} \FunctionTok{seq.int}\NormalTok{(}\DecValTok{20}\NormalTok{, }\DecValTok{400}\NormalTok{, }\DecValTok{20}\NormalTok{))\{}
\NormalTok{  img\_size }\OtherTok{\textless{}{-}} \FunctionTok{object.size}\NormalTok{(}\FunctionTok{writeJPEG}\NormalTok{(}\FunctionTok{IMG\_BIS}\NormalTok{(SVD, index)}\SpecialCharTok{$}\NormalTok{img))}
\NormalTok{  vecteur }\OtherTok{\textless{}{-}} \FunctionTok{append}\NormalTok{(vecteur,img\_size, index)}
\NormalTok{\}}
\FunctionTok{plot.new}\NormalTok{()}
\FunctionTok{plot}\NormalTok{(vecteur, }\AttributeTok{main=}\StringTok{"Poids de l\textquotesingle{}image en fonction du nombre de composantes"}\NormalTok{, }
     \AttributeTok{xlab=}\StringTok{"Composantes de 20 en 20"}\NormalTok{,}\AttributeTok{ylab=}\StringTok{"Poids de l\textquotesingle{}image"}\NormalTok{)}
\end{Highlighting}
\end{Shaded}

\includegraphics{Sujet_TD2_ACP_files/figure-latex/unnamed-chunk-14-1.pdf}

\hypertarget{question-bonus}{%
\subsection{Question BONUS}\label{question-bonus}}

Imaginons le scénario suivant : nous avons un satellite qui prend un
certain nombre de photos par jour de la Terre, et qui transmet tous les
jours ses données à un datacenter sur Terre. Il ne peut envoyer que 8GB
de données par jour. Pour pouvoir envoyer plus d'images, on estime qu'on
peut envoyer une image qui retransmet 50\% (environ) de l'information
contenue dans l'image initiale. Comparer le nombre d'images à pleine
résolution au nombre d'images à 50\% qu'il est possible d'envoyer. On
admettra qu'une compression par SVD sur l'image satellite a le même
impact qu'une compression sur l'image du pic du Midi d'Ossau, et on
admettra que la taille des images est similaire. Pour répondre à cette
question, vous aurez donc besoin de déterminer le nombre de composantes
à utiliser pour parvenir à une résolution d'environ 50\%, puis de
compresser l'image et de comparer sa taille avec l'image initiale. Vous
pourrez vous aider des graphiques de la question 5).

En nous aidant de la question 5, on remarque qu'en prenant 50
composantes, on peut avoir 50\% de l'information, toutes bandes de
couleur confondues.

\hypertarget{tailles}{%
\subsubsection{Tailles}\label{tailles}}

\begin{Shaded}
\begin{Highlighting}[]
\NormalTok{img\_size }\OtherTok{\textless{}{-}} \FunctionTok{object.size}\NormalTok{(}\FunctionTok{writeJPEG}\NormalTok{(}\FunctionTok{IMG\_BIS}\NormalTok{(SVD, }\DecValTok{50}\NormalTok{)}\SpecialCharTok{$}\NormalTok{img))}
\NormalTok{img\_size}
\end{Highlighting}
\end{Shaded}

\begin{verbatim}
## 178904 bytes
\end{verbatim}

\begin{Shaded}
\begin{Highlighting}[]
\NormalTok{pic\_size }\OtherTok{\textless{}{-}} \FunctionTok{object.size}\NormalTok{(PIC)}
\NormalTok{pic\_size}
\end{Highlighting}
\end{Shaded}

\begin{verbatim}
## 46080224 bytes
\end{verbatim}

\hypertarget{pour-8gb}{%
\subsubsection{Pour 8GB}\label{pour-8gb}}

De ce fait : 8GB = 8000000000 Bytes

\begin{Shaded}
\begin{Highlighting}[]
\NormalTok{img\_mem }\OtherTok{\textless{}{-}} \DecValTok{8000000000}\SpecialCharTok{/}\NormalTok{img\_size}
\NormalTok{img\_mem}
\end{Highlighting}
\end{Shaded}

\begin{verbatim}
## 44716.7 bytes
\end{verbatim}

\begin{Shaded}
\begin{Highlighting}[]
\NormalTok{pic\_mem }\OtherTok{\textless{}{-}} \DecValTok{8000000000}\SpecialCharTok{/}\NormalTok{pic\_size}
\NormalTok{pic\_mem}
\end{Highlighting}
\end{Shaded}

\begin{verbatim}
## 173.6 bytes
\end{verbatim}

\begin{Shaded}
\begin{Highlighting}[]
\NormalTok{img\_mem }\SpecialCharTok{{-}}\NormalTok{ pic\_mem}
\end{Highlighting}
\end{Shaded}

\begin{verbatim}
## 44543.1 bytes
\end{verbatim}

On remarque une grande différence entre les deux mémoires. De ce fait,
nous pouvons donc passer de 173 images à 44 716 images en prenant
uniquement 50 composantes.

\end{document}
