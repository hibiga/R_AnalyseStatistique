% Options for packages loaded elsewhere
\PassOptionsToPackage{unicode}{hyperref}
\PassOptionsToPackage{hyphens}{url}
%
\documentclass[
]{article}
\title{TD5 -Partial Least Squares - Discriminant Analysis}
\author{Courtenay Rebecca \& Ducros Chloé \& Lasson Marie}
\date{}

\usepackage{amsmath,amssymb}
\usepackage{lmodern}
\usepackage{iftex}
\ifPDFTeX
  \usepackage[T1]{fontenc}
  \usepackage[utf8]{inputenc}
  \usepackage{textcomp} % provide euro and other symbols
\else % if luatex or xetex
  \usepackage{unicode-math}
  \defaultfontfeatures{Scale=MatchLowercase}
  \defaultfontfeatures[\rmfamily]{Ligatures=TeX,Scale=1}
\fi
% Use upquote if available, for straight quotes in verbatim environments
\IfFileExists{upquote.sty}{\usepackage{upquote}}{}
\IfFileExists{microtype.sty}{% use microtype if available
  \usepackage[]{microtype}
  \UseMicrotypeSet[protrusion]{basicmath} % disable protrusion for tt fonts
}{}
\makeatletter
\@ifundefined{KOMAClassName}{% if non-KOMA class
  \IfFileExists{parskip.sty}{%
    \usepackage{parskip}
  }{% else
    \setlength{\parindent}{0pt}
    \setlength{\parskip}{6pt plus 2pt minus 1pt}}
}{% if KOMA class
  \KOMAoptions{parskip=half}}
\makeatother
\usepackage{xcolor}
\IfFileExists{xurl.sty}{\usepackage{xurl}}{} % add URL line breaks if available
\IfFileExists{bookmark.sty}{\usepackage{bookmark}}{\usepackage{hyperref}}
\hypersetup{
  pdftitle={TD5 -Partial Least Squares - Discriminant Analysis},
  pdfauthor={Courtenay Rebecca \& Ducros Chloé \& Lasson Marie},
  hidelinks,
  pdfcreator={LaTeX via pandoc}}
\urlstyle{same} % disable monospaced font for URLs
\usepackage[margin=1in]{geometry}
\usepackage{color}
\usepackage{fancyvrb}
\newcommand{\VerbBar}{|}
\newcommand{\VERB}{\Verb[commandchars=\\\{\}]}
\DefineVerbatimEnvironment{Highlighting}{Verbatim}{commandchars=\\\{\}}
% Add ',fontsize=\small' for more characters per line
\usepackage{framed}
\definecolor{shadecolor}{RGB}{248,248,248}
\newenvironment{Shaded}{\begin{snugshade}}{\end{snugshade}}
\newcommand{\AlertTok}[1]{\textcolor[rgb]{0.94,0.16,0.16}{#1}}
\newcommand{\AnnotationTok}[1]{\textcolor[rgb]{0.56,0.35,0.01}{\textbf{\textit{#1}}}}
\newcommand{\AttributeTok}[1]{\textcolor[rgb]{0.77,0.63,0.00}{#1}}
\newcommand{\BaseNTok}[1]{\textcolor[rgb]{0.00,0.00,0.81}{#1}}
\newcommand{\BuiltInTok}[1]{#1}
\newcommand{\CharTok}[1]{\textcolor[rgb]{0.31,0.60,0.02}{#1}}
\newcommand{\CommentTok}[1]{\textcolor[rgb]{0.56,0.35,0.01}{\textit{#1}}}
\newcommand{\CommentVarTok}[1]{\textcolor[rgb]{0.56,0.35,0.01}{\textbf{\textit{#1}}}}
\newcommand{\ConstantTok}[1]{\textcolor[rgb]{0.00,0.00,0.00}{#1}}
\newcommand{\ControlFlowTok}[1]{\textcolor[rgb]{0.13,0.29,0.53}{\textbf{#1}}}
\newcommand{\DataTypeTok}[1]{\textcolor[rgb]{0.13,0.29,0.53}{#1}}
\newcommand{\DecValTok}[1]{\textcolor[rgb]{0.00,0.00,0.81}{#1}}
\newcommand{\DocumentationTok}[1]{\textcolor[rgb]{0.56,0.35,0.01}{\textbf{\textit{#1}}}}
\newcommand{\ErrorTok}[1]{\textcolor[rgb]{0.64,0.00,0.00}{\textbf{#1}}}
\newcommand{\ExtensionTok}[1]{#1}
\newcommand{\FloatTok}[1]{\textcolor[rgb]{0.00,0.00,0.81}{#1}}
\newcommand{\FunctionTok}[1]{\textcolor[rgb]{0.00,0.00,0.00}{#1}}
\newcommand{\ImportTok}[1]{#1}
\newcommand{\InformationTok}[1]{\textcolor[rgb]{0.56,0.35,0.01}{\textbf{\textit{#1}}}}
\newcommand{\KeywordTok}[1]{\textcolor[rgb]{0.13,0.29,0.53}{\textbf{#1}}}
\newcommand{\NormalTok}[1]{#1}
\newcommand{\OperatorTok}[1]{\textcolor[rgb]{0.81,0.36,0.00}{\textbf{#1}}}
\newcommand{\OtherTok}[1]{\textcolor[rgb]{0.56,0.35,0.01}{#1}}
\newcommand{\PreprocessorTok}[1]{\textcolor[rgb]{0.56,0.35,0.01}{\textit{#1}}}
\newcommand{\RegionMarkerTok}[1]{#1}
\newcommand{\SpecialCharTok}[1]{\textcolor[rgb]{0.00,0.00,0.00}{#1}}
\newcommand{\SpecialStringTok}[1]{\textcolor[rgb]{0.31,0.60,0.02}{#1}}
\newcommand{\StringTok}[1]{\textcolor[rgb]{0.31,0.60,0.02}{#1}}
\newcommand{\VariableTok}[1]{\textcolor[rgb]{0.00,0.00,0.00}{#1}}
\newcommand{\VerbatimStringTok}[1]{\textcolor[rgb]{0.31,0.60,0.02}{#1}}
\newcommand{\WarningTok}[1]{\textcolor[rgb]{0.56,0.35,0.01}{\textbf{\textit{#1}}}}
\usepackage{graphicx}
\makeatletter
\def\maxwidth{\ifdim\Gin@nat@width>\linewidth\linewidth\else\Gin@nat@width\fi}
\def\maxheight{\ifdim\Gin@nat@height>\textheight\textheight\else\Gin@nat@height\fi}
\makeatother
% Scale images if necessary, so that they will not overflow the page
% margins by default, and it is still possible to overwrite the defaults
% using explicit options in \includegraphics[width, height, ...]{}
\setkeys{Gin}{width=\maxwidth,height=\maxheight,keepaspectratio}
% Set default figure placement to htbp
\makeatletter
\def\fps@figure{htbp}
\makeatother
\setlength{\emergencystretch}{3em} % prevent overfull lines
\providecommand{\tightlist}{%
  \setlength{\itemsep}{0pt}\setlength{\parskip}{0pt}}
\setcounter{secnumdepth}{5}
\ifLuaTeX
  \usepackage{selnolig}  % disable illegal ligatures
\fi

\begin{document}
\maketitle

{
\setcounter{tocdepth}{2}
\tableofcontents
}
\hypertarget{introduction}{%
\section{Introduction}\label{introduction}}

Dans ce TD, nous allons faire un PLS-DA. Regardez votre cours partie 5.3
(page 43).

On utilise un dataset indiquant l'expression de 6830 gènes dans des
échantillons de 64 cancers.

On importe les données et on retire du dataset les sous-types de cancer
suivants : UNKNOWN, K562B-repro,K562A-repro, MCF7A-repro, MCF7D-repro (à
faire).

\begin{Shaded}
\begin{Highlighting}[]
\NormalTok{dat}\FloatTok{.1} \OtherTok{\textless{}{-}} \FunctionTok{read.table}\NormalTok{(}\AttributeTok{file=}\StringTok{"nci.data.txt"}\NormalTok{)}
\NormalTok{names.data }\OtherTok{\textless{}{-}} \FunctionTok{as.character}\NormalTok{(}\FunctionTok{read.csv}\NormalTok{(}\AttributeTok{file=}\StringTok{"names{-}sample.csv"}\NormalTok{,}\AttributeTok{header=}\ConstantTok{FALSE}\NormalTok{,}\AttributeTok{sep=}\StringTok{" "}\NormalTok{)[,}\DecValTok{1}\NormalTok{])}
\end{Highlighting}
\end{Shaded}

\hypertarget{question}{%
\section{Question}\label{question}}

\hypertarget{retirer-de-dat.1-et-de-names.final-les-sous-types-de-cancer-suivants-unknown-k562b-reprok562a-repro-mcf7a-repro-mcf7d-repro.}{%
\subsection{Retirer de dat.1 et de names.final les sous-types de cancer
suivants : UNKNOWN, K562B-repro,K562A-repro, MCF7A-repro,
MCF7D-repro.}\label{retirer-de-dat.1-et-de-names.final-les-sous-types-de-cancer-suivants-unknown-k562b-reprok562a-repro-mcf7a-repro-mcf7d-repro.}}

\begin{Shaded}
\begin{Highlighting}[]
\NormalTok{ech }\OtherTok{\textless{}{-}} \FunctionTok{c}\NormalTok{(}\StringTok{"UNKNOWN"}\NormalTok{, }\StringTok{"K562B{-}repro"}\NormalTok{, }\StringTok{"K562A{-}repro"}\NormalTok{, }\StringTok{"MCF7A{-}repro"}\NormalTok{, }\StringTok{"MCF7D{-}repro"}\NormalTok{)}
\NormalTok{supp }\OtherTok{\textless{}{-}} \FunctionTok{which}\NormalTok{(names.data}\SpecialCharTok{\%in\%}\NormalTok{ech)}
\NormalTok{names.final }\OtherTok{\textless{}{-}}\NormalTok{ names.data[}\SpecialCharTok{{-}}\NormalTok{supp]}
\NormalTok{dat}\FloatTok{.1} \OtherTok{\textless{}{-}}\NormalTok{ dat}\FloatTok{.1}\NormalTok{[,}\SpecialCharTok{{-}}\NormalTok{supp]}
\end{Highlighting}
\end{Shaded}

\begin{Shaded}
\begin{Highlighting}[]
\FunctionTok{table}\NormalTok{(names.final)}
\end{Highlighting}
\end{Shaded}

\begin{verbatim}
## names.final
##            BREAST      CNS    COLON LEUKEMIA MELANOMA    NSCLC  OVARIAN 
##        4        7        5        7        6        8        9        6 
## PROSTATE    RENAL 
##        2        9
\end{verbatim}

Une fois les sous-types de cancer retirés, on travaille avec 59
échantillons.

On définit alors \(Y\), une variable de réponse binaire :

\begin{itemize}
\tightlist
\item
  1 pour les sous-types de cancer : colon, leukemia, prostate, NSCLC
\item
  0 pour : BREAST, SCNS, MELANOMA, OVARIAN, RENAL
\end{itemize}

\begin{Shaded}
\begin{Highlighting}[]
\NormalTok{Y }\OtherTok{\textless{}{-}} \FunctionTok{rep}\NormalTok{(}\DecValTok{0}\NormalTok{,}\DecValTok{59}\NormalTok{)}
\NormalTok{group1 }\OtherTok{\textless{}{-}} \FunctionTok{which}\NormalTok{(names.final}\SpecialCharTok{\%in\%}\FunctionTok{c}\NormalTok{(}\StringTok{"COLON"}\NormalTok{,}\StringTok{"LEUKEMIA"}\NormalTok{,}\StringTok{"PROSTATE"}\NormalTok{,}\StringTok{"NSCLC"}\NormalTok{))}
\NormalTok{Y[group1] }\OtherTok{\textless{}{-}} \DecValTok{1}
\FunctionTok{table}\NormalTok{(Y)}
\end{Highlighting}
\end{Shaded}

\begin{verbatim}
## Y
##  0  1 
## 35 24
\end{verbatim}

On importe ensuite \texttt{mixOmics} et on définit notre \(X\) et \(Y\).

\begin{Shaded}
\begin{Highlighting}[]
\FunctionTok{library}\NormalTok{(mixOmics)}
\NormalTok{X }\OtherTok{\textless{}{-}} \FunctionTok{t}\NormalTok{(dat}\FloatTok{.1}\NormalTok{)}
\NormalTok{Y }\OtherTok{\textless{}{-}} \FunctionTok{factor}\NormalTok{(Y)}
\end{Highlighting}
\end{Shaded}

\hypertarget{expliquer-pourquoi-on-effectue-les-deux-lignes-x---tdat.1-et-y---factory-2-points}{%
\subsection{\texorpdfstring{Expliquer pourquoi on effectue les deux
lignes \texttt{X\ \textless{}-\ t(dat.1)} et
\texttt{Y\ \textless{}-\ factor(Y)} (2
points)}{Expliquer pourquoi on effectue les deux lignes X \textless- t(dat.1) et Y \textless- factor(Y) (2 points)}}\label{expliquer-pourquoi-on-effectue-les-deux-lignes-x---tdat.1-et-y---factory-2-points}}

Afin de pouvoir calculer faire des produits matricielles entre Y et X,
il est nécessaire de prendre la transposé de X.

Concernant Y, comme il ne peut prendre que 1 ou 0, celui-ci est un
vecteur dont les éléments ne peuvent prendre que des modalités
prédéfinies (numérique : 0 et 1). C'est pour cela qu'on le converti en
facteur afin qu'il définisse ces deux classes.

\hypertarget{effectuer-la-pls-da-des-donnuxe9es-et-tracer-le-graphe-des-individus-et-le-cercle-des-corruxe9lations-des-variables.}{%
\subsection{Effectuer la PLS-DA des données et tracer le graphe des
individus et le cercle des corrélations des
variables.}\label{effectuer-la-pls-da-des-donnuxe9es-et-tracer-le-graphe-des-individus-et-le-cercle-des-corruxe9lations-des-variables.}}

\begin{Shaded}
\begin{Highlighting}[]
\CommentTok{\#Run the method}
\NormalTok{pls\_da }\OtherTok{\textless{}{-}} \FunctionTok{plsda}\NormalTok{(X, Y)}
\CommentTok{\#Plot the samples}
\FunctionTok{plotIndiv}\NormalTok{(pls\_da,}\AttributeTok{legend=}\ConstantTok{TRUE}\NormalTok{, }\AttributeTok{ellipse =} \ConstantTok{TRUE}\NormalTok{, }\AttributeTok{title =} \StringTok{\textquotesingle{}Graphe des individus\textquotesingle{}}\NormalTok{)}
\end{Highlighting}
\end{Shaded}

\includegraphics{Sujet_TD5_files/figure-latex/unnamed-chunk-6-1.pdf}

\begin{Shaded}
\begin{Highlighting}[]
\CommentTok{\#Plot the variables}
\FunctionTok{plotVar}\NormalTok{(pls\_da, }\AttributeTok{var.names=}\ConstantTok{FALSE}\NormalTok{,}\AttributeTok{cex=}\FloatTok{0.2}\NormalTok{, }\AttributeTok{title=}\StringTok{"Cercle de corrélation des variables"}\NormalTok{)}
\end{Highlighting}
\end{Shaded}

\includegraphics{Sujet_TD5_files/figure-latex/unnamed-chunk-6-2.pdf}

\hypertarget{tracer-le-cercle-des-variables-avec-seulement-les-variables-ayant-contribuuxe9-uxe0-plus-de-40-uxe0-la-cruxe9ation-des-axes.}{%
\subsection{Tracer le cercle des variables avec seulement les variables
ayant contribué à plus de 40\% à la création des
axes.}\label{tracer-le-cercle-des-variables-avec-seulement-les-variables-ayant-contribuuxe9-uxe0-plus-de-40-uxe0-la-cruxe9ation-des-axes.}}

\begin{Shaded}
\begin{Highlighting}[]
\FunctionTok{plotVar}\NormalTok{(pls\_da, }\AttributeTok{cutoff=}\FloatTok{0.4}\NormalTok{, }\AttributeTok{var.names=}\ConstantTok{FALSE}\NormalTok{, }\AttributeTok{cex=}\FloatTok{0.2}\NormalTok{,}
        \AttributeTok{title=}\StringTok{"Cercle de corrélation des variables ayant contribué à plus de 40\%"}\NormalTok{)}
\end{Highlighting}
\end{Shaded}

\includegraphics{Sujet_TD5_files/figure-latex/unnamed-chunk-7-1.pdf}

\hypertarget{effectuer-la-pls-da-avec-10-composants-puis-effectuer-la-classification-avec-la-fonction-perf-et-10-ruxe9puxe9titions.}{%
\subsection{\texorpdfstring{Effectuer la PLS-DA avec 10 composants, puis
effectuer la classification avec la fonction \texttt{perf} et 10
répétitions.}{Effectuer la PLS-DA avec 10 composants, puis effectuer la classification avec la fonction perf et 10 répétitions.}}\label{effectuer-la-pls-da-avec-10-composants-puis-effectuer-la-classification-avec-la-fonction-perf-et-10-ruxe9puxe9titions.}}

Pour la reproductibilité de cette question, on définira la seed à 2.

Ensuite, afficher le résultat grâce à:
\texttt{plot(résultat\_de\_perf,\ col\ =\ color.mixo(5:7),\ sd\ =\ TRUE,\ legend.position\ =\ "horizontal",ylim=c(0,0.4))}

\begin{Shaded}
\begin{Highlighting}[]
\FunctionTok{set.seed}\NormalTok{(}\DecValTok{2}\NormalTok{) }
\NormalTok{pls\_da2 }\OtherTok{\textless{}{-}} \FunctionTok{plsda}\NormalTok{(X,Y, }\AttributeTok{ncomp=}\DecValTok{10}\NormalTok{)}
\NormalTok{résultat\_de\_perf }\OtherTok{\textless{}{-}}  \FunctionTok{perf}\NormalTok{(pls\_da2, }\AttributeTok{nrepeat =} \DecValTok{10}\NormalTok{)}
\FunctionTok{plot}\NormalTok{(résultat\_de\_perf, }\AttributeTok{col =} \FunctionTok{color.mixo}\NormalTok{(}\DecValTok{5}\SpecialCharTok{:}\DecValTok{7}\NormalTok{), }\AttributeTok{sd =} \ConstantTok{TRUE}\NormalTok{, }
     \AttributeTok{legend.position =} \StringTok{"horizontal"}\NormalTok{,}\AttributeTok{ylim=}\FunctionTok{c}\NormalTok{(}\FloatTok{0.1}\NormalTok{,}\FloatTok{0.4}\NormalTok{))}
\end{Highlighting}
\end{Shaded}

\includegraphics{Sujet_TD5_files/figure-latex/unnamed-chunk-8-1.pdf}

\hypertarget{lancer-une-sparse-pls-da-en-ne-gardant-dans-les-loadings-de-x-que-4-variables-sur-le-premier-axe-et-3-sur-le-deuxiuxe8me-et-afficher-le-graphe-des-individus-et-le-cercle-des-variables.}{%
\subsection{\texorpdfstring{Lancer une sparse PLS-DA en ne gardant dans
les loadings de \(X\) que 4 variables sur le premier axe et 3 sur le
deuxième et afficher le graphe des individus et le cercle des
variables.}{Lancer une sparse PLS-DA en ne gardant dans les loadings de X que 4 variables sur le premier axe et 3 sur le deuxième et afficher le graphe des individus et le cercle des variables.}}\label{lancer-une-sparse-pls-da-en-ne-gardant-dans-les-loadings-de-x-que-4-variables-sur-le-premier-axe-et-3-sur-le-deuxiuxe8me-et-afficher-le-graphe-des-individus-et-le-cercle-des-variables.}}

Si besoin, se référer à la documentation :
\url{https://www.rdocumentation.org/packages/mixOmics/versions/6.3.2/topics/splsda}
(2 points)

\begin{Shaded}
\begin{Highlighting}[]
\NormalTok{spls\_da }\OtherTok{\textless{}{-}} \FunctionTok{splsda}\NormalTok{(X, Y, }\AttributeTok{keepX =} \FunctionTok{c}\NormalTok{(}\DecValTok{4}\NormalTok{,}\DecValTok{3}\NormalTok{))}
\CommentTok{\#Plot the samples}
\FunctionTok{plotIndiv}\NormalTok{(spls\_da,}\AttributeTok{legend=}\ConstantTok{TRUE}\NormalTok{, }\AttributeTok{ellipse =} \ConstantTok{TRUE}\NormalTok{, }\AttributeTok{title =} \StringTok{\textquotesingle{}Graphe des individus\textquotesingle{}}\NormalTok{)}
\end{Highlighting}
\end{Shaded}

\includegraphics{Sujet_TD5_files/figure-latex/unnamed-chunk-9-1.pdf}

\begin{Shaded}
\begin{Highlighting}[]
\CommentTok{\#Plot the variables}
\FunctionTok{plotVar}\NormalTok{(spls\_da,}\AttributeTok{title=}\StringTok{"Cercle de corrélation des variables"}\NormalTok{)}
\end{Highlighting}
\end{Shaded}

\includegraphics{Sujet_TD5_files/figure-latex/unnamed-chunk-9-2.pdf}

\hypertarget{on-cherche-uxe0-trouver-le-nombre-optimal-de-variables-uxe0-garder.-pour-cela-on-cruxe9e-une-liste-de-nombres-quon-va-tester}{%
\subsection{On cherche à trouver le nombre optimal de variables à
garder. Pour cela on crée une liste de nombres qu'on va tester
:}\label{on-cherche-uxe0-trouver-le-nombre-optimal-de-variables-uxe0-garder.-pour-cela-on-cruxe9e-une-liste-de-nombres-quon-va-tester}}

\begin{Shaded}
\begin{Highlighting}[]
\NormalTok{list.keepX }\OtherTok{\textless{}{-}} \FunctionTok{c}\NormalTok{(}\DecValTok{5}\SpecialCharTok{:}\DecValTok{10}\NormalTok{, }\FunctionTok{seq}\NormalTok{(}\DecValTok{20}\NormalTok{, }\DecValTok{200}\NormalTok{, }\DecValTok{10}\NormalTok{))}
\NormalTok{list.keepX}
\end{Highlighting}
\end{Shaded}

\begin{verbatim}
##  [1]   5   6   7   8   9  10  20  30  40  50  60  70  80  90 100 110 120 130 140
## [20] 150 160 170 180 190 200
\end{verbatim}

Fixer la seed à 2 et, grâce à la fonction \texttt{tune.splsda}, trouver
le nombre optimal de variables à garder sur les deux premiers axes.
Ensuite, afficher le résultat de l'appel grâce à la fonction `plot' (4
points)

\begin{Shaded}
\begin{Highlighting}[]
\FunctionTok{set.seed}\NormalTok{(}\DecValTok{2}\NormalTok{)}
\NormalTok{spls\_da\_tune }\OtherTok{\textless{}{-}} \FunctionTok{tune.splsda}\NormalTok{(X, Y, }\AttributeTok{ncomp =} \DecValTok{3}\NormalTok{, }\AttributeTok{dist =} \StringTok{\textquotesingle{}max.dist\textquotesingle{}}\NormalTok{, }\AttributeTok{progressBar =} \ConstantTok{FALSE}\NormalTok{, }
                            \AttributeTok{validation =} \StringTok{\textquotesingle{}Mfold\textquotesingle{}}\NormalTok{, }\AttributeTok{folds =} \DecValTok{3}\NormalTok{, }\AttributeTok{measure =} \StringTok{"BER"}\NormalTok{, }
                            \AttributeTok{test.keepX =}\NormalTok{ list.keepX, }\AttributeTok{nrepeat =} \DecValTok{10}\NormalTok{)}

\NormalTok{spls\_da\_tune}\SpecialCharTok{$}\NormalTok{choice.keepX[}\DecValTok{1}\SpecialCharTok{:}\DecValTok{2}\NormalTok{]}
\end{Highlighting}
\end{Shaded}

\begin{verbatim}
## comp1 comp2 
##   180    50
\end{verbatim}

\begin{Shaded}
\begin{Highlighting}[]
\FunctionTok{plot}\NormalTok{(spls\_da\_tune)}
\end{Highlighting}
\end{Shaded}

\includegraphics{Sujet_TD5_files/figure-latex/unnamed-chunk-12-1.pdf}

\hypertarget{moduxe8le-final}{%
\subsection{Modèle final}\label{moduxe8le-final}}

Réaliser la PLS-DA avec le nombre optimal de variables sur les deux
premiers axes, puis tracer le graphe des individus.

\begin{Shaded}
\begin{Highlighting}[]
\NormalTok{select.keepX }\OtherTok{\textless{}{-}}\NormalTok{ spls\_da\_tune}\SpecialCharTok{$}\NormalTok{choice.keepX[}\DecValTok{1}\SpecialCharTok{:}\DecValTok{2}\NormalTok{]}
\NormalTok{pls\_da\_final }\OtherTok{\textless{}{-}} \FunctionTok{splsda}\NormalTok{(X, Y, }\AttributeTok{keepX =}\NormalTok{ select.keepX)}

\FunctionTok{plotIndiv}\NormalTok{(pls\_da\_final,}\AttributeTok{legend=}\ConstantTok{TRUE}\NormalTok{, }\AttributeTok{ellipse =} \ConstantTok{TRUE}\NormalTok{, }\AttributeTok{title =} \StringTok{\textquotesingle{}Graphe des individus\textquotesingle{}}\NormalTok{)}
\end{Highlighting}
\end{Shaded}

\includegraphics{Sujet_TD5_files/figure-latex/unnamed-chunk-13-1.pdf}

\end{document}
