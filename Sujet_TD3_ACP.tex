% Options for packages loaded elsewhere
\PassOptionsToPackage{unicode}{hyperref}
\PassOptionsToPackage{hyphens}{url}
%
\documentclass[
]{article}
\title{TD3 - Implémentation de l'ACP}
\author{Courtenay Rebecca \& Ducros Chloé \& Lasson Marie}
\date{}

\usepackage{amsmath,amssymb}
\usepackage{lmodern}
\usepackage{iftex}
\ifPDFTeX
  \usepackage[T1]{fontenc}
  \usepackage[utf8]{inputenc}
  \usepackage{textcomp} % provide euro and other symbols
\else % if luatex or xetex
  \usepackage{unicode-math}
  \defaultfontfeatures{Scale=MatchLowercase}
  \defaultfontfeatures[\rmfamily]{Ligatures=TeX,Scale=1}
\fi
% Use upquote if available, for straight quotes in verbatim environments
\IfFileExists{upquote.sty}{\usepackage{upquote}}{}
\IfFileExists{microtype.sty}{% use microtype if available
  \usepackage[]{microtype}
  \UseMicrotypeSet[protrusion]{basicmath} % disable protrusion for tt fonts
}{}
\makeatletter
\@ifundefined{KOMAClassName}{% if non-KOMA class
  \IfFileExists{parskip.sty}{%
    \usepackage{parskip}
  }{% else
    \setlength{\parindent}{0pt}
    \setlength{\parskip}{6pt plus 2pt minus 1pt}}
}{% if KOMA class
  \KOMAoptions{parskip=half}}
\makeatother
\usepackage{xcolor}
\IfFileExists{xurl.sty}{\usepackage{xurl}}{} % add URL line breaks if available
\IfFileExists{bookmark.sty}{\usepackage{bookmark}}{\usepackage{hyperref}}
\hypersetup{
  pdftitle={TD3 - Implémentation de l'ACP},
  pdfauthor={Courtenay Rebecca \& Ducros Chloé \& Lasson Marie},
  hidelinks,
  pdfcreator={LaTeX via pandoc}}
\urlstyle{same} % disable monospaced font for URLs
\usepackage[margin=1in]{geometry}
\usepackage{color}
\usepackage{fancyvrb}
\newcommand{\VerbBar}{|}
\newcommand{\VERB}{\Verb[commandchars=\\\{\}]}
\DefineVerbatimEnvironment{Highlighting}{Verbatim}{commandchars=\\\{\}}
% Add ',fontsize=\small' for more characters per line
\usepackage{framed}
\definecolor{shadecolor}{RGB}{248,248,248}
\newenvironment{Shaded}{\begin{snugshade}}{\end{snugshade}}
\newcommand{\AlertTok}[1]{\textcolor[rgb]{0.94,0.16,0.16}{#1}}
\newcommand{\AnnotationTok}[1]{\textcolor[rgb]{0.56,0.35,0.01}{\textbf{\textit{#1}}}}
\newcommand{\AttributeTok}[1]{\textcolor[rgb]{0.77,0.63,0.00}{#1}}
\newcommand{\BaseNTok}[1]{\textcolor[rgb]{0.00,0.00,0.81}{#1}}
\newcommand{\BuiltInTok}[1]{#1}
\newcommand{\CharTok}[1]{\textcolor[rgb]{0.31,0.60,0.02}{#1}}
\newcommand{\CommentTok}[1]{\textcolor[rgb]{0.56,0.35,0.01}{\textit{#1}}}
\newcommand{\CommentVarTok}[1]{\textcolor[rgb]{0.56,0.35,0.01}{\textbf{\textit{#1}}}}
\newcommand{\ConstantTok}[1]{\textcolor[rgb]{0.00,0.00,0.00}{#1}}
\newcommand{\ControlFlowTok}[1]{\textcolor[rgb]{0.13,0.29,0.53}{\textbf{#1}}}
\newcommand{\DataTypeTok}[1]{\textcolor[rgb]{0.13,0.29,0.53}{#1}}
\newcommand{\DecValTok}[1]{\textcolor[rgb]{0.00,0.00,0.81}{#1}}
\newcommand{\DocumentationTok}[1]{\textcolor[rgb]{0.56,0.35,0.01}{\textbf{\textit{#1}}}}
\newcommand{\ErrorTok}[1]{\textcolor[rgb]{0.64,0.00,0.00}{\textbf{#1}}}
\newcommand{\ExtensionTok}[1]{#1}
\newcommand{\FloatTok}[1]{\textcolor[rgb]{0.00,0.00,0.81}{#1}}
\newcommand{\FunctionTok}[1]{\textcolor[rgb]{0.00,0.00,0.00}{#1}}
\newcommand{\ImportTok}[1]{#1}
\newcommand{\InformationTok}[1]{\textcolor[rgb]{0.56,0.35,0.01}{\textbf{\textit{#1}}}}
\newcommand{\KeywordTok}[1]{\textcolor[rgb]{0.13,0.29,0.53}{\textbf{#1}}}
\newcommand{\NormalTok}[1]{#1}
\newcommand{\OperatorTok}[1]{\textcolor[rgb]{0.81,0.36,0.00}{\textbf{#1}}}
\newcommand{\OtherTok}[1]{\textcolor[rgb]{0.56,0.35,0.01}{#1}}
\newcommand{\PreprocessorTok}[1]{\textcolor[rgb]{0.56,0.35,0.01}{\textit{#1}}}
\newcommand{\RegionMarkerTok}[1]{#1}
\newcommand{\SpecialCharTok}[1]{\textcolor[rgb]{0.00,0.00,0.00}{#1}}
\newcommand{\SpecialStringTok}[1]{\textcolor[rgb]{0.31,0.60,0.02}{#1}}
\newcommand{\StringTok}[1]{\textcolor[rgb]{0.31,0.60,0.02}{#1}}
\newcommand{\VariableTok}[1]{\textcolor[rgb]{0.00,0.00,0.00}{#1}}
\newcommand{\VerbatimStringTok}[1]{\textcolor[rgb]{0.31,0.60,0.02}{#1}}
\newcommand{\WarningTok}[1]{\textcolor[rgb]{0.56,0.35,0.01}{\textbf{\textit{#1}}}}
\usepackage{graphicx}
\makeatletter
\def\maxwidth{\ifdim\Gin@nat@width>\linewidth\linewidth\else\Gin@nat@width\fi}
\def\maxheight{\ifdim\Gin@nat@height>\textheight\textheight\else\Gin@nat@height\fi}
\makeatother
% Scale images if necessary, so that they will not overflow the page
% margins by default, and it is still possible to overwrite the defaults
% using explicit options in \includegraphics[width, height, ...]{}
\setkeys{Gin}{width=\maxwidth,height=\maxheight,keepaspectratio}
% Set default figure placement to htbp
\makeatletter
\def\fps@figure{htbp}
\makeatother
\setlength{\emergencystretch}{3em} % prevent overfull lines
\providecommand{\tightlist}{%
  \setlength{\itemsep}{0pt}\setlength{\parskip}{0pt}}
\setcounter{secnumdepth}{5}
\ifLuaTeX
  \usepackage{selnolig}  % disable illegal ligatures
\fi

\begin{document}
\maketitle

{
\setcounter{tocdepth}{2}
\tableofcontents
}
\hypertarget{question-pruxe9liminaires-bonus-3-points}{%
\section{Question préliminaires (bonus, 3
points)}\label{question-pruxe9liminaires-bonus-3-points}}

\begin{enumerate}
\def\labelenumi{\alph{enumi}.}
\tightlist
\item
  Montrer que pour toutes variables aléatoires \(X\) et \(Y\), on a :
  \[\text E[(X-\text E[X])(Y-\text E[Y])] = \text E[XY] - \text E[X] \text E[Y]\]
\end{enumerate}

On a : \newline \(\text E[(X-\text E[X])(Y-\text E[Y])]\)
\newline \(= \text E[XY] - \text E[X]E[Y] - \text E[X]E[Y] + \text E[X]E[Y]\)
\newline \(= \text E[XY] - E[X]E[Y]\)

\begin{enumerate}
\def\labelenumi{\alph{enumi}.}
\setcounter{enumi}{1}
\tightlist
\item
  Soit \(C\) la matrice de covariance d'une matrice de données \(X\).
  Montrer que la matrice \(C\) est symétrique.
\end{enumerate}

Les éléments diagonaux de C représentent les variances des données de X.
De plus, les autres éléments décrivent la covariance entre \(X_{i}\) et
\(X_{j}\) avec i \(\ne\) j. Et comme,
\(cov(X_{i},X_{j}) = cov(X_{j},X_{i})\) alors, la matrice \(C\) est
symétrique.

\begin{enumerate}
\def\labelenumi{\alph{enumi}.}
\setcounter{enumi}{2}
\tightlist
\item
  Montrer que la matrice \(C\) est semi-définie positive. Rappel : la
  matrice symétrique \(C\) est semi-définie positive ssi
  \(\forall x \in \mathbb R^p, \text{ } x^TCx \geq 0\)
\end{enumerate}

On sait que :
\(x^TCx = \sum_{i,j}x_{i }Cov(X_{i},X_{j})x_{j} = \sum_{i,j}Cov(x_{i}X_{i},x_{j}X_{j}) = Var(x_{1}X_{1}+...+x_{n}X_{n}) \ge\)
0. \newline Donc la matrice \(C\) est semi-définie positive.

\begin{enumerate}
\def\labelenumi{\alph{enumi}.}
\setcounter{enumi}{3}
\tightlist
\item
  Montrer que pour toute matrice \(C\), il y a équivalence entre :
  ``\(C\) est semi-définie positive'' et ``les valeurs propres de \(C\)
  sont toutes positives ou nulles''.
\end{enumerate}

On suppose que \(C\) est une matrice semi-définie positive et
\(\lambda\) une de ses valeurs propres. \newline D'après la question c,
\(x^TCx \ge\) 0, donc \(\lambda (X^TX) \ge\) 0 \(\Rightarrow\)
\(\lambda\) \(|X|^2\) \(\ge\) 0 \(\Rightarrow\) \(\lambda \ge\) 0
\newline On peut donc conclure que : les valeurs propres de \(C\) sont
toutes positives ou nulles.

\hypertarget{impluxe9mentation-de-lacp-20-points}{%
\section{Implémentation de l'ACP (20
points)}\label{impluxe9mentation-de-lacp-20-points}}

L'implémentation de toutes les fonctions de base demandées sera notée
sur 10 points. Tous les apports et améliorations (cf question5) seront
notés sur 10 points.

\hypertarget{premiuxe8re-uxe9tape-calcul-de-lacp}{%
\subsection{Première étape : calcul de
l'ACP}\label{premiuxe8re-uxe9tape-calcul-de-lacp}}

On utilisera le jeu de données de décathlon :

\begin{Shaded}
\begin{Highlighting}[]
\NormalTok{decathlon }\OtherTok{\textless{}{-}} \FunctionTok{read.table}\NormalTok{(}\AttributeTok{file =} \StringTok{"AnaDo\_JeuDonnees\_Decathlon.csv"}\NormalTok{, }
                        \AttributeTok{header=}\ConstantTok{TRUE}\NormalTok{,}\AttributeTok{sep=}\StringTok{";"}\NormalTok{,}\AttributeTok{dec=}\StringTok{"."}\NormalTok{, }\AttributeTok{row.names=}\DecValTok{1}\NormalTok{, }
                        \AttributeTok{check.names=}\ConstantTok{FALSE}\NormalTok{, }\AttributeTok{fileEncoding=}\StringTok{"latin1"}\NormalTok{)}
\end{Highlighting}
\end{Shaded}

On pourra donc s'appuyer sur le fichier pdf ``Commandes utiles'' fourni
dans le TD1 pour comparer nos résultats. Vos graphes seront sûrement le
symétrique de ceux affichés par le package FactoMineR, vous pouvez y
remédier en multipliant votre matrice de données par \((-1)\) mais ce
n'est pas obligatoire.

Pour tracer les cercles des corrélations vous aurez sûrement besoin du
package \texttt{plotrix} qui permet de dessiner des cercles, que vous
pouvez installer via la commande \texttt{install.packages("plotrix")}.

\begin{Shaded}
\begin{Highlighting}[]
\FunctionTok{library}\NormalTok{(plotrix)}
\end{Highlighting}
\end{Shaded}

De la même façon que dans le TD1, on effectuait la commande
\texttt{res\ \textless{}-\ PCA(fertilite{[},1:6{]})}, on veut une
fonction qui prend en entrée le tableau de données et qui effectue
l'ACP.

Dans un premier temps, on ne tiendra pas compte des variables
qualitatives supplémentaires, on les traitera plus tard.

En notant \(X\) notre matrice de données (sans les variables
descriptives supplémentaires), l'ACP consiste à trouver une matrice
\(Y\) contenant les mêmes données mais dans une base orthogonale
différente maximisant l'explicabilité de la variance. Les étapes du
calcul de l'ACP sont les suivantes :

\begin{itemize}
\tightlist
\item
  Centrer et réduire \(X\)
\end{itemize}

\begin{Shaded}
\begin{Highlighting}[]
\NormalTok{X }\OtherTok{\textless{}{-}}\NormalTok{ decathlon[,}\FunctionTok{c}\NormalTok{(}\DecValTok{1}\SpecialCharTok{:}\DecValTok{10}\NormalTok{)]}
\NormalTok{ctr }\OtherTok{\textless{}{-}} \FunctionTok{scale}\NormalTok{(X)}
\end{Highlighting}
\end{Shaded}

\begin{itemize}
\tightlist
\item
  Calcul de la matrice de covariance \(C\) de \(X\)
\end{itemize}

\begin{Shaded}
\begin{Highlighting}[]
\NormalTok{C }\OtherTok{\textless{}{-}} \FunctionTok{cov}\NormalTok{(ctr)}
\end{Highlighting}
\end{Shaded}

\begin{itemize}
\tightlist
\item
  Calcul des vecteurs propres et valeurs propres de \(C\)
\end{itemize}

\begin{Shaded}
\begin{Highlighting}[]
\NormalTok{VAP\_VEP }\OtherTok{\textless{}{-}} \ControlFlowTok{function}\NormalTok{(X)\{}
\NormalTok{  data }\OtherTok{\textless{}{-}} \FunctionTok{cov}\NormalTok{(X)}
\NormalTok{  vap }\OtherTok{\textless{}{-}} \FunctionTok{eigen}\NormalTok{(data)}\SpecialCharTok{$}\NormalTok{values}
\NormalTok{  vep }\OtherTok{\textless{}{-}} \FunctionTok{eigen}\NormalTok{(data)}\SpecialCharTok{$}\NormalTok{vectors}
  
  \FunctionTok{return}\NormalTok{(}\FunctionTok{list}\NormalTok{(vap, vep))}
\NormalTok{\}}
\end{Highlighting}
\end{Shaded}

\begin{itemize}
\tightlist
\item
  Tri des vecteurs propres et valeurs propres par ordre décroissant de
  l'importance des valeurs propres
\end{itemize}

La fonction eigen permet de trier les valeurs. Nous venons de le faire
dans la fonction VAP\_VEP donc nous n'avons pas besoin d'appliquer la
fonction sort/order (qui fait exactement la même chose). Il nous suffit
d'appliquer : VAP\_VEP(X){[}{[}1{]}{]} pour avoir les valeurs propres
triées et VAP\_VEP(X){[}{[}2{]}{]} pour avoir les vecteurs propres
triés.

Ecriture des matrices V et L

\begin{Shaded}
\begin{Highlighting}[]
\NormalTok{L}\OtherTok{\textless{}{-}}\FunctionTok{diag}\NormalTok{(}\FunctionTok{VAP\_VEP}\NormalTok{(X)[[}\DecValTok{1}\NormalTok{]]) }\CommentTok{\#matrice diagonale des valeurs propres}
\NormalTok{V}\OtherTok{\textless{}{-}}\FunctionTok{VAP\_VEP}\NormalTok{(X)[[}\DecValTok{2}\NormalTok{]] }\CommentTok{\#matrice des vecteurs propres}
\end{Highlighting}
\end{Shaded}

\begin{itemize}
\tightlist
\item
  Ecriture des données dans la base orthonormée des vecteurs propres, ce
  qui correspond à une projection orthogonale de nos données sur la
  matrice \(W\) des vecteurs propres, ce qui se fait avec le produit
  matriciel \(X \cdot W\). Plus d'explications ci-dessous.
\end{itemize}

Nous affichons que les 6 premières données.

\begin{Shaded}
\begin{Highlighting}[]
\NormalTok{W }\OtherTok{\textless{}{-}} \FunctionTok{as.matrix}\NormalTok{(}\FunctionTok{VAP\_VEP}\NormalTok{(X)[[}\DecValTok{1}\NormalTok{]],}\FunctionTok{VAP\_VEP}\NormalTok{(X)[[}\DecValTok{2}\NormalTok{]])}
\NormalTok{X2}\OtherTok{\textless{}{-}}\FunctionTok{as.matrix}\NormalTok{(X)}
\FunctionTok{head}\NormalTok{(X2}\SpecialCharTok{\%*\%}\NormalTok{W)}
\end{Highlighting}
\end{Shaded}

\begin{verbatim}
##               [,1]
## Sebrle    1875.137
## Clay      1810.327
## Karpov    1820.948
## Macey     1865.555
## Warners   1821.186
## Zsivoczky 1856.528
\end{verbatim}

\begin{enumerate}
\def\labelenumi{\arabic{enumi}.}
\tightlist
\item
  Ecrire la fonction \texttt{ACP} qui prend en entrée un tableau de
  données et qui retourne l'ensemble des valeurs propres (dans une
  variable \texttt{\$vap}) et des vecteurs propres (dans une variable
  \texttt{\$vep} ) de la matrice de covariance, le tout trié par ordre
  décroissant. Dans une variable \texttt{\$data} on stockera également
  les données centrées réduites. Pour la tester, on pourra faire l'appel
  suivant : \texttt{ACP(decathlon{[},1:10{]})}
\end{enumerate}

\begin{Shaded}
\begin{Highlighting}[]
\NormalTok{ACP }\OtherTok{\textless{}{-}} \ControlFlowTok{function}\NormalTok{(X)\{}
\NormalTok{  data }\OtherTok{\textless{}{-}} \FunctionTok{scale}\NormalTok{(X)}
\NormalTok{  cov }\OtherTok{\textless{}{-}} \FunctionTok{cov}\NormalTok{(data)}
\NormalTok{  vap }\OtherTok{\textless{}{-}} \FunctionTok{eigen}\NormalTok{(cov)}\SpecialCharTok{$}\NormalTok{values}
\NormalTok{  vep }\OtherTok{\textless{}{-}} \FunctionTok{eigen}\NormalTok{(cov)}\SpecialCharTok{$}\NormalTok{vectors}
  
  \FunctionTok{return}\NormalTok{(}\FunctionTok{list}\NormalTok{(}\StringTok{"vap"}\OtherTok{=}\NormalTok{vap,}\StringTok{"vep"}\OtherTok{=}\NormalTok{vep,}\StringTok{"data"}\OtherTok{=}\NormalTok{data,}\StringTok{"cov"}\OtherTok{=}\NormalTok{cov))}
\NormalTok{\}}
\end{Highlighting}
\end{Shaded}

Valeurs propres :

\begin{Shaded}
\begin{Highlighting}[]
\FunctionTok{ACP}\NormalTok{(X)}\SpecialCharTok{$}\NormalTok{vap}
\end{Highlighting}
\end{Shaded}

\begin{verbatim}
##  [1] 3.2719055 1.7371310 1.4049167 1.0568504 0.6847735 0.5992687 0.4512353
##  [8] 0.3968766 0.2148149 0.1822275
\end{verbatim}

Vecteurs propres : Nous affichons que les 6 premières données.

\begin{Shaded}
\begin{Highlighting}[]
\FunctionTok{head}\NormalTok{(}\FunctionTok{ACP}\NormalTok{(X)}\SpecialCharTok{$}\NormalTok{vep)}
\end{Highlighting}
\end{Shaded}

\begin{verbatim}
##            [,1]       [,2]        [,3]        [,4]        [,5]        [,6]
## [1,]  0.4282963  0.1419891  0.15557953  0.03678703 -0.36518741  0.29607739
## [2,] -0.4101520 -0.2620794 -0.15372674 -0.09901016 -0.04432336 -0.30612478
## [3,] -0.3441444  0.4539470  0.01972378 -0.18539458 -0.13431954  0.30547299
## [4,] -0.3161944  0.2657761  0.21894349  0.13189684 -0.67121760 -0.46777116
## [5,]  0.3757157  0.4320460 -0.11091758 -0.02850297  0.10597034 -0.33252178
## [6,]  0.4125544  0.1735910  0.07815576 -0.28290068 -0.19857266 -0.09963776
##             [,7]        [,8]       [,9]       [,10]
## [1,] -0.38177608  0.46160211  0.1047577  0.42428269
## [2,] -0.62769317 -0.02101165  0.4826691  0.08104448
## [3,]  0.30972542 -0.31393005  0.4272908  0.39028424
## [4,]  0.09145002  0.12509166 -0.2436605 -0.10642724
## [5,]  0.12442114  0.21339819  0.5521294 -0.41399532
## [6,] -0.35733030 -0.71111429 -0.1501343 -0.09086448
\end{verbatim}

Données centrée-réduite :

Nous affichons que les 6 premières données.

\begin{Shaded}
\begin{Highlighting}[]
\FunctionTok{head}\NormalTok{(}\FunctionTok{ACP}\NormalTok{(X)}\SpecialCharTok{$}\NormalTok{data)}
\end{Highlighting}
\end{Shaded}

\begin{verbatim}
##                 100m   Longueur       Poids     Hauteur       400m      110m H
## Sebrle    -0.5628739  1.8331131 2.283919585  1.60955474 -1.0892025 -1.17818270
## Clay      -2.1216730  2.2123779 0.913271989  0.93502243 -0.3696226 -1.00861538
## Karpov    -1.8935560  1.7382969 1.762345721  1.27228858 -2.4329962 -1.34775002
## Macey     -0.4107959  0.6637134 1.519753226  1.94682089 -0.5603546 -0.09719103
## Warners   -1.4373221  1.5170591 0.003550134 -0.07677604 -1.4273183 -1.26296636
## Zsivoczky -0.3347570 -0.3792648 1.010308987  1.60955474 -0.1875602  0.72944966
##               Disque     Perche     Javelot       1500m
## Sebrle     1.3009450  0.8545364  2.52825135  0.08439142
## Clay       1.7124500  0.4948240  2.36043901  0.25486670
## Karpov     2.1683620 -0.5843134 -0.57524110 -0.07837391
## Macey      1.1884472 -1.3037383  0.02971203 -1.16547502
## Warners   -0.1763283  0.4948240 -0.60631746 -0.08351387
## Zsivoczky  0.3832000 -0.2246009  1.06351893 -0.81253124
\end{verbatim}

Matrice de covariance : Nous affichons que les 6 premières données.

\begin{Shaded}
\begin{Highlighting}[]
\FunctionTok{head}\NormalTok{(}\FunctionTok{ACP}\NormalTok{(X)}\SpecialCharTok{$}\NormalTok{cov)}
\end{Highlighting}
\end{Shaded}

\begin{verbatim}
##                100m   Longueur      Poids    Hauteur       400m     110m H
## 100m      1.0000000 -0.5986777 -0.3564823 -0.2462529  0.5202982  0.5798889
## Longueur -0.5986777  1.0000000  0.1833044  0.2946444 -0.6020626 -0.5054101
## Poids    -0.3564823  0.1833044  1.0000000  0.4892115 -0.1384329 -0.2516157
## Hauteur  -0.2462529  0.2946444  0.4892115  1.0000000 -0.1879569 -0.2832891
## 400m      0.5202982 -0.6020626 -0.1384329 -0.1879569  1.0000000  0.5479878
## 110m H    0.5798889 -0.5054101 -0.2516157 -0.2832891  0.5479878  1.0000000
##              Disque       Perche      Javelot       1500m
## 100m     -0.2217076 -0.082536834 -0.157746452 -0.06054645
## Longueur  0.1943101  0.204014112  0.119758933 -0.03368613
## Poids     0.6157681  0.061181853  0.374955509  0.11580306
## Hauteur   0.3692183 -0.156180742  0.171880092 -0.04490252
## 400m     -0.1178794 -0.079292469  0.004232096  0.40810643
## 110m H   -0.3262010 -0.002703885  0.008743251  0.03754024
\end{verbatim}

\hypertarget{deuxiuxe8me-uxe9tape-affichage-de-lacp-pour-les-individus}{%
\subsection{Deuxième étape : affichage de l'ACP pour les
individus}\label{deuxiuxe8me-uxe9tape-affichage-de-lacp-pour-les-individus}}

\begin{enumerate}
\def\labelenumi{\arabic{enumi}.}
\setcounter{enumi}{1}
\tightlist
\item
  Ecrire la fonction \texttt{individusACP} qui trace l'ACP des individus
  sur les axes sélectionnés. Elle aura donc en entrée un argument
  ``axes'' sur lequel tracer les données. On pourra la tester sur les
  deux premières dimensions avec l'appel :
  \texttt{individusACP(axes=1:2)}
\end{enumerate}

La visualisation de cette fonction se trouve dans la 5ème partie avec
une version améliorée.

\begin{Shaded}
\begin{Highlighting}[]
\NormalTok{individusACP }\OtherTok{\textless{}{-}} \ControlFlowTok{function}\NormalTok{(axes, PCA) \{}
\NormalTok{  V }\OtherTok{\textless{}{-}} \FunctionTok{as.matrix}\NormalTok{(}\FunctionTok{ACP}\NormalTok{(PCA)}\SpecialCharTok{$}\NormalTok{vep)}
\NormalTok{  D }\OtherTok{\textless{}{-}} \FunctionTok{as.matrix}\NormalTok{(}\FunctionTok{ACP}\NormalTok{(PCA)}\SpecialCharTok{$}\NormalTok{data)}
\NormalTok{  res }\OtherTok{\textless{}{-}}\NormalTok{ D}\SpecialCharTok{\%*\%}\NormalTok{V}
  
\NormalTok{  dim1 }\OtherTok{\textless{}{-}} \FunctionTok{paste}\NormalTok{(}\StringTok{"Dim"}\NormalTok{,}\FunctionTok{as.character}\NormalTok{(axes[}\DecValTok{1}\NormalTok{]))}
\NormalTok{  dim2 }\OtherTok{\textless{}{-}} \FunctionTok{paste}\NormalTok{(}\StringTok{"Dim"}\NormalTok{,}\FunctionTok{as.character}\NormalTok{(axes[}\DecValTok{2}\NormalTok{]))}
  
  \FunctionTok{par}\NormalTok{(}\AttributeTok{mar=}\FunctionTok{c}\NormalTok{(}\DecValTok{4}\NormalTok{,}\DecValTok{4}\NormalTok{,}\DecValTok{4}\NormalTok{,}\DecValTok{4}\NormalTok{),}\AttributeTok{pty =} \StringTok{"s"}\NormalTok{)}
  \FunctionTok{plot}\NormalTok{(res[,axes], }\AttributeTok{xlab =}\NormalTok{ dim1, }\AttributeTok{ylab =}\NormalTok{ dim2, }\AttributeTok{pch =} \DecValTok{1}\NormalTok{, }\AttributeTok{col=}\StringTok{"red"}\NormalTok{)}
  \FunctionTok{text}\NormalTok{(res[,axes], }\AttributeTok{labels =} \FunctionTok{row.names}\NormalTok{(}\FunctionTok{ACP}\NormalTok{(PCA)}\SpecialCharTok{$}\NormalTok{data), }\AttributeTok{cex =} \FloatTok{0.6}\NormalTok{, }\AttributeTok{pos =} \DecValTok{1}\NormalTok{)}
  \FunctionTok{abline}\NormalTok{(}\AttributeTok{v =} \DecValTok{0}\NormalTok{, }\AttributeTok{lty =} \DecValTok{2}\NormalTok{)}
  \FunctionTok{abline}\NormalTok{(}\AttributeTok{h =} \DecValTok{0}\NormalTok{, }\AttributeTok{lty =} \DecValTok{2}\NormalTok{)}
  \FunctionTok{title}\NormalTok{(}\StringTok{"ACP"}\NormalTok{)}
  \FunctionTok{return}\NormalTok{(res)}
\NormalTok{\}}
\end{Highlighting}
\end{Shaded}

\hypertarget{troisiuxe8me-uxe9tape-fonction-summary}{%
\subsection{Troisième étape : fonction
summary}\label{troisiuxe8me-uxe9tape-fonction-summary}}

\begin{enumerate}
\def\labelenumi{\arabic{enumi}.}
\setcounter{enumi}{2}
\tightlist
\item
  Ecrire la fonction \texttt{summary}, permettant de retourner les
  informations principales de l'ACP. Il est conseillé (mais pas
  obligatoire) d'écrire plusieurs sous-fonctions pour composer la
  fonction \texttt{summary}.
\end{enumerate}

La visualisation de cette fonction se trouve dans la 5ème partie avec
une version améliorée.

\begin{Shaded}
\begin{Highlighting}[]
\NormalTok{summary }\OtherTok{\textless{}{-}} \ControlFlowTok{function}\NormalTok{(PCA) \{}
  
  \CommentTok{\#Eigenvalues}
\NormalTok{  vap }\OtherTok{\textless{}{-}} \FunctionTok{ACP}\NormalTok{(PCA)}\SpecialCharTok{$}\NormalTok{vap}
\NormalTok{  vect }\OtherTok{\textless{}{-}} \FunctionTok{c}\NormalTok{()}
  \ControlFlowTok{for}\NormalTok{ (i }\ControlFlowTok{in} \FunctionTok{seq.int}\NormalTok{(}\DecValTok{0}\NormalTok{,}\FunctionTok{length}\NormalTok{(vap))) \{}
\NormalTok{    somme }\OtherTok{\textless{}{-}} \FunctionTok{sum}\NormalTok{(vap)}
\NormalTok{    pourcentage }\OtherTok{\textless{}{-}}\NormalTok{ (vap[i]}\SpecialCharTok{/}\NormalTok{somme)}\SpecialCharTok{*}\DecValTok{100}
\NormalTok{    vect }\OtherTok{\textless{}{-}} \FunctionTok{append}\NormalTok{(vect, pourcentage)}
\NormalTok{  \}}
\NormalTok{  eigenvalues }\OtherTok{\textless{}{-}} \FunctionTok{cbind}\NormalTok{(vap, vect)}
  \FunctionTok{colnames}\NormalTok{(eigenvalues) }\OtherTok{\textless{}{-}} \FunctionTok{c}\NormalTok{(}\StringTok{"Variance"}\NormalTok{, }\StringTok{"\% of var."}\NormalTok{)}
  \FunctionTok{print}\NormalTok{(}\StringTok{"Eigenvalues"}\NormalTok{)}
  \FunctionTok{print}\NormalTok{(}\FunctionTok{head}\NormalTok{(eigenvalues))}
  
  \CommentTok{\#Individuals}
\NormalTok{  dim }\OtherTok{\textless{}{-}} \FunctionTok{individusACP}\NormalTok{(}\AttributeTok{axes=}\DecValTok{1}\SpecialCharTok{:}\DecValTok{2}\NormalTok{, PCA)}
  \FunctionTok{graphics.off}\NormalTok{()}

    \CommentTok{\#Dim1}
\NormalTok{  contr\_1 }\OtherTok{\textless{}{-}} \FunctionTok{c}\NormalTok{()}
\NormalTok{  sommecarre\_1 }\OtherTok{\textless{}{-}}\NormalTok{ dim[,}\DecValTok{1}\NormalTok{]}\SpecialCharTok{\%*\%}\NormalTok{dim[,}\DecValTok{1}\NormalTok{]}
  \ControlFlowTok{for}\NormalTok{ (i }\ControlFlowTok{in} \FunctionTok{seq.int}\NormalTok{(}\DecValTok{0}\NormalTok{,}\FunctionTok{length}\NormalTok{(dim[,}\DecValTok{1}\NormalTok{]))) \{}
\NormalTok{    contr\_1 }\OtherTok{\textless{}{-}} \FunctionTok{append}\NormalTok{(contr\_1, ((dim[,}\DecValTok{1}\NormalTok{][i])}\SpecialCharTok{**}\DecValTok{2}\SpecialCharTok{/}\NormalTok{sommecarre\_1)}\SpecialCharTok{*}\DecValTok{100}\NormalTok{)}
\NormalTok{  \}}
\NormalTok{  individuals }\OtherTok{\textless{}{-}} \FunctionTok{cbind}\NormalTok{(dim[,}\DecValTok{1}\NormalTok{], contr\_1)}
  
\NormalTok{  cos2\_1 }\OtherTok{\textless{}{-}} \FunctionTok{c}\NormalTok{()}
  \ControlFlowTok{for}\NormalTok{ (i }\ControlFlowTok{in} \FunctionTok{seq.int}\NormalTok{(}\DecValTok{0}\NormalTok{,}\FunctionTok{length}\NormalTok{(dim[,}\DecValTok{1}\NormalTok{]))) \{}
\NormalTok{    cos2\_1 }\OtherTok{\textless{}{-}} \FunctionTok{append}\NormalTok{(cos2\_1, dim[i,}\DecValTok{1}\NormalTok{]}\SpecialCharTok{**}\DecValTok{2}\SpecialCharTok{/}\FunctionTok{sqrt}\NormalTok{(}\FunctionTok{sum}\NormalTok{(dim[i,]}\SpecialCharTok{**}\DecValTok{2}\NormalTok{))}\SpecialCharTok{**}\DecValTok{2}\NormalTok{)}
\NormalTok{  \}}
\NormalTok{  individuals }\OtherTok{\textless{}{-}} \FunctionTok{cbind}\NormalTok{(individuals, cos2\_1)}
  
    \CommentTok{\#Dim2}
\NormalTok{  contr\_2 }\OtherTok{\textless{}{-}} \FunctionTok{c}\NormalTok{()}
\NormalTok{  sommecarre\_2 }\OtherTok{\textless{}{-}}\NormalTok{ dim[,}\DecValTok{2}\NormalTok{]}\SpecialCharTok{\%*\%}\NormalTok{dim[,}\DecValTok{2}\NormalTok{]}
\NormalTok{  individuals }\OtherTok{\textless{}{-}} \FunctionTok{cbind}\NormalTok{(individuals, dim[,}\DecValTok{2}\NormalTok{])}
  
  \ControlFlowTok{for}\NormalTok{ (i }\ControlFlowTok{in} \FunctionTok{seq.int}\NormalTok{(}\DecValTok{0}\NormalTok{,}\FunctionTok{length}\NormalTok{(dim[,}\DecValTok{2}\NormalTok{]))) \{}
\NormalTok{    contr\_2 }\OtherTok{\textless{}{-}} \FunctionTok{append}\NormalTok{(contr\_2, ((dim[,}\DecValTok{2}\NormalTok{][i])}\SpecialCharTok{**}\DecValTok{2}\SpecialCharTok{/}\NormalTok{sommecarre\_2)}\SpecialCharTok{*}\DecValTok{100}\NormalTok{)}
\NormalTok{  \}}
\NormalTok{  individuals }\OtherTok{\textless{}{-}} \FunctionTok{cbind}\NormalTok{(individuals, contr\_2)}

\NormalTok{  cos2\_2 }\OtherTok{\textless{}{-}} \FunctionTok{c}\NormalTok{()}
  \ControlFlowTok{for}\NormalTok{ (i }\ControlFlowTok{in} \FunctionTok{seq.int}\NormalTok{(}\DecValTok{0}\NormalTok{,}\FunctionTok{length}\NormalTok{(dim[,}\DecValTok{2}\NormalTok{]))) \{}
\NormalTok{    cos2\_2 }\OtherTok{\textless{}{-}} \FunctionTok{append}\NormalTok{(cos2\_2, dim[i,}\DecValTok{2}\NormalTok{]}\SpecialCharTok{**}\DecValTok{2}\SpecialCharTok{/}\FunctionTok{sqrt}\NormalTok{(}\FunctionTok{sum}\NormalTok{(dim[i,]}\SpecialCharTok{**}\DecValTok{2}\NormalTok{))}\SpecialCharTok{**}\DecValTok{2}\NormalTok{)}
\NormalTok{  \}}
\NormalTok{  individuals }\OtherTok{\textless{}{-}} \FunctionTok{cbind}\NormalTok{(individuals, cos2\_2)}
  
  \FunctionTok{colnames}\NormalTok{(individuals) }\OtherTok{\textless{}{-}} \FunctionTok{c}\NormalTok{(}\StringTok{"Dim 1"}\NormalTok{,}\StringTok{"ctr"}\NormalTok{, }\StringTok{"cos2"}\NormalTok{, }\StringTok{"Dim 2"}\NormalTok{,}\StringTok{"ctr"}\NormalTok{, }\StringTok{"cos2"}\NormalTok{)}
  
  \FunctionTok{print}\NormalTok{(}\StringTok{"Individuals"}\NormalTok{)}
  \FunctionTok{print}\NormalTok{(}\FunctionTok{head}\NormalTok{(individuals))}
  
  \CommentTok{\#Variables}
    \CommentTok{\#Dim1}
\NormalTok{  corr\_1 }\OtherTok{\textless{}{-}} \FunctionTok{c}\NormalTok{()}
  \ControlFlowTok{for}\NormalTok{ (i }\ControlFlowTok{in} \FunctionTok{seq.int}\NormalTok{(}\DecValTok{0}\NormalTok{, }\FunctionTok{length}\NormalTok{(vap))) \{}
\NormalTok{    corr\_1 }\OtherTok{\textless{}{-}} \FunctionTok{append}\NormalTok{(corr\_1, }\FunctionTok{sqrt}\NormalTok{(vap[}\DecValTok{1}\NormalTok{])}\SpecialCharTok{*}\FunctionTok{ACP}\NormalTok{(PCA)}\SpecialCharTok{$}\NormalTok{vep[i,}\DecValTok{1}\NormalTok{])}
\NormalTok{  \}}
\NormalTok{  variables }\OtherTok{\textless{}{-}} \FunctionTok{as.data.frame}\NormalTok{(corr\_1)}

\NormalTok{  cos2\_1 }\OtherTok{\textless{}{-}} \FunctionTok{c}\NormalTok{()}
  \ControlFlowTok{for}\NormalTok{ (i }\ControlFlowTok{in} \FunctionTok{seq.int}\NormalTok{(}\DecValTok{0}\NormalTok{, }\FunctionTok{length}\NormalTok{(corr\_1))) \{}
\NormalTok{    cos2\_1 }\OtherTok{\textless{}{-}} \FunctionTok{append}\NormalTok{(cos2\_1, corr\_1[i]}\SpecialCharTok{**}\DecValTok{2}\NormalTok{)}
\NormalTok{  \}}
\NormalTok{  variables }\OtherTok{\textless{}{-}} \FunctionTok{cbind}\NormalTok{(variables, cos2\_1)}
  
\NormalTok{  contr\_1 }\OtherTok{\textless{}{-}} \FunctionTok{c}\NormalTok{()}
\NormalTok{  sumSquare1 }\OtherTok{\textless{}{-}} \FunctionTok{ACP}\NormalTok{(PCA)}\SpecialCharTok{$}\NormalTok{vep[,}\DecValTok{1}\NormalTok{]}\SpecialCharTok{\%*\%}\FunctionTok{ACP}\NormalTok{(PCA)}\SpecialCharTok{$}\NormalTok{vep[,}\DecValTok{1}\NormalTok{]}
  \ControlFlowTok{for}\NormalTok{ (i }\ControlFlowTok{in} \FunctionTok{seq.int}\NormalTok{(}\DecValTok{0}\NormalTok{,}\FunctionTok{length}\NormalTok{(}\FunctionTok{ACP}\NormalTok{(PCA)}\SpecialCharTok{$}\NormalTok{vep[,}\DecValTok{1}\NormalTok{]))) \{}
\NormalTok{    contr\_1 }\OtherTok{\textless{}{-}} \FunctionTok{append}\NormalTok{(contr\_1, (}\FunctionTok{ACP}\NormalTok{(PCA)}\SpecialCharTok{$}\NormalTok{vep[i,}\DecValTok{1}\NormalTok{]}\SpecialCharTok{**}\DecValTok{2}\SpecialCharTok{/}\NormalTok{sumSquare1)}\SpecialCharTok{*}\DecValTok{100}\NormalTok{)}
\NormalTok{  \}}
\NormalTok{  variables }\OtherTok{\textless{}{-}} \FunctionTok{cbind}\NormalTok{(variables, contr\_1)}
  
    \CommentTok{\#Dim2}
\NormalTok{  corr\_2 }\OtherTok{\textless{}{-}} \FunctionTok{c}\NormalTok{()}
  \ControlFlowTok{for}\NormalTok{ (i }\ControlFlowTok{in} \FunctionTok{seq.int}\NormalTok{(}\DecValTok{0}\NormalTok{, }\FunctionTok{length}\NormalTok{(vap))) \{}
\NormalTok{    corr\_2 }\OtherTok{\textless{}{-}} \FunctionTok{append}\NormalTok{(corr\_2, }\FunctionTok{sqrt}\NormalTok{(vap[}\DecValTok{2}\NormalTok{])}\SpecialCharTok{*}\FunctionTok{ACP}\NormalTok{(PCA)}\SpecialCharTok{$}\NormalTok{vep[i,}\DecValTok{2}\NormalTok{])}
\NormalTok{  \}}
\NormalTok{  variables }\OtherTok{\textless{}{-}} \FunctionTok{cbind}\NormalTok{(variables, corr\_2)}
  
\NormalTok{  cos2\_2 }\OtherTok{\textless{}{-}} \FunctionTok{c}\NormalTok{()}
  \ControlFlowTok{for}\NormalTok{ (i }\ControlFlowTok{in} \FunctionTok{seq.int}\NormalTok{(}\DecValTok{0}\NormalTok{, }\FunctionTok{length}\NormalTok{(corr\_2))) \{}
\NormalTok{    cos2\_2 }\OtherTok{\textless{}{-}} \FunctionTok{append}\NormalTok{(cos2\_2, corr\_2[i]}\SpecialCharTok{**}\DecValTok{2}\NormalTok{)}
\NormalTok{  \}}
\NormalTok{  variables }\OtherTok{\textless{}{-}} \FunctionTok{cbind}\NormalTok{(variables, cos2\_2)}

\NormalTok{  contr\_2 }\OtherTok{\textless{}{-}} \FunctionTok{c}\NormalTok{()}
\NormalTok{  sumSquare2 }\OtherTok{\textless{}{-}} \FunctionTok{ACP}\NormalTok{(PCA)}\SpecialCharTok{$}\NormalTok{vep[,}\DecValTok{2}\NormalTok{]}\SpecialCharTok{\%*\%}\FunctionTok{ACP}\NormalTok{(PCA)}\SpecialCharTok{$}\NormalTok{vep[,}\DecValTok{2}\NormalTok{]}
  \ControlFlowTok{for}\NormalTok{ (i }\ControlFlowTok{in} \FunctionTok{seq.int}\NormalTok{(}\DecValTok{0}\NormalTok{,}\FunctionTok{length}\NormalTok{(}\FunctionTok{ACP}\NormalTok{(PCA)}\SpecialCharTok{$}\NormalTok{vep[,}\DecValTok{2}\NormalTok{]))) \{}
\NormalTok{    contr\_2 }\OtherTok{\textless{}{-}} \FunctionTok{append}\NormalTok{(contr\_2, (}\FunctionTok{ACP}\NormalTok{(PCA)}\SpecialCharTok{$}\NormalTok{vep[i,}\DecValTok{2}\NormalTok{]}\SpecialCharTok{**}\DecValTok{2}\SpecialCharTok{/}\NormalTok{sumSquare2)}\SpecialCharTok{*}\DecValTok{100}\NormalTok{)}
\NormalTok{  \}}
\NormalTok{  variables }\OtherTok{\textless{}{-}} \FunctionTok{cbind}\NormalTok{(variables, contr\_2)}
  
\NormalTok{  nom\_Var }\OtherTok{\textless{}{-}} \FunctionTok{colnames}\NormalTok{(}\FunctionTok{ACP}\NormalTok{(PCA)}\SpecialCharTok{$}\NormalTok{data)}
  \FunctionTok{row.names}\NormalTok{(variables) }\OtherTok{\textless{}{-}}\NormalTok{ nom\_Var}
  \FunctionTok{colnames}\NormalTok{(variables) }\OtherTok{\textless{}{-}} \FunctionTok{c}\NormalTok{(}\StringTok{"Corr 1"}\NormalTok{, }\StringTok{"Cos2 1"}\NormalTok{, }\StringTok{"Ctr 1"}\NormalTok{,}\StringTok{"Corr 2"}\NormalTok{, }\StringTok{"Cos2 2"}\NormalTok{, }\StringTok{"Ctr 2"}\NormalTok{)}
  
  \FunctionTok{print}\NormalTok{(}\StringTok{"Variables"}\NormalTok{)}
  \FunctionTok{print}\NormalTok{(}\FunctionTok{head}\NormalTok{(variables))}
\NormalTok{\}}
\end{Highlighting}
\end{Shaded}

\hypertarget{quatriuxe8me-uxe9tape-affichage-de-lacp-pour-les-variables}{%
\subsection{Quatrième étape : affichage de l'ACP pour les
variables}\label{quatriuxe8me-uxe9tape-affichage-de-lacp-pour-les-variables}}

\begin{enumerate}
\def\labelenumi{\arabic{enumi}.}
\setcounter{enumi}{3}
\tightlist
\item
  Ecrire la fonction \texttt{variablesACP} qui trace le cercle de
  corrélation des variables sur les axes sélectionnés. Elle aura donc en
  entrée un argument ``axes'' sur lequel tracer les données. On pourra
  la tester sur les deux premières dimensions avec l'appel :
  \texttt{variablesACP(axes=1:2)}. Il ne sera pas obligatoire de
  dessiner des flèches (comme dans le package FactoMineR).
\end{enumerate}

La visualisation de cette fonction se trouve dans la 5ème partie avec
une version améliorée.

\begin{Shaded}
\begin{Highlighting}[]
\NormalTok{variablesACP }\OtherTok{\textless{}{-}} \ControlFlowTok{function}\NormalTok{(axes, PCA) \{ }
  
\NormalTok{  vep }\OtherTok{\textless{}{-}} \FunctionTok{ACP}\NormalTok{(PCA)}\SpecialCharTok{$}\NormalTok{vep}
\NormalTok{  vap }\OtherTok{\textless{}{-}} \FunctionTok{ACP}\NormalTok{(PCA)}\SpecialCharTok{$}\NormalTok{vap}
\NormalTok{  cor\_v }\OtherTok{\textless{}{-}} \FunctionTok{c}\NormalTok{()}
\NormalTok{  data }\OtherTok{\textless{}{-}} \FunctionTok{ACP}\NormalTok{(PCA)}\SpecialCharTok{$}\NormalTok{data}
\NormalTok{  taille\_v }\OtherTok{\textless{}{-}} \FunctionTok{length}\NormalTok{(vep[,}\DecValTok{1}\NormalTok{])}
  
  \ControlFlowTok{for}\NormalTok{ (k }\ControlFlowTok{in} \FunctionTok{seq.int}\NormalTok{(}\DecValTok{1}\NormalTok{, taille\_v)) \{}
\NormalTok{    cor\_dv }\OtherTok{\textless{}{-}} \FunctionTok{c}\NormalTok{()}
    \ControlFlowTok{for}\NormalTok{ (i }\ControlFlowTok{in} \FunctionTok{seq.int}\NormalTok{(}\DecValTok{0}\NormalTok{, }\FunctionTok{length}\NormalTok{(vap))) \{}
\NormalTok{      cor\_dv }\OtherTok{\textless{}{-}} \FunctionTok{append}\NormalTok{(cor\_dv, }\FunctionTok{sqrt}\NormalTok{(vap[k])}\SpecialCharTok{*}\NormalTok{vep[i,k])}
\NormalTok{    \}}
\NormalTok{    cor\_v }\OtherTok{\textless{}{-}} \FunctionTok{cbind}\NormalTok{(cor\_v, cor\_dv)}
\NormalTok{  \}}
  
\NormalTok{  dim1 }\OtherTok{\textless{}{-}} \FunctionTok{paste}\NormalTok{(}\StringTok{"Dim"}\NormalTok{,}\FunctionTok{as.character}\NormalTok{(axes[}\DecValTok{1}\NormalTok{]))}
\NormalTok{  dim2 }\OtherTok{\textless{}{-}} \FunctionTok{paste}\NormalTok{(}\StringTok{"Dim"}\NormalTok{,}\FunctionTok{as.character}\NormalTok{(axes[}\DecValTok{2}\NormalTok{]))}

  \FunctionTok{par}\NormalTok{(}\AttributeTok{mar=}\FunctionTok{c}\NormalTok{(}\DecValTok{4}\NormalTok{,}\DecValTok{4}\NormalTok{,}\DecValTok{4}\NormalTok{,}\DecValTok{4}\NormalTok{),}\AttributeTok{pty =} \StringTok{"s"}\NormalTok{)}
  \FunctionTok{plot}\NormalTok{(cor\_v[,axes],}\AttributeTok{xlim =} \FunctionTok{c}\NormalTok{(}\SpecialCharTok{{-}}\DecValTok{1}\NormalTok{,}\DecValTok{1}\NormalTok{),}\AttributeTok{ylim =} \FunctionTok{c}\NormalTok{(}\SpecialCharTok{{-}}\DecValTok{1}\NormalTok{,}\DecValTok{1}\NormalTok{),}\AttributeTok{xlab=}\NormalTok{dim1,}\AttributeTok{ylab=}\NormalTok{dim2)}
  \FunctionTok{text}\NormalTok{(cor\_v[,axes], }\AttributeTok{labels =} \FunctionTok{colnames}\NormalTok{(}\FunctionTok{ACP}\NormalTok{(PCA)}\SpecialCharTok{$}\NormalTok{data), }\AttributeTok{cex =} \FloatTok{0.6}\NormalTok{, }\AttributeTok{pos =} \DecValTok{1}\NormalTok{)}
  \FunctionTok{abline}\NormalTok{(}\AttributeTok{v =} \DecValTok{0}\NormalTok{, }\AttributeTok{lty =} \DecValTok{2}\NormalTok{)}
  \FunctionTok{abline}\NormalTok{(}\AttributeTok{h =} \DecValTok{0}\NormalTok{, }\AttributeTok{lty =} \DecValTok{2}\NormalTok{)}
  \FunctionTok{title}\NormalTok{(}\StringTok{"Graphique des variables"}\NormalTok{)}
  \FunctionTok{draw.circle}\NormalTok{(}\DecValTok{0}\NormalTok{, }\DecValTok{0}\NormalTok{, }\DecValTok{1}\NormalTok{, }\AttributeTok{nv =} \DecValTok{500}\NormalTok{, }\AttributeTok{border =} \ConstantTok{NULL}\NormalTok{, }\AttributeTok{col =} \ConstantTok{NA}\NormalTok{, }\AttributeTok{lty =} \DecValTok{1}\NormalTok{, }\AttributeTok{lwd =} \DecValTok{1}\NormalTok{)}
\NormalTok{\}}
\end{Highlighting}
\end{Shaded}

\hypertarget{derniuxe8re-uxe9tape-amuxe9lioration-des-fonctions}{%
\subsection{Dernière étape : amélioration des
fonctions}\label{derniuxe8re-uxe9tape-amuxe9lioration-des-fonctions}}

\begin{enumerate}
\def\labelenumi{\arabic{enumi}.}
\setcounter{enumi}{4}
\tightlist
\item
  En vous inspirant des fonctions du package FactoMineR utilisées lors
  du TD1, améliorer toutes les fonctions écrites ci-dessus pour y
  inclure des options supplémentaires. Par exemple, voici une liste (non
  exhaustive) des choses qu'il est possible d'améliorer
\end{enumerate}

En annexe se trouvent les fonctions intermédiaires que nous avons créées
avant d'avoir ces versions finales.

\hypertarget{pca}{%
\subsubsection{PCA}\label{pca}}

\begin{itemize}
\tightlist
\item
  Dans la fonction ACP, ajouter deux paramètres supplémentaires :
  \texttt{quanti.sup} et \texttt{quali.sup}, qui indiqueront des
  colonnes. Si ces paramètres ne sont pas indiqués par l'utilisateur,
  alors l'ACP se comporte de la même façon que la fonction précédemment
  écrite. S'ils sont entrés par l'utilisateur, ils indiquent les
  colonnes des variables quantitatives ou qualitatives supplémentaires.
  Par exemple, la fonction répondra à l'appel :
  \texttt{PCA(decathlon,\ quanti.sup=11:12,\ quali.sup=13)}. Bien sûr,
  il est possible de rentrer un seul des deux arguments (comme pour le
  jeu de donnée de fertilité en Europe)
\item
  Ajouter un paramètre en entrée de l'ACP pour savoir si l'on souhaite
  centrer et réduire les données. En effet, cela n'est pas toujours
  souhaitable (notamment quand on a des informations avec beaucoup de
  bruit, car le fait de normaliser les données va mettre le bruit à la
  même valeur que les autres données)
\item
  Effectuer les calculs à chaque nouvel appel de \texttt{summarize} ou
  des fonctions graphiques n'est pas optimal (redondant). Le mieux
  serait, dès l'appel à la fonction \texttt{ACP}, de faire tous les
  calculs nécessaires et de les stocker dans des variables, qui seraient
  ensuite simplement retournées ou utilisées lors de l'appel des
  fonctions
\end{itemize}

\begin{Shaded}
\begin{Highlighting}[]
\NormalTok{ACP\_FINAL }\OtherTok{\textless{}{-}} \ControlFlowTok{function}\NormalTok{(X, }\AttributeTok{quanti.sup=}\ConstantTok{NULL}\NormalTok{, }\AttributeTok{quali.sup=}\ConstantTok{NULL}\NormalTok{, }\AttributeTok{cr =} \ConstantTok{TRUE}\NormalTok{) \{}

    \CommentTok{\#quanti.sub \& quali.sub}
\NormalTok{  supp }\OtherTok{\textless{}{-}} \FunctionTok{c}\NormalTok{()}
  \ControlFlowTok{if}\NormalTok{ (}\SpecialCharTok{!}\FunctionTok{is.null}\NormalTok{(quanti.sup))\{}
    \ControlFlowTok{for}\NormalTok{ (i }\ControlFlowTok{in}\NormalTok{ quanti.sup) \{}
\NormalTok{      supp }\OtherTok{\textless{}{-}} \FunctionTok{append}\NormalTok{(supp, i)}
\NormalTok{    \}}
\NormalTok{  \}}
  \ControlFlowTok{if}\NormalTok{ (}\SpecialCharTok{!}\FunctionTok{is.null}\NormalTok{(quali.sup)) \{}
    \ControlFlowTok{for}\NormalTok{ (i }\ControlFlowTok{in}\NormalTok{ quali.sup) \{}
\NormalTok{      supp }\OtherTok{\textless{}{-}} \FunctionTok{append}\NormalTok{(supp, i)}
\NormalTok{    \}}
\NormalTok{  \}}
  \ControlFlowTok{if}\NormalTok{ (}\SpecialCharTok{!}\FunctionTok{is.null}\NormalTok{(supp)) \{}
\NormalTok{    supp }\OtherTok{\textless{}{-}} \FunctionTok{sort}\NormalTok{(supp)}
\NormalTok{    X }\OtherTok{\textless{}{-}}\NormalTok{ X[,}\SpecialCharTok{{-}}\NormalTok{supp]}
\NormalTok{  \}}
  
    \CommentTok{\#centree{-}reduite}
  \ControlFlowTok{if}\NormalTok{ (cr }\SpecialCharTok{==} \ConstantTok{FALSE}\NormalTok{) \{}
\NormalTok{    data }\OtherTok{\textless{}{-}}\NormalTok{ X}
\NormalTok{  \}}
  \ControlFlowTok{else}\NormalTok{ \{}
\NormalTok{    data }\OtherTok{\textless{}{-}} \FunctionTok{scale}\NormalTok{(X)}
\NormalTok{  \}}

    \CommentTok{\#covariance}
\NormalTok{  cov }\OtherTok{\textless{}{-}} \FunctionTok{cov}\NormalTok{(data) }
    
    \CommentTok{\#valeurs/vecteurs propres}
\NormalTok{  vap }\OtherTok{\textless{}{-}} \FunctionTok{eigen}\NormalTok{(cov)}\SpecialCharTok{$}\NormalTok{values}
\NormalTok{  vep }\OtherTok{\textless{}{-}} \FunctionTok{eigen}\NormalTok{(cov)}\SpecialCharTok{$}\NormalTok{vectors}
  
    \CommentTok{\#Eigenvalues}
\NormalTok{  vect }\OtherTok{\textless{}{-}} \FunctionTok{c}\NormalTok{()}
  \ControlFlowTok{for}\NormalTok{ (i }\ControlFlowTok{in} \FunctionTok{seq.int}\NormalTok{(}\DecValTok{0}\NormalTok{,}\FunctionTok{length}\NormalTok{(vap))) \{}
\NormalTok{    somme }\OtherTok{\textless{}{-}} \FunctionTok{sum}\NormalTok{(vap)}
\NormalTok{    pourcentage }\OtherTok{\textless{}{-}}\NormalTok{ (vap[i]}\SpecialCharTok{/}\NormalTok{somme)}\SpecialCharTok{*}\DecValTok{100}
\NormalTok{    vect }\OtherTok{\textless{}{-}} \FunctionTok{append}\NormalTok{(vect, pourcentage)}
\NormalTok{  \}}
\NormalTok{  cumul\_var }\OtherTok{\textless{}{-}} \FunctionTok{cumsum}\NormalTok{(vect)}
\NormalTok{  eigenvalues }\OtherTok{\textless{}{-}} \FunctionTok{cbind}\NormalTok{(vap, vect, cumul\_var)}
  \FunctionTok{colnames}\NormalTok{(eigenvalues) }\OtherTok{\textless{}{-}} \FunctionTok{c}\NormalTok{(}\StringTok{"Variance"}\NormalTok{, }\StringTok{"\% of var."}\NormalTok{, }\StringTok{"Cumulative \% of var"}\NormalTok{)}
  
    \CommentTok{\#Individus}
      \CommentTok{\#Cor}
\NormalTok{  V }\OtherTok{\textless{}{-}} \FunctionTok{as.matrix}\NormalTok{(vep)}
\NormalTok{  D }\OtherTok{\textless{}{-}} \FunctionTok{as.matrix}\NormalTok{(data)}
\NormalTok{  cor\_i }\OtherTok{\textless{}{-}}\NormalTok{ D}\SpecialCharTok{\%*\%}\NormalTok{V}
      \CommentTok{\#Contribution}
\NormalTok{  contr\_i }\OtherTok{\textless{}{-}} \FunctionTok{c}\NormalTok{()}
  \ControlFlowTok{for}\NormalTok{ (k }\ControlFlowTok{in} \FunctionTok{seq.int}\NormalTok{(}\DecValTok{1}\NormalTok{, }\FunctionTok{length}\NormalTok{(cor\_i[}\DecValTok{1}\NormalTok{,]))) \{}
\NormalTok{    sommecarre }\OtherTok{\textless{}{-}}\NormalTok{ cor\_i[,k]}\SpecialCharTok{\%*\%}\NormalTok{cor\_i[,k]}
\NormalTok{    contr\_di }\OtherTok{\textless{}{-}} \FunctionTok{c}\NormalTok{()}
    \ControlFlowTok{for}\NormalTok{ (i }\ControlFlowTok{in} \FunctionTok{seq.int}\NormalTok{(}\DecValTok{0}\NormalTok{,}\FunctionTok{length}\NormalTok{(cor\_i[,k]))) \{}
\NormalTok{      contr\_di }\OtherTok{\textless{}{-}} \FunctionTok{append}\NormalTok{(contr\_di, ((cor\_i[,k][i])}\SpecialCharTok{**}\DecValTok{2}\SpecialCharTok{/}\NormalTok{sommecarre)}\SpecialCharTok{*}\DecValTok{100}\NormalTok{)}
\NormalTok{    \}}
\NormalTok{    contr\_i }\OtherTok{\textless{}{-}} \FunctionTok{cbind}\NormalTok{(contr\_i, contr\_di)}
\NormalTok{  \}}
      \CommentTok{\#Cos2}
\NormalTok{  cos2\_i }\OtherTok{\textless{}{-}} \FunctionTok{c}\NormalTok{()}
  \ControlFlowTok{for}\NormalTok{ (k }\ControlFlowTok{in} \FunctionTok{seq.int}\NormalTok{(}\DecValTok{1}\NormalTok{, }\FunctionTok{length}\NormalTok{(cor\_i[}\DecValTok{1}\NormalTok{,]))) \{}
\NormalTok{    cos2\_di }\OtherTok{\textless{}{-}} \FunctionTok{c}\NormalTok{()}
    \ControlFlowTok{for}\NormalTok{ (i }\ControlFlowTok{in} \FunctionTok{seq.int}\NormalTok{(}\DecValTok{0}\NormalTok{,}\FunctionTok{length}\NormalTok{(cor\_i[,k]))) \{}
\NormalTok{      cos2\_di }\OtherTok{\textless{}{-}} \FunctionTok{append}\NormalTok{(cos2\_di, cor\_i[i,k]}\SpecialCharTok{**}\DecValTok{2}\SpecialCharTok{/}\FunctionTok{sqrt}\NormalTok{(}\FunctionTok{sum}\NormalTok{(cor\_i[i,]}\SpecialCharTok{**}\DecValTok{2}\NormalTok{))}\SpecialCharTok{**}\DecValTok{2}\NormalTok{)}
\NormalTok{    \}}
\NormalTok{    cos2\_i }\OtherTok{\textless{}{-}} \FunctionTok{cbind}\NormalTok{(cos2\_i, cos2\_di)}
\NormalTok{  \}}
  
    \CommentTok{\#Variables}
\NormalTok{  taille\_v }\OtherTok{\textless{}{-}} \FunctionTok{length}\NormalTok{(vep[,}\DecValTok{1}\NormalTok{])}
      \CommentTok{\#Cor}
\NormalTok{  cor\_v }\OtherTok{\textless{}{-}} \FunctionTok{c}\NormalTok{()}
  \ControlFlowTok{for}\NormalTok{ (k }\ControlFlowTok{in} \FunctionTok{seq.int}\NormalTok{(}\DecValTok{1}\NormalTok{, taille\_v)) \{}
\NormalTok{    cor\_dv }\OtherTok{\textless{}{-}} \FunctionTok{c}\NormalTok{()}
    \ControlFlowTok{for}\NormalTok{ (i }\ControlFlowTok{in} \FunctionTok{seq.int}\NormalTok{(}\DecValTok{0}\NormalTok{, }\FunctionTok{length}\NormalTok{(vap))) \{}
\NormalTok{      cor\_dv }\OtherTok{\textless{}{-}} \FunctionTok{append}\NormalTok{(cor\_dv, }\FunctionTok{sqrt}\NormalTok{(vap[k])}\SpecialCharTok{*}\NormalTok{vep[i,k])}
\NormalTok{    \}}
\NormalTok{    cor\_v }\OtherTok{\textless{}{-}} \FunctionTok{cbind}\NormalTok{(cor\_v, cor\_dv)}
\NormalTok{  \}}
  
    \CommentTok{\#Cos2}
\NormalTok{  cos2\_v }\OtherTok{\textless{}{-}} \FunctionTok{c}\NormalTok{()}
  \ControlFlowTok{for}\NormalTok{ (k }\ControlFlowTok{in} \FunctionTok{seq.int}\NormalTok{(}\DecValTok{1}\NormalTok{, taille\_v)) \{}
\NormalTok{    cos2\_dv }\OtherTok{\textless{}{-}} \FunctionTok{c}\NormalTok{()}
    \ControlFlowTok{for}\NormalTok{ (i }\ControlFlowTok{in} \FunctionTok{seq.int}\NormalTok{(}\DecValTok{0}\NormalTok{, }\FunctionTok{length}\NormalTok{(cor\_v[k,]))) \{}
\NormalTok{      cos2\_dv }\OtherTok{\textless{}{-}} \FunctionTok{append}\NormalTok{(cos2\_dv, cor\_v[k,i]}\SpecialCharTok{**}\DecValTok{2}\NormalTok{)}
\NormalTok{    \}}
\NormalTok{    cos2\_v }\OtherTok{\textless{}{-}} \FunctionTok{cbind}\NormalTok{(cos2\_v, cos2\_dv)}
\NormalTok{  \}}
    \CommentTok{\#Contribution}
\NormalTok{  contr\_v }\OtherTok{\textless{}{-}} \FunctionTok{c}\NormalTok{()}
  \ControlFlowTok{for}\NormalTok{ (k }\ControlFlowTok{in} \FunctionTok{seq.int}\NormalTok{(}\DecValTok{1}\NormalTok{, taille\_v)) \{}
\NormalTok{    contr\_dv }\OtherTok{\textless{}{-}} \FunctionTok{c}\NormalTok{()}
\NormalTok{    sumSquare }\OtherTok{\textless{}{-}}\NormalTok{ vep[,k]}\SpecialCharTok{\%*\%}\NormalTok{vep[,k]}
    \ControlFlowTok{for}\NormalTok{ (i }\ControlFlowTok{in} \FunctionTok{seq.int}\NormalTok{(}\DecValTok{0}\NormalTok{,}\FunctionTok{length}\NormalTok{(vep[k,]))) \{}
\NormalTok{      contr\_dv }\OtherTok{\textless{}{-}} \FunctionTok{append}\NormalTok{(contr\_dv, (vep[i,k]}\SpecialCharTok{**}\DecValTok{2}\SpecialCharTok{/}\NormalTok{sumSquare)}\SpecialCharTok{*}\DecValTok{100}\NormalTok{)}
\NormalTok{    \}}
\NormalTok{    contr\_v }\OtherTok{\textless{}{-}} \FunctionTok{cbind}\NormalTok{(contr\_v, contr\_dv)}
\NormalTok{  \}}
 

  \FunctionTok{return}\NormalTok{(}\FunctionTok{list}\NormalTok{(}\StringTok{"vap"}\OtherTok{=}\NormalTok{vap,}\StringTok{"vep"}\OtherTok{=}\NormalTok{vep,}\StringTok{"data"}\OtherTok{=}\NormalTok{data,}\StringTok{"cov"}\OtherTok{=}\NormalTok{cov,}\StringTok{"eigenvalues"} \OtherTok{=}\NormalTok{ eigenvalues, }
              \StringTok{"cor\_ind"} \OtherTok{=}\NormalTok{ cor\_i, }\StringTok{"contr\_ind"} \OtherTok{=}\NormalTok{ contr\_i, }\StringTok{"cos2\_ind"} \OtherTok{=}\NormalTok{ cos2\_i,}
              \StringTok{"cor\_var"} \OtherTok{=}\NormalTok{ cor\_v, }\StringTok{"cos2\_var"} \OtherTok{=}\NormalTok{ cos2\_v, }\StringTok{"contr\_var"} \OtherTok{=}\NormalTok{ contr\_v))}
\NormalTok{\}}
\end{Highlighting}
\end{Shaded}

\hypertarget{individus}{%
\subsubsection{Individus}\label{individus}}

\begin{itemize}
\tightlist
\item
  Dans la fonction \texttt{individusACP} on pourra alors ajouter un
  argument ``habillage'' qui colorie les individus selon leur modalité
  (ie selon la colonne indiquée par l'argument)
\end{itemize}

Dans ce cas, nous avons prédéfini les intervalles de couleurs nous même
: il y a trois intervalles de couleur allant du vert au rouge en passant
par l'orange. Plus nous nous approchons du vert plus l'individu, avec la
dimension indiqué dans l'argument habillage, est significatif. A
contrario de la couleur rouge.

\begin{Shaded}
\begin{Highlighting}[]
\NormalTok{individusACP\_FINAL }\OtherTok{\textless{}{-}} \ControlFlowTok{function}\NormalTok{(axes, PCA, habillage) \{}
\NormalTok{  res }\OtherTok{\textless{}{-}} \FunctionTok{ACP\_FINAL}\NormalTok{(PCA)}\SpecialCharTok{$}\NormalTok{cor\_ind}
\NormalTok{  colour }\OtherTok{\textless{}{-}} \FunctionTok{c}\NormalTok{()}
  
  \ControlFlowTok{for}\NormalTok{ (i }\ControlFlowTok{in} \FunctionTok{seq.int}\NormalTok{(}\DecValTok{1}\SpecialCharTok{:}\FunctionTok{nrow}\NormalTok{(res))) \{}
    \ControlFlowTok{if}\NormalTok{ (}\FunctionTok{abs}\NormalTok{(res[i,habillage]) }\SpecialCharTok{\textless{}} \DecValTok{1}\NormalTok{) \{}
\NormalTok{      clr }\OtherTok{\textless{}{-}} \StringTok{"green"}
\NormalTok{    \}}
    \ControlFlowTok{else} \ControlFlowTok{if}\NormalTok{ (}\FunctionTok{abs}\NormalTok{(res[i,habillage]) }\SpecialCharTok{\textless{}} \FloatTok{2.5}\NormalTok{) \{}
\NormalTok{      clr }\OtherTok{\textless{}{-}} \StringTok{"orange"}
\NormalTok{    \}}
    \ControlFlowTok{else}\NormalTok{ \{}
\NormalTok{      clr }\OtherTok{\textless{}{-}} \StringTok{"red"}
\NormalTok{    \}}
\NormalTok{    colour }\OtherTok{\textless{}{-}} \FunctionTok{append}\NormalTok{(colour, clr)}
\NormalTok{  \}}

\NormalTok{  dim1 }\OtherTok{\textless{}{-}} \FunctionTok{paste}\NormalTok{(}\StringTok{"Dim"}\NormalTok{,}\FunctionTok{as.character}\NormalTok{(axes[}\DecValTok{1}\NormalTok{]))}
\NormalTok{  dim2 }\OtherTok{\textless{}{-}} \FunctionTok{paste}\NormalTok{(}\StringTok{"Dim"}\NormalTok{,}\FunctionTok{as.character}\NormalTok{(axes[}\DecValTok{2}\NormalTok{]))}
  \FunctionTok{par}\NormalTok{(}\AttributeTok{mar=}\FunctionTok{c}\NormalTok{(}\DecValTok{4}\NormalTok{,}\DecValTok{4}\NormalTok{,}\DecValTok{4}\NormalTok{,}\DecValTok{4}\NormalTok{),}\AttributeTok{pty =} \StringTok{"s"}\NormalTok{)}
  \FunctionTok{plot}\NormalTok{(res[,axes], }\AttributeTok{xlab =}\NormalTok{ dim1, }\AttributeTok{ylab =}\NormalTok{ dim2, }\AttributeTok{pch =} \DecValTok{1}\NormalTok{, }\AttributeTok{col =}\NormalTok{ colour )}
  \FunctionTok{text}\NormalTok{(res[,axes], }\AttributeTok{labels =} \FunctionTok{row.names}\NormalTok{(}\FunctionTok{ACP\_FINAL}\NormalTok{(X)}\SpecialCharTok{$}\NormalTok{data), }\AttributeTok{cex =} \FloatTok{0.6}\NormalTok{, }\AttributeTok{pos =} \DecValTok{1}\NormalTok{)}
  \FunctionTok{abline}\NormalTok{(}\AttributeTok{v =} \DecValTok{0}\NormalTok{, }\AttributeTok{lty =} \DecValTok{2}\NormalTok{)}
  \FunctionTok{abline}\NormalTok{(}\AttributeTok{h =} \DecValTok{0}\NormalTok{, }\AttributeTok{lty =} \DecValTok{2}\NormalTok{)}
  \FunctionTok{title}\NormalTok{(}\StringTok{"ACP"}\NormalTok{)}
\NormalTok{\}}
\FunctionTok{individusACP\_FINAL}\NormalTok{(}\AttributeTok{axes=}\DecValTok{1}\SpecialCharTok{:}\DecValTok{2}\NormalTok{, }\AttributeTok{PCA=}\NormalTok{X, }\AttributeTok{habillage=}\DecValTok{1}\NormalTok{)}
\end{Highlighting}
\end{Shaded}

\includegraphics{Sujet_TD3_ACP_files/figure-latex/unnamed-chunk-17-1.pdf}

\hypertarget{summary}{%
\subsubsection{Summary}\label{summary}}

\begin{itemize}
\tightlist
\item
  Dans la fonction \texttt{summary}, ajouter un paramètre ``nbelements''
  (par défaut égal à 10) qui détermine le nombre d'individus qui doivent
  être affichés
\item
  Dans la fonction \texttt{summary}, pour les eigenvalues, afficher en
  plus leur pourcentage cumulatif de variance
\end{itemize}

\begin{Shaded}
\begin{Highlighting}[]
\NormalTok{summary\_FINAL }\OtherTok{\textless{}{-}} \ControlFlowTok{function}\NormalTok{(PCA, }\AttributeTok{nbelements=}\DecValTok{10}\NormalTok{) \{}
  
  \CommentTok{\#Eigenvalues}
  \FunctionTok{print}\NormalTok{(}\StringTok{"Eigenvalues"}\NormalTok{)}
  \FunctionTok{print}\NormalTok{(}\FunctionTok{head}\NormalTok{(}\FunctionTok{ACP\_FINAL}\NormalTok{(PCA)}\SpecialCharTok{$}\NormalTok{eigenvalues))}
  
  \CommentTok{\#Individuals}
  
\NormalTok{  individuals }\OtherTok{\textless{}{-}} \FunctionTok{cbind}\NormalTok{(}
    \FunctionTok{ACP\_FINAL}\NormalTok{(PCA)}\SpecialCharTok{$}\NormalTok{cor\_ind[,}\DecValTok{1}\NormalTok{], }\FunctionTok{ACP\_FINAL}\NormalTok{(PCA)}\SpecialCharTok{$}\NormalTok{contr\_ind[,}\DecValTok{1}\NormalTok{],}
    \FunctionTok{ACP\_FINAL}\NormalTok{(PCA)}\SpecialCharTok{$}\NormalTok{cos2\_ind[,}\DecValTok{1}\NormalTok{],}\FunctionTok{ACP\_FINAL}\NormalTok{(PCA)}\SpecialCharTok{$}\NormalTok{cor\_ind[,}\DecValTok{2}\NormalTok{],}
    \FunctionTok{ACP\_FINAL}\NormalTok{(PCA)}\SpecialCharTok{$}\NormalTok{contr\_ind[,}\DecValTok{2}\NormalTok{], }\FunctionTok{ACP\_FINAL}\NormalTok{(PCA)}\SpecialCharTok{$}\NormalTok{cos2\_ind[,}\DecValTok{2}\NormalTok{])}
                       
  
  \FunctionTok{colnames}\NormalTok{(individuals) }\OtherTok{\textless{}{-}} \FunctionTok{c}\NormalTok{(}\StringTok{"Dim 1"}\NormalTok{,}\StringTok{"ctr"}\NormalTok{, }\StringTok{"cos2"}\NormalTok{, }\StringTok{"Dim 2"}\NormalTok{,}\StringTok{"ctr"}\NormalTok{, }\StringTok{"cos2"}\NormalTok{)}
  
  \FunctionTok{print}\NormalTok{(}\StringTok{"Individuals"}\NormalTok{)}
  \FunctionTok{print}\NormalTok{(individuals[}\DecValTok{1}\SpecialCharTok{:}\NormalTok{nbelements,])}
  
  \CommentTok{\#Variables}
\NormalTok{  variables }\OtherTok{\textless{}{-}} \FunctionTok{cbind}\NormalTok{(}\FunctionTok{ACP\_FINAL}\NormalTok{(PCA)}\SpecialCharTok{$}\NormalTok{cor\_var[,}\DecValTok{1}\NormalTok{], }\FunctionTok{ACP\_FINAL}\NormalTok{(PCA)}\SpecialCharTok{$}\NormalTok{contr\_var[,}\DecValTok{1}\NormalTok{],}
                     \FunctionTok{ACP\_FINAL}\NormalTok{(PCA)}\SpecialCharTok{$}\NormalTok{cos2\_var[,}\DecValTok{1}\NormalTok{],}\FunctionTok{ACP\_FINAL}\NormalTok{(PCA)}\SpecialCharTok{$}\NormalTok{cor\_var[,}\DecValTok{2}\NormalTok{],}
                     \FunctionTok{ACP\_FINAL}\NormalTok{(PCA)}\SpecialCharTok{$}\NormalTok{contr\_var[,}\DecValTok{2}\NormalTok{], }\FunctionTok{ACP\_FINAL}\NormalTok{(PCA)}\SpecialCharTok{$}\NormalTok{cos2\_var[,}\DecValTok{2}\NormalTok{])}

\NormalTok{  nom\_Var }\OtherTok{\textless{}{-}} \FunctionTok{colnames}\NormalTok{(}\FunctionTok{ACP\_FINAL}\NormalTok{(PCA)}\SpecialCharTok{$}\NormalTok{data)}
  \FunctionTok{row.names}\NormalTok{(variables) }\OtherTok{\textless{}{-}}\NormalTok{ nom\_Var}
  \FunctionTok{colnames}\NormalTok{(variables) }\OtherTok{\textless{}{-}} \FunctionTok{c}\NormalTok{(}\StringTok{"Corr 1"}\NormalTok{, }\StringTok{"Cos2 1"}\NormalTok{, }\StringTok{"Ctr 1"}\NormalTok{,}\StringTok{"Corr 2"}\NormalTok{, }\StringTok{"Cos2 2"}\NormalTok{, }\StringTok{"Ctr 2"}\NormalTok{)}
  
  \FunctionTok{print}\NormalTok{(}\StringTok{"Variables"}\NormalTok{)}
  \FunctionTok{print}\NormalTok{(}\FunctionTok{head}\NormalTok{(variables))}
\NormalTok{\}}
\FunctionTok{summary\_FINAL}\NormalTok{(X, }\AttributeTok{nbelements =} \DecValTok{4}\NormalTok{)}
\end{Highlighting}
\end{Shaded}

\begin{verbatim}
## [1] "Eigenvalues"
##       Variance % of var. Cumulative % of var
## [1,] 3.2719055 32.719055            32.71906
## [2,] 1.7371310 17.371310            50.09037
## [3,] 1.4049167 14.049167            64.13953
## [4,] 1.0568504 10.568504            74.70804
## [5,] 0.6847735  6.847735            81.55577
## [6,] 0.5992687  5.992687            87.54846
## [1] "Individuals"
##            Dim 1       ctr      cos2      Dim 2         ctr         cos2
## Sebrle -3.988895 12.157506 0.6954102 1.34906683 2.619234357 7.954314e-02
## Clay   -3.871273 11.451090 0.7112052 0.82669151 0.983545343 3.243204e-02
## Karpov -4.563298 15.910981 0.8517553 0.03950447 0.002245949 6.383365e-05
## Macey  -2.206055  3.718536 0.4230486 1.02898331 1.523786399 9.203950e-02
## [1] "Variables"
##              Corr 1    Cos2 1       Ctr 1     Corr 2    Cos2 2       Ctr 2
## 100m      0.7747198 18.343770 0.600190812  0.1871420  2.016090 0.550415232
## Longueur -0.7418997 16.822467 0.035022125 -0.3454213  6.868559 0.119315870
## Poids    -0.6225026 11.843540 0.034005993  0.5983033 20.606785 0.033200868
## Hauteur  -0.5719453  9.997887 0.001430221  0.3502936  7.063694 0.010360317
## 400m      0.6796099 14.116229 0.091322660  0.5694378 18.666374 0.001345279
## 110m H    0.7462453 17.020115 0.052532985  0.2287933  3.013382 0.056158895
\end{verbatim}

\hypertarget{variables}{%
\subsubsection{Variables}\label{variables}}

\begin{itemize}
\tightlist
\item
  Lors de l'affichage de l'ACP pour les individus et les variables,
  afficher en plus les axes des abscisses et des ordonnées. Afficher les
  variables sous forme de vecteurs
\end{itemize}

\begin{Shaded}
\begin{Highlighting}[]
\NormalTok{variablesACP\_FINAL }\OtherTok{\textless{}{-}} \ControlFlowTok{function}\NormalTok{(axes, PCA) \{ }

\NormalTok{  V }\OtherTok{\textless{}{-}} \FunctionTok{ACP\_FINAL}\NormalTok{(PCA)}\SpecialCharTok{$}\NormalTok{cor\_var}
  
\NormalTok{  dim1 }\OtherTok{\textless{}{-}} \FunctionTok{paste}\NormalTok{(}\StringTok{"Dim"}\NormalTok{,}\FunctionTok{as.character}\NormalTok{(axes[}\DecValTok{1}\NormalTok{]))}
\NormalTok{  dim2 }\OtherTok{\textless{}{-}} \FunctionTok{paste}\NormalTok{(}\StringTok{"Dim"}\NormalTok{,}\FunctionTok{as.character}\NormalTok{(axes[}\DecValTok{2}\NormalTok{]))}
  
  \FunctionTok{par}\NormalTok{(}\AttributeTok{mar=}\FunctionTok{c}\NormalTok{(}\DecValTok{4}\NormalTok{,}\DecValTok{4}\NormalTok{,}\DecValTok{4}\NormalTok{,}\DecValTok{4}\NormalTok{),}\AttributeTok{pty =} \StringTok{"s"}\NormalTok{)}
  \FunctionTok{plot}\NormalTok{(V[,axes],}\AttributeTok{xlim =} \FunctionTok{c}\NormalTok{(}\SpecialCharTok{{-}}\DecValTok{1}\NormalTok{,}\DecValTok{1}\NormalTok{),}\AttributeTok{ylim =} \FunctionTok{c}\NormalTok{(}\SpecialCharTok{{-}}\DecValTok{1}\NormalTok{,}\DecValTok{1}\NormalTok{),}\AttributeTok{xlab=}\NormalTok{dim1,}\AttributeTok{ylab=}\NormalTok{dim2, }\AttributeTok{cex=}\DecValTok{0}\NormalTok{)}
  \FunctionTok{text}\NormalTok{(V[,axes], }\AttributeTok{labels =} \FunctionTok{colnames}\NormalTok{(}\FunctionTok{ACP\_FINAL}\NormalTok{(PCA)}\SpecialCharTok{$}\NormalTok{data), }\AttributeTok{cex =} \FloatTok{0.6}\NormalTok{, }\AttributeTok{pos =} \DecValTok{1}\NormalTok{)}
  \FunctionTok{abline}\NormalTok{(}\AttributeTok{v =} \DecValTok{0}\NormalTok{, }\AttributeTok{lty =} \DecValTok{2}\NormalTok{)}
  \FunctionTok{abline}\NormalTok{(}\AttributeTok{h =} \DecValTok{0}\NormalTok{, }\AttributeTok{lty =} \DecValTok{2}\NormalTok{)}
  \FunctionTok{title}\NormalTok{(}\StringTok{"Graphique des variables"}\NormalTok{)}
  \FunctionTok{draw.circle}\NormalTok{(}\DecValTok{0}\NormalTok{, }\DecValTok{0}\NormalTok{, }\DecValTok{1}\NormalTok{, }\AttributeTok{nv =} \DecValTok{500}\NormalTok{, }\AttributeTok{border =} \ConstantTok{NULL}\NormalTok{, }\AttributeTok{col =} \ConstantTok{NA}\NormalTok{, }\AttributeTok{lty =} \DecValTok{1}\NormalTok{, }\AttributeTok{lwd =} \DecValTok{1}\NormalTok{)}
  \FunctionTok{arrows}\NormalTok{(}\AttributeTok{x0 =} \FunctionTok{rep}\NormalTok{(}\DecValTok{0}\NormalTok{, }\AttributeTok{times =} \FunctionTok{length}\NormalTok{(V[,}\DecValTok{1}\NormalTok{])), }
         \AttributeTok{y0 =} \FunctionTok{rep}\NormalTok{(}\DecValTok{0}\NormalTok{, }\AttributeTok{times =} \FunctionTok{length}\NormalTok{(V[,}\DecValTok{1}\NormalTok{])), }
         \AttributeTok{x1 =}\NormalTok{ V[,}\DecValTok{1}\NormalTok{], }\AttributeTok{y1 =}\NormalTok{ V[,}\DecValTok{2}\NormalTok{], }
         \AttributeTok{code =} \DecValTok{2}\NormalTok{, }\AttributeTok{length =} \FloatTok{0.1}\NormalTok{)}

\NormalTok{\}}
\FunctionTok{variablesACP\_FINAL}\NormalTok{(}\AttributeTok{axes=}\DecValTok{1}\SpecialCharTok{:}\DecValTok{2}\NormalTok{, X)}
\end{Highlighting}
\end{Shaded}

\includegraphics{Sujet_TD3_ACP_files/figure-latex/unnamed-chunk-19-1.pdf}

\hypertarget{annexes}{%
\section{Annexes}\label{annexes}}

Voici les codes intermédiaires entre les premières et les dernières
fonctions que nous avons créées tout au long de ce projet. Par exemple,
pour l'ACP, entre ACP et ACP\_FINAL nous avons créé ACP\_N.

\begin{Shaded}
\begin{Highlighting}[]
\NormalTok{ACP\_N }\OtherTok{\textless{}{-}} \ControlFlowTok{function}\NormalTok{(X, }\AttributeTok{quanti.sup=}\ConstantTok{NULL}\NormalTok{, }\AttributeTok{quali.sup=}\ConstantTok{NULL}\NormalTok{, }\AttributeTok{cr =} \ConstantTok{TRUE}\NormalTok{) \{}

\NormalTok{  supp }\OtherTok{\textless{}{-}} \FunctionTok{c}\NormalTok{()}
  
  \ControlFlowTok{if}\NormalTok{ (}\SpecialCharTok{!}\FunctionTok{is.null}\NormalTok{(quanti.sup))\{}
    \ControlFlowTok{for}\NormalTok{ (i }\ControlFlowTok{in}\NormalTok{ quanti.sup) \{}
\NormalTok{      supp }\OtherTok{\textless{}{-}} \FunctionTok{append}\NormalTok{(supp, i)}
\NormalTok{    \}}
\NormalTok{  \}}

  \ControlFlowTok{if}\NormalTok{ (}\SpecialCharTok{!}\FunctionTok{is.null}\NormalTok{(quali.sup)) \{}
    \ControlFlowTok{for}\NormalTok{ (i }\ControlFlowTok{in}\NormalTok{ quali.sup) \{}
\NormalTok{      supp }\OtherTok{\textless{}{-}} \FunctionTok{append}\NormalTok{(supp, i)}
\NormalTok{    \}}
\NormalTok{  \}}
  
  \ControlFlowTok{if}\NormalTok{ (}\SpecialCharTok{!}\FunctionTok{is.null}\NormalTok{(supp)) \{}
\NormalTok{    supp }\OtherTok{\textless{}{-}} \FunctionTok{sort}\NormalTok{(supp)}
\NormalTok{    X }\OtherTok{\textless{}{-}}\NormalTok{ X[,}\SpecialCharTok{{-}}\NormalTok{supp]}
\NormalTok{  \}}
  
  \ControlFlowTok{if}\NormalTok{ (cr }\SpecialCharTok{==} \ConstantTok{FALSE}\NormalTok{) \{}
\NormalTok{    data }\OtherTok{\textless{}{-}}\NormalTok{ X}
\NormalTok{  \}}
  \ControlFlowTok{else}\NormalTok{ \{}
\NormalTok{    data }\OtherTok{\textless{}{-}} \FunctionTok{scale}\NormalTok{(X)}
\NormalTok{  \}}

\NormalTok{  cov }\OtherTok{\textless{}{-}} \FunctionTok{cov}\NormalTok{(data) }\CommentTok{\#ne peut pas faire le scale d\textquotesingle{}une variable qualitative}
\NormalTok{  vap }\OtherTok{\textless{}{-}} \FunctionTok{eigen}\NormalTok{(cov)}\SpecialCharTok{$}\NormalTok{values}
\NormalTok{  vep }\OtherTok{\textless{}{-}} \FunctionTok{eigen}\NormalTok{(cov)}\SpecialCharTok{$}\NormalTok{vectors}
  
  \FunctionTok{return}\NormalTok{(}\FunctionTok{list}\NormalTok{(}\StringTok{"vap"}\OtherTok{=}\NormalTok{vap,}\StringTok{"vep"}\OtherTok{=}\NormalTok{vep,}\StringTok{"data"}\OtherTok{=}\NormalTok{data,}\StringTok{"cov"}\OtherTok{=}\NormalTok{cov))}
\NormalTok{\}}
\end{Highlighting}
\end{Shaded}

Individus

\begin{Shaded}
\begin{Highlighting}[]
\NormalTok{individusACP\_N }\OtherTok{\textless{}{-}} \ControlFlowTok{function}\NormalTok{(axes, PCA, habillage) \{}
\NormalTok{  V }\OtherTok{\textless{}{-}} \FunctionTok{as.matrix}\NormalTok{(}\FunctionTok{ACP\_N}\NormalTok{(X)}\SpecialCharTok{$}\NormalTok{vep)}
\NormalTok{  D }\OtherTok{\textless{}{-}} \FunctionTok{as.matrix}\NormalTok{(}\FunctionTok{ACP\_N}\NormalTok{(X)}\SpecialCharTok{$}\NormalTok{data)}
\NormalTok{  res }\OtherTok{\textless{}{-}}\NormalTok{ D}\SpecialCharTok{\%*\%}\NormalTok{V}
\NormalTok{  colour }\OtherTok{\textless{}{-}} \FunctionTok{c}\NormalTok{()}
  
  \ControlFlowTok{for}\NormalTok{ (i }\ControlFlowTok{in} \FunctionTok{seq.int}\NormalTok{(}\DecValTok{1}\SpecialCharTok{:}\FunctionTok{nrow}\NormalTok{(res))) \{}
    \ControlFlowTok{if}\NormalTok{ (}\FunctionTok{abs}\NormalTok{(res[i,habillage]) }\SpecialCharTok{\textless{}} \DecValTok{1}\NormalTok{) \{}
\NormalTok{      clr }\OtherTok{\textless{}{-}} \StringTok{"green"}
\NormalTok{    \}}
    \ControlFlowTok{else} \ControlFlowTok{if}\NormalTok{ (}\FunctionTok{abs}\NormalTok{(res[i,habillage]) }\SpecialCharTok{\textless{}} \FloatTok{2.5}\NormalTok{) \{}
\NormalTok{      clr }\OtherTok{\textless{}{-}} \StringTok{"orange"}
\NormalTok{    \}}
    \ControlFlowTok{else}\NormalTok{ \{}
\NormalTok{      clr }\OtherTok{\textless{}{-}} \StringTok{"red"}
\NormalTok{    \}}
\NormalTok{    colour }\OtherTok{\textless{}{-}} \FunctionTok{append}\NormalTok{(colour, clr)}
\NormalTok{  \}}

  
\NormalTok{  dim1 }\OtherTok{\textless{}{-}} \FunctionTok{paste}\NormalTok{(}\StringTok{"Dim"}\NormalTok{,}\FunctionTok{as.character}\NormalTok{(axes[}\DecValTok{1}\NormalTok{]))}
\NormalTok{  dim2 }\OtherTok{\textless{}{-}} \FunctionTok{paste}\NormalTok{(}\StringTok{"Dim"}\NormalTok{,}\FunctionTok{as.character}\NormalTok{(axes[}\DecValTok{2}\NormalTok{]))}
  
  \FunctionTok{par}\NormalTok{(}\AttributeTok{mar=}\FunctionTok{c}\NormalTok{(}\DecValTok{4}\NormalTok{,}\DecValTok{4}\NormalTok{,}\DecValTok{4}\NormalTok{,}\DecValTok{4}\NormalTok{),}\AttributeTok{pty =} \StringTok{"s"}\NormalTok{)}
  \FunctionTok{plot}\NormalTok{(res[,axes], }\AttributeTok{xlab =}\NormalTok{ dim1, }\AttributeTok{ylab =}\NormalTok{ dim2, }\AttributeTok{pch =} \DecValTok{1}\NormalTok{, }\AttributeTok{col =}\NormalTok{ colour )}
  \FunctionTok{text}\NormalTok{(res[,axes], }\AttributeTok{labels =} \FunctionTok{row.names}\NormalTok{(}\FunctionTok{ACP\_N}\NormalTok{(PCA)}\SpecialCharTok{$}\NormalTok{data), }\AttributeTok{cex =} \FloatTok{0.6}\NormalTok{, }\AttributeTok{pos =} \DecValTok{1}\NormalTok{)}
  \FunctionTok{abline}\NormalTok{(}\AttributeTok{v =} \DecValTok{0}\NormalTok{, }\AttributeTok{lty =} \DecValTok{2}\NormalTok{)}
  \FunctionTok{abline}\NormalTok{(}\AttributeTok{h =} \DecValTok{0}\NormalTok{, }\AttributeTok{lty =} \DecValTok{2}\NormalTok{)}
  \FunctionTok{title}\NormalTok{(}\StringTok{"ACP"}\NormalTok{)}
  \FunctionTok{return}\NormalTok{(res)}
\NormalTok{\}}
\end{Highlighting}
\end{Shaded}

Summary

\begin{Shaded}
\begin{Highlighting}[]
\NormalTok{summary\_N }\OtherTok{\textless{}{-}} \ControlFlowTok{function}\NormalTok{(PCA,}\AttributeTok{nbelements=}\DecValTok{10}\NormalTok{) \{}
  
  \CommentTok{\#Eigenvalues}
\NormalTok{  vap }\OtherTok{\textless{}{-}} \FunctionTok{ACP\_N}\NormalTok{(PCA)}\SpecialCharTok{$}\NormalTok{vap}
\NormalTok{  vect }\OtherTok{\textless{}{-}} \FunctionTok{c}\NormalTok{()}
  \ControlFlowTok{for}\NormalTok{ (i }\ControlFlowTok{in} \FunctionTok{seq.int}\NormalTok{(}\DecValTok{0}\NormalTok{,}\FunctionTok{length}\NormalTok{(vap))) \{}
\NormalTok{    somme }\OtherTok{\textless{}{-}} \FunctionTok{sum}\NormalTok{(vap)}
\NormalTok{    pourcentage }\OtherTok{\textless{}{-}}\NormalTok{ (vap[i]}\SpecialCharTok{/}\NormalTok{somme)}\SpecialCharTok{*}\DecValTok{100}
\NormalTok{    vect }\OtherTok{\textless{}{-}} \FunctionTok{append}\NormalTok{(vect, pourcentage)}
\NormalTok{  \}}
\NormalTok{  eigenvalues }\OtherTok{\textless{}{-}} \FunctionTok{cbind}\NormalTok{(vap, vect,  }\FunctionTok{cumsum}\NormalTok{(vect))}
  \FunctionTok{colnames}\NormalTok{(eigenvalues) }\OtherTok{\textless{}{-}} \FunctionTok{c}\NormalTok{(}\StringTok{"Variance"}\NormalTok{, }\StringTok{"\% of var."}\NormalTok{, }\StringTok{"Cumulative \% of var"}\NormalTok{)}
  \FunctionTok{print}\NormalTok{(}\StringTok{"Eigenvalues"}\NormalTok{)}
  \FunctionTok{print}\NormalTok{(eigenvalues)}
  
  \CommentTok{\#Individuals}
\NormalTok{  dim }\OtherTok{\textless{}{-}} \FunctionTok{individusACP\_N}\NormalTok{(}\AttributeTok{axes=}\DecValTok{1}\SpecialCharTok{:}\DecValTok{2}\NormalTok{, PCA)}
   
    \CommentTok{\#Dim1}
\NormalTok{  contr\_1 }\OtherTok{\textless{}{-}} \FunctionTok{c}\NormalTok{()}
\NormalTok{  sommecarre\_1 }\OtherTok{\textless{}{-}}\NormalTok{ dim[,}\DecValTok{1}\NormalTok{]}\SpecialCharTok{\%*\%}\NormalTok{dim[,}\DecValTok{1}\NormalTok{]}
  \ControlFlowTok{for}\NormalTok{ (i }\ControlFlowTok{in} \FunctionTok{seq.int}\NormalTok{(}\DecValTok{0}\NormalTok{,}\FunctionTok{length}\NormalTok{(dim[,}\DecValTok{1}\NormalTok{]))) \{}
\NormalTok{    contr\_1 }\OtherTok{\textless{}{-}} \FunctionTok{append}\NormalTok{(contr\_1, ((dim[,}\DecValTok{1}\NormalTok{][i])}\SpecialCharTok{**}\DecValTok{2}\SpecialCharTok{/}\NormalTok{sommecarre\_1)}\SpecialCharTok{*}\DecValTok{100}\NormalTok{)}
\NormalTok{  \}}
\NormalTok{  individuals }\OtherTok{\textless{}{-}} \FunctionTok{cbind}\NormalTok{(dim[,}\DecValTok{1}\NormalTok{], contr\_1)}
  
\NormalTok{  cos2\_1 }\OtherTok{\textless{}{-}} \FunctionTok{c}\NormalTok{()}
  \ControlFlowTok{for}\NormalTok{ (i }\ControlFlowTok{in} \FunctionTok{seq.int}\NormalTok{(}\DecValTok{0}\NormalTok{,}\FunctionTok{length}\NormalTok{(dim[,}\DecValTok{1}\NormalTok{]))) \{}
\NormalTok{    cos2\_1 }\OtherTok{\textless{}{-}} \FunctionTok{append}\NormalTok{(cos2\_1, dim[i,}\DecValTok{1}\NormalTok{]}\SpecialCharTok{**}\DecValTok{2}\SpecialCharTok{/}\FunctionTok{sqrt}\NormalTok{(}\FunctionTok{sum}\NormalTok{(dim[i,]}\SpecialCharTok{**}\DecValTok{2}\NormalTok{))}\SpecialCharTok{**}\DecValTok{2}\NormalTok{)}
\NormalTok{  \}}
\NormalTok{  individuals }\OtherTok{\textless{}{-}} \FunctionTok{cbind}\NormalTok{(individuals, cos2\_1)}
  
    \CommentTok{\#Dim2}
\NormalTok{  contr\_2 }\OtherTok{\textless{}{-}} \FunctionTok{c}\NormalTok{()}
\NormalTok{  sommecarre\_2 }\OtherTok{\textless{}{-}}\NormalTok{ dim[,}\DecValTok{2}\NormalTok{]}\SpecialCharTok{\%*\%}\NormalTok{dim[,}\DecValTok{2}\NormalTok{]}
\NormalTok{  individuals }\OtherTok{\textless{}{-}} \FunctionTok{cbind}\NormalTok{(individuals, dim[,}\DecValTok{2}\NormalTok{])}
  
  \ControlFlowTok{for}\NormalTok{ (i }\ControlFlowTok{in} \FunctionTok{seq.int}\NormalTok{(}\DecValTok{0}\NormalTok{,}\FunctionTok{length}\NormalTok{(dim[,}\DecValTok{2}\NormalTok{]))) \{}
\NormalTok{    contr\_2 }\OtherTok{\textless{}{-}} \FunctionTok{append}\NormalTok{(contr\_2, ((dim[,}\DecValTok{2}\NormalTok{][i])}\SpecialCharTok{**}\DecValTok{2}\SpecialCharTok{/}\NormalTok{sommecarre\_2)}\SpecialCharTok{*}\DecValTok{100}\NormalTok{)}
\NormalTok{  \}}
\NormalTok{  individuals }\OtherTok{\textless{}{-}} \FunctionTok{cbind}\NormalTok{(individuals, contr\_2)}

\NormalTok{  cos2\_2 }\OtherTok{\textless{}{-}} \FunctionTok{c}\NormalTok{()}
  \ControlFlowTok{for}\NormalTok{ (i }\ControlFlowTok{in} \FunctionTok{seq.int}\NormalTok{(}\DecValTok{0}\NormalTok{,}\FunctionTok{length}\NormalTok{(dim[,}\DecValTok{2}\NormalTok{]))) \{}
\NormalTok{    cos2\_2 }\OtherTok{\textless{}{-}} \FunctionTok{append}\NormalTok{(cos2\_2, dim[i,}\DecValTok{2}\NormalTok{]}\SpecialCharTok{**}\DecValTok{2}\SpecialCharTok{/}\FunctionTok{sqrt}\NormalTok{(}\FunctionTok{sum}\NormalTok{(dim[i,]}\SpecialCharTok{**}\DecValTok{2}\NormalTok{))}\SpecialCharTok{**}\DecValTok{2}\NormalTok{)}
\NormalTok{  \}}
\NormalTok{  individuals }\OtherTok{\textless{}{-}} \FunctionTok{cbind}\NormalTok{(individuals, cos2\_2)}
  
  \FunctionTok{colnames}\NormalTok{(individuals) }\OtherTok{\textless{}{-}} \FunctionTok{c}\NormalTok{(}\StringTok{"Dim 1"}\NormalTok{,}\StringTok{"ctr"}\NormalTok{, }\StringTok{"cos2"}\NormalTok{, }\StringTok{"Dim 2"}\NormalTok{,}\StringTok{"ctr"}\NormalTok{, }\StringTok{"cos2"}\NormalTok{)}
  
  \FunctionTok{print}\NormalTok{(}\StringTok{"Individuals"}\NormalTok{)}
  \FunctionTok{print}\NormalTok{(individuals[}\DecValTok{1}\SpecialCharTok{:}\NormalTok{nbelements,])}
  
  \CommentTok{\#Variables}
    \CommentTok{\#Dim1}
\NormalTok{  corr\_1 }\OtherTok{\textless{}{-}} \FunctionTok{c}\NormalTok{()}
  \ControlFlowTok{for}\NormalTok{ (i }\ControlFlowTok{in} \FunctionTok{seq.int}\NormalTok{(}\DecValTok{0}\NormalTok{, }\FunctionTok{length}\NormalTok{(vap))) \{}
\NormalTok{    corr\_1 }\OtherTok{\textless{}{-}} \FunctionTok{append}\NormalTok{(corr\_1, }\FunctionTok{sqrt}\NormalTok{(vap[}\DecValTok{1}\NormalTok{])}\SpecialCharTok{*}\FunctionTok{ACP\_N}\NormalTok{(PCA)}\SpecialCharTok{$}\NormalTok{vep[i,}\DecValTok{1}\NormalTok{])}
\NormalTok{  \}}
\NormalTok{  variables }\OtherTok{\textless{}{-}} \FunctionTok{as.data.frame}\NormalTok{(corr\_1)}

\NormalTok{  cos2\_1 }\OtherTok{\textless{}{-}} \FunctionTok{c}\NormalTok{()}
  \ControlFlowTok{for}\NormalTok{ (i }\ControlFlowTok{in} \FunctionTok{seq.int}\NormalTok{(}\DecValTok{0}\NormalTok{, }\FunctionTok{length}\NormalTok{(corr\_1))) \{}
\NormalTok{    cos2\_1 }\OtherTok{\textless{}{-}} \FunctionTok{append}\NormalTok{(cos2\_1, corr\_1[i]}\SpecialCharTok{**}\DecValTok{2}\NormalTok{)}
\NormalTok{  \}}
\NormalTok{  variables }\OtherTok{\textless{}{-}} \FunctionTok{cbind}\NormalTok{(variables, cos2\_1)}
  
\NormalTok{  contr\_1 }\OtherTok{\textless{}{-}} \FunctionTok{c}\NormalTok{()}
\NormalTok{  sumSquare1 }\OtherTok{\textless{}{-}} \FunctionTok{ACP\_N}\NormalTok{(PCA)}\SpecialCharTok{$}\NormalTok{vep[,}\DecValTok{1}\NormalTok{]}\SpecialCharTok{\%*\%}\FunctionTok{ACP\_N}\NormalTok{(PCA)}\SpecialCharTok{$}\NormalTok{vep[,}\DecValTok{1}\NormalTok{]}
  \ControlFlowTok{for}\NormalTok{ (i }\ControlFlowTok{in} \FunctionTok{seq.int}\NormalTok{(}\DecValTok{0}\NormalTok{,}\FunctionTok{length}\NormalTok{(}\FunctionTok{ACP\_N}\NormalTok{(PCA)}\SpecialCharTok{$}\NormalTok{vep[,}\DecValTok{1}\NormalTok{]))) \{}
\NormalTok{    contr\_1 }\OtherTok{\textless{}{-}} \FunctionTok{append}\NormalTok{(contr\_1, (}\FunctionTok{ACP\_N}\NormalTok{(PCA)}\SpecialCharTok{$}\NormalTok{vep[i,}\DecValTok{1}\NormalTok{]}\SpecialCharTok{**}\DecValTok{2}\SpecialCharTok{/}\NormalTok{sumSquare1)}\SpecialCharTok{*}\DecValTok{100}\NormalTok{)}
\NormalTok{  \}}
\NormalTok{  variables }\OtherTok{\textless{}{-}} \FunctionTok{cbind}\NormalTok{(variables, contr\_1)}
  
    \CommentTok{\#Dim2}
\NormalTok{  corr\_2 }\OtherTok{\textless{}{-}} \FunctionTok{c}\NormalTok{()}
  \ControlFlowTok{for}\NormalTok{ (i }\ControlFlowTok{in} \FunctionTok{seq.int}\NormalTok{(}\DecValTok{0}\NormalTok{, }\FunctionTok{length}\NormalTok{(vap))) \{}
\NormalTok{    corr\_2 }\OtherTok{\textless{}{-}} \FunctionTok{append}\NormalTok{(corr\_2, }\FunctionTok{sqrt}\NormalTok{(vap[}\DecValTok{2}\NormalTok{])}\SpecialCharTok{*}\FunctionTok{ACP\_N}\NormalTok{(PCA)}\SpecialCharTok{$}\NormalTok{vep[i,}\DecValTok{2}\NormalTok{])}
\NormalTok{  \}}
\NormalTok{  variables }\OtherTok{\textless{}{-}} \FunctionTok{cbind}\NormalTok{(variables, corr\_2)}
  
\NormalTok{  cos2\_2 }\OtherTok{\textless{}{-}} \FunctionTok{c}\NormalTok{()}
  \ControlFlowTok{for}\NormalTok{ (i }\ControlFlowTok{in} \FunctionTok{seq.int}\NormalTok{(}\DecValTok{0}\NormalTok{, }\FunctionTok{length}\NormalTok{(corr\_2))) \{}
\NormalTok{    cos2\_2 }\OtherTok{\textless{}{-}} \FunctionTok{append}\NormalTok{(cos2\_2, corr\_2[i]}\SpecialCharTok{**}\DecValTok{2}\NormalTok{)}
\NormalTok{  \}}
\NormalTok{  variables }\OtherTok{\textless{}{-}} \FunctionTok{cbind}\NormalTok{(variables, cos2\_2)}

\NormalTok{  contr\_2 }\OtherTok{\textless{}{-}} \FunctionTok{c}\NormalTok{()}
\NormalTok{  sumSquare2 }\OtherTok{\textless{}{-}} \FunctionTok{ACP\_N}\NormalTok{(PCA)}\SpecialCharTok{$}\NormalTok{vep[,}\DecValTok{2}\NormalTok{]}\SpecialCharTok{\%*\%}\FunctionTok{ACP\_N}\NormalTok{(PCA)}\SpecialCharTok{$}\NormalTok{vep[,}\DecValTok{2}\NormalTok{]}
  \ControlFlowTok{for}\NormalTok{ (i }\ControlFlowTok{in} \FunctionTok{seq.int}\NormalTok{(}\DecValTok{0}\NormalTok{,}\FunctionTok{length}\NormalTok{(}\FunctionTok{ACP\_N}\NormalTok{(PCA)}\SpecialCharTok{$}\NormalTok{vep[,}\DecValTok{2}\NormalTok{]))) \{}
\NormalTok{    contr\_2 }\OtherTok{\textless{}{-}} \FunctionTok{append}\NormalTok{(contr\_2, (}\FunctionTok{ACP\_N}\NormalTok{(PCA)}\SpecialCharTok{$}\NormalTok{vep[i,}\DecValTok{2}\NormalTok{]}\SpecialCharTok{**}\DecValTok{2}\SpecialCharTok{/}\NormalTok{sumSquare2)}\SpecialCharTok{*}\DecValTok{100}\NormalTok{)}
\NormalTok{  \}}
\NormalTok{  variables }\OtherTok{\textless{}{-}} \FunctionTok{cbind}\NormalTok{(variables, contr\_2)}
  
\NormalTok{  nom\_Var }\OtherTok{\textless{}{-}} \FunctionTok{colnames}\NormalTok{(}\FunctionTok{ACP\_N}\NormalTok{(PCA)}\SpecialCharTok{$}\NormalTok{data)}
  \FunctionTok{row.names}\NormalTok{(variables) }\OtherTok{\textless{}{-}}\NormalTok{ nom\_Var}
  \FunctionTok{colnames}\NormalTok{(variables) }\OtherTok{\textless{}{-}} \FunctionTok{c}\NormalTok{(}\StringTok{"Corr 1"}\NormalTok{, }\StringTok{"Cos2 1"}\NormalTok{, }\StringTok{"Ctr 1"}\NormalTok{,}\StringTok{"Corr 2"}\NormalTok{, }\StringTok{"Cos2 2"}\NormalTok{, }\StringTok{"Ctr 2"}\NormalTok{)}
  
  \FunctionTok{print}\NormalTok{(}\StringTok{"Variables"}\NormalTok{)}
  \FunctionTok{print}\NormalTok{(variables)}
  
\NormalTok{\}}
\end{Highlighting}
\end{Shaded}

Variables

\begin{Shaded}
\begin{Highlighting}[]
\NormalTok{variablesACP\_N }\OtherTok{\textless{}{-}} \ControlFlowTok{function}\NormalTok{(axes, PCA) \{ }
\NormalTok{  vep }\OtherTok{\textless{}{-}} \FunctionTok{ACP\_N}\NormalTok{(PCA)}\SpecialCharTok{$}\NormalTok{vep}
\NormalTok{  vap }\OtherTok{\textless{}{-}} \FunctionTok{ACP\_N}\NormalTok{(PCA)}\SpecialCharTok{$}\NormalTok{vap}
\NormalTok{  cor\_v }\OtherTok{\textless{}{-}} \FunctionTok{c}\NormalTok{()}
\NormalTok{  data }\OtherTok{\textless{}{-}} \FunctionTok{ACP\_N}\NormalTok{(PCA)}\SpecialCharTok{$}\NormalTok{data}
\NormalTok{  taille\_v }\OtherTok{\textless{}{-}} \FunctionTok{length}\NormalTok{(vep[,}\DecValTok{1}\NormalTok{])}
  
  \ControlFlowTok{for}\NormalTok{ (k }\ControlFlowTok{in} \FunctionTok{seq.int}\NormalTok{(}\DecValTok{1}\NormalTok{, taille\_v)) \{}
\NormalTok{    cor\_dv }\OtherTok{\textless{}{-}} \FunctionTok{c}\NormalTok{()}
    \ControlFlowTok{for}\NormalTok{ (i }\ControlFlowTok{in} \FunctionTok{seq.int}\NormalTok{(}\DecValTok{0}\NormalTok{, }\FunctionTok{length}\NormalTok{(vap))) \{}
\NormalTok{      cor\_dv }\OtherTok{\textless{}{-}} \FunctionTok{append}\NormalTok{(cor\_dv, }\FunctionTok{sqrt}\NormalTok{(vap[k])}\SpecialCharTok{*}\NormalTok{vep[i,k])}
\NormalTok{    \}}
\NormalTok{    cor\_v }\OtherTok{\textless{}{-}} \FunctionTok{cbind}\NormalTok{(cor\_v, cor\_dv)}
\NormalTok{  \}}

\NormalTok{  dim1 }\OtherTok{\textless{}{-}} \FunctionTok{paste}\NormalTok{(}\StringTok{"Dim"}\NormalTok{,}\FunctionTok{as.character}\NormalTok{(axes[}\DecValTok{1}\NormalTok{]))}
\NormalTok{  dim2 }\OtherTok{\textless{}{-}} \FunctionTok{paste}\NormalTok{(}\StringTok{"Dim"}\NormalTok{,}\FunctionTok{as.character}\NormalTok{(axes[}\DecValTok{2}\NormalTok{]))}
  
  \FunctionTok{par}\NormalTok{(}\AttributeTok{mar=}\FunctionTok{c}\NormalTok{(}\DecValTok{4}\NormalTok{,}\DecValTok{4}\NormalTok{,}\DecValTok{4}\NormalTok{,}\DecValTok{4}\NormalTok{),}\AttributeTok{pty =} \StringTok{"s"}\NormalTok{)}
  \FunctionTok{plot}\NormalTok{(cor\_v[,axes],}\AttributeTok{xlim =} \FunctionTok{c}\NormalTok{(}\SpecialCharTok{{-}}\DecValTok{1}\NormalTok{,}\DecValTok{1}\NormalTok{),}\AttributeTok{ylim =} \FunctionTok{c}\NormalTok{(}\SpecialCharTok{{-}}\DecValTok{1}\NormalTok{,}\DecValTok{1}\NormalTok{),}\AttributeTok{xlab=}\NormalTok{dim1,}\AttributeTok{ylab=}\NormalTok{dim2, }\AttributeTok{cex =} \DecValTok{0}\NormalTok{)}
  \FunctionTok{text}\NormalTok{(cor\_v[,axes], }\AttributeTok{labels =} \FunctionTok{colnames}\NormalTok{(data), }\AttributeTok{cex =} \FloatTok{0.6}\NormalTok{, }\AttributeTok{pos =} \DecValTok{1}\NormalTok{)}
  \FunctionTok{abline}\NormalTok{(}\AttributeTok{v =} \DecValTok{0}\NormalTok{, }\AttributeTok{lty =} \DecValTok{2}\NormalTok{)}
  \FunctionTok{abline}\NormalTok{(}\AttributeTok{h =} \DecValTok{0}\NormalTok{, }\AttributeTok{lty =} \DecValTok{2}\NormalTok{)}
  \FunctionTok{title}\NormalTok{(}\StringTok{"Graphique des variables"}\NormalTok{)}
  \FunctionTok{draw.circle}\NormalTok{(}\DecValTok{0}\NormalTok{, }\DecValTok{0}\NormalTok{, }\DecValTok{1}\NormalTok{, }\AttributeTok{nv =} \DecValTok{500}\NormalTok{, }\AttributeTok{border =} \ConstantTok{NULL}\NormalTok{, }\AttributeTok{col =} \ConstantTok{NA}\NormalTok{, }\AttributeTok{lty =} \DecValTok{1}\NormalTok{, }\AttributeTok{lwd =} \DecValTok{1}\NormalTok{)}
  \FunctionTok{arrows}\NormalTok{(}\AttributeTok{x0 =} \FunctionTok{rep}\NormalTok{(}\DecValTok{0}\NormalTok{, }\AttributeTok{times =} \FunctionTok{length}\NormalTok{(cor\_v[,}\DecValTok{1}\NormalTok{])), }
         \AttributeTok{y0 =} \FunctionTok{rep}\NormalTok{(}\DecValTok{0}\NormalTok{, }\AttributeTok{times =} \FunctionTok{length}\NormalTok{(cor\_v[,}\DecValTok{1}\NormalTok{])), }
         \AttributeTok{x1 =}\NormalTok{ cor\_v[,}\DecValTok{1}\NormalTok{], }\AttributeTok{y1 =}\NormalTok{ cor\_v[,}\DecValTok{2}\NormalTok{], }
         \AttributeTok{code =} \DecValTok{2}\NormalTok{, }\AttributeTok{length =} \FloatTok{0.1}\NormalTok{)}

\NormalTok{\}}
\end{Highlighting}
\end{Shaded}


\end{document}
