% Options for packages loaded elsewhere
\PassOptionsToPackage{unicode}{hyperref}
\PassOptionsToPackage{hyphens}{url}
%
\documentclass[
]{article}
\title{TD4 - Partial Least Squares}
\author{Courtenay Rebecca \& Ducros Chloé \& Lasson Marie}
\date{}

\usepackage{amsmath,amssymb}
\usepackage{lmodern}
\usepackage{iftex}
\ifPDFTeX
  \usepackage[T1]{fontenc}
  \usepackage[utf8]{inputenc}
  \usepackage{textcomp} % provide euro and other symbols
\else % if luatex or xetex
  \usepackage{unicode-math}
  \defaultfontfeatures{Scale=MatchLowercase}
  \defaultfontfeatures[\rmfamily]{Ligatures=TeX,Scale=1}
\fi
% Use upquote if available, for straight quotes in verbatim environments
\IfFileExists{upquote.sty}{\usepackage{upquote}}{}
\IfFileExists{microtype.sty}{% use microtype if available
  \usepackage[]{microtype}
  \UseMicrotypeSet[protrusion]{basicmath} % disable protrusion for tt fonts
}{}
\makeatletter
\@ifundefined{KOMAClassName}{% if non-KOMA class
  \IfFileExists{parskip.sty}{%
    \usepackage{parskip}
  }{% else
    \setlength{\parindent}{0pt}
    \setlength{\parskip}{6pt plus 2pt minus 1pt}}
}{% if KOMA class
  \KOMAoptions{parskip=half}}
\makeatother
\usepackage{xcolor}
\IfFileExists{xurl.sty}{\usepackage{xurl}}{} % add URL line breaks if available
\IfFileExists{bookmark.sty}{\usepackage{bookmark}}{\usepackage{hyperref}}
\hypersetup{
  pdftitle={TD4 - Partial Least Squares},
  pdfauthor={Courtenay Rebecca \& Ducros Chloé \& Lasson Marie},
  hidelinks,
  pdfcreator={LaTeX via pandoc}}
\urlstyle{same} % disable monospaced font for URLs
\usepackage[margin=1in]{geometry}
\usepackage{color}
\usepackage{fancyvrb}
\newcommand{\VerbBar}{|}
\newcommand{\VERB}{\Verb[commandchars=\\\{\}]}
\DefineVerbatimEnvironment{Highlighting}{Verbatim}{commandchars=\\\{\}}
% Add ',fontsize=\small' for more characters per line
\usepackage{framed}
\definecolor{shadecolor}{RGB}{248,248,248}
\newenvironment{Shaded}{\begin{snugshade}}{\end{snugshade}}
\newcommand{\AlertTok}[1]{\textcolor[rgb]{0.94,0.16,0.16}{#1}}
\newcommand{\AnnotationTok}[1]{\textcolor[rgb]{0.56,0.35,0.01}{\textbf{\textit{#1}}}}
\newcommand{\AttributeTok}[1]{\textcolor[rgb]{0.77,0.63,0.00}{#1}}
\newcommand{\BaseNTok}[1]{\textcolor[rgb]{0.00,0.00,0.81}{#1}}
\newcommand{\BuiltInTok}[1]{#1}
\newcommand{\CharTok}[1]{\textcolor[rgb]{0.31,0.60,0.02}{#1}}
\newcommand{\CommentTok}[1]{\textcolor[rgb]{0.56,0.35,0.01}{\textit{#1}}}
\newcommand{\CommentVarTok}[1]{\textcolor[rgb]{0.56,0.35,0.01}{\textbf{\textit{#1}}}}
\newcommand{\ConstantTok}[1]{\textcolor[rgb]{0.00,0.00,0.00}{#1}}
\newcommand{\ControlFlowTok}[1]{\textcolor[rgb]{0.13,0.29,0.53}{\textbf{#1}}}
\newcommand{\DataTypeTok}[1]{\textcolor[rgb]{0.13,0.29,0.53}{#1}}
\newcommand{\DecValTok}[1]{\textcolor[rgb]{0.00,0.00,0.81}{#1}}
\newcommand{\DocumentationTok}[1]{\textcolor[rgb]{0.56,0.35,0.01}{\textbf{\textit{#1}}}}
\newcommand{\ErrorTok}[1]{\textcolor[rgb]{0.64,0.00,0.00}{\textbf{#1}}}
\newcommand{\ExtensionTok}[1]{#1}
\newcommand{\FloatTok}[1]{\textcolor[rgb]{0.00,0.00,0.81}{#1}}
\newcommand{\FunctionTok}[1]{\textcolor[rgb]{0.00,0.00,0.00}{#1}}
\newcommand{\ImportTok}[1]{#1}
\newcommand{\InformationTok}[1]{\textcolor[rgb]{0.56,0.35,0.01}{\textbf{\textit{#1}}}}
\newcommand{\KeywordTok}[1]{\textcolor[rgb]{0.13,0.29,0.53}{\textbf{#1}}}
\newcommand{\NormalTok}[1]{#1}
\newcommand{\OperatorTok}[1]{\textcolor[rgb]{0.81,0.36,0.00}{\textbf{#1}}}
\newcommand{\OtherTok}[1]{\textcolor[rgb]{0.56,0.35,0.01}{#1}}
\newcommand{\PreprocessorTok}[1]{\textcolor[rgb]{0.56,0.35,0.01}{\textit{#1}}}
\newcommand{\RegionMarkerTok}[1]{#1}
\newcommand{\SpecialCharTok}[1]{\textcolor[rgb]{0.00,0.00,0.00}{#1}}
\newcommand{\SpecialStringTok}[1]{\textcolor[rgb]{0.31,0.60,0.02}{#1}}
\newcommand{\StringTok}[1]{\textcolor[rgb]{0.31,0.60,0.02}{#1}}
\newcommand{\VariableTok}[1]{\textcolor[rgb]{0.00,0.00,0.00}{#1}}
\newcommand{\VerbatimStringTok}[1]{\textcolor[rgb]{0.31,0.60,0.02}{#1}}
\newcommand{\WarningTok}[1]{\textcolor[rgb]{0.56,0.35,0.01}{\textbf{\textit{#1}}}}
\usepackage{graphicx}
\makeatletter
\def\maxwidth{\ifdim\Gin@nat@width>\linewidth\linewidth\else\Gin@nat@width\fi}
\def\maxheight{\ifdim\Gin@nat@height>\textheight\textheight\else\Gin@nat@height\fi}
\makeatother
% Scale images if necessary, so that they will not overflow the page
% margins by default, and it is still possible to overwrite the defaults
% using explicit options in \includegraphics[width, height, ...]{}
\setkeys{Gin}{width=\maxwidth,height=\maxheight,keepaspectratio}
% Set default figure placement to htbp
\makeatletter
\def\fps@figure{htbp}
\makeatother
\setlength{\emergencystretch}{3em} % prevent overfull lines
\providecommand{\tightlist}{%
  \setlength{\itemsep}{0pt}\setlength{\parskip}{0pt}}
\setcounter{secnumdepth}{5}
\ifLuaTeX
  \usepackage{selnolig}  % disable illegal ligatures
\fi

\begin{document}
\maketitle

{
\setcounter{tocdepth}{2}
\tableofcontents
}
\hypertarget{introduction}{%
\section{Introduction}\label{introduction}}

\hypertarget{pca-vs-pls}{%
\subsection{PCA vs PLS}\label{pca-vs-pls}}

On se souvient que l'ACP correspondait à écrire une matrice de donnée
\(X\) dans la base des vecteurs propres \(V\). Si on note \(\tilde X\)
la matrice de données dans la nouvelle base, on avait la décomposition
suivante : \(X^T = V \cdot \tilde X^T\), soit :
\(X = \tilde X \cdot V^T\). Pour être tout à fait exacts, il fallait
ajouter un résidu (une erreur) si jamais on ne projetait pas sur la base
entière des vecteurs propres, mais sur seulement quelques uns d'entre
eux (par exemple quand \(V=(v_1,v_2)\)). En notant \(F\) cette matrice
d'erreur on avait alors l'équation de l'ACP qui était:

\[ X = \tilde X \cdot V^T + F\] La PLS est similaire: en notant deux
jeux de données \(X \in \mathbb R^{n,p}\) et \(Y \in \mathbb R^{n,q}\),
la PLS consiste à trouver les solutions de:

\[\left\{
    \begin{array}{ll}
         X = T \cdot C^T + F_X \\
         Y = W \cdot E^T + F_Y
    \end{array}
\right.\]

où \(F_X\) et \(F_Y\) sont les erreurs résiduelles.

Dans la PLS, on cherche à maximiser la covariance entre \(T\) et \(W\)
là où, dans l'ACP, on cherchait à maximiser la variance dans
\(\tilde X\).

\hypertarget{algorithme}{%
\subsection{Algorithme}\label{algorithme}}

On va reprendre et implémenter l'algorithme décrit dans la partie 3.5 du
cours (page 23), en adaptant quelques notations. L'algorithme permet de
résoudre l'équation décrite ci-dessus, la lettre \(\xi\) dans le cours
correspondant à \(T\) dans notre équation. Le critère de convergence
(étape 6) est établi sur \(w\), c'est à dire que si le \(w\) d'une étape
est très proche de celui de l'étape précédente, on considère qu'il y a
convergence.

Rappel : pour un vecteur donné \(w\), le produit \(w^Tw\) peut s'écrire
simplement sous R: \texttt{sum(w\^{}2)}

Voici le pseudo-code pour une composante:

\begin{Shaded}
\begin{Highlighting}[]
\NormalTok{PLS }\OtherTok{\textless{}{-}} \ControlFlowTok{function}\NormalTok{(...) \{}
\NormalTok{  si X et Y ne sont pas centrées}\SpecialCharTok{{-}}\NormalTok{réduites, le faire}
\NormalTok{  initialiser w à la première colonne de Y}
\NormalTok{  initialiser w\_old à w}
\NormalTok{  initialiser t,u,v}
\NormalTok{  tant }\FunctionTok{que}\NormalTok{ (w proche de w\_old) \{}
    \FloatTok{1.}\NormalTok{a calcul de u}
    \FloatTok{1.}\NormalTok{b normaliser u}
    \FloatTok{2.}\NormalTok{ calcul de t}
    \FloatTok{3.}\NormalTok{a calcul de v}
    \FloatTok{3.}\NormalTok{b normaliser v}
    \FloatTok{4.}\NormalTok{a w\_old prend la valeur de w}
    \FloatTok{4.}\NormalTok{b calcul du nouveau w}
\NormalTok{  \}}
  \FloatTok{6.}\NormalTok{ calcul de c et de e}
  \FloatTok{7.}\NormalTok{ deflation}\SpecialCharTok{:}\NormalTok{ calcul des matrices résidues de X et Y}
\NormalTok{  renvoyer w,t,u,v,c,e et les résidus}
\NormalTok{\}}
\end{Highlighting}
\end{Shaded}

Pour faire ce calcul pour plus de composantes, il faut refaire ces
opérations sur les données dont la variance n'est pas encore expliquée,
i.e.~en utilisant les matrices résidues.

Pour pouvoir tester notre algorithme et le comparer, on aura besoin de:

\begin{Shaded}
\begin{Highlighting}[]
\FunctionTok{library}\NormalTok{(matlib)}
\FunctionTok{library}\NormalTok{(plotrix)}
\CommentTok{\#install.packages("devtools")}
\FunctionTok{library}\NormalTok{(devtools)}
\CommentTok{\#install\_github("mixOmicsTeam/mixOmics")}
\FunctionTok{library}\NormalTok{(mixOmics)}
\FunctionTok{data}\NormalTok{(nutrimouse)}
\NormalTok{X }\OtherTok{\textless{}{-}}\NormalTok{ nutrimouse}\SpecialCharTok{$}\NormalTok{gene}
\NormalTok{Y }\OtherTok{\textless{}{-}}\NormalTok{ nutrimouse}\SpecialCharTok{$}\NormalTok{lipid}
\end{Highlighting}
\end{Shaded}

\hypertarget{questions}{%
\section{Questions}\label{questions}}

\begin{enumerate}
\def\labelenumi{\arabic{enumi}.}
\setcounter{enumi}{-1}
\tightlist
\item
  BONUS : Expliquer avec vos mots pourquoi \texttt{sum(w\^{}2)}
  correspond au produit \(w^Tw\). (1 point)
\end{enumerate}

On sait que : \(w^Tw = \sum_{i=0}w_{i }*w_{i }\). De ce
fait,\texttt{sum(w\^{}2)} correspond au produit \(w^Tw\).

\begin{enumerate}
\def\labelenumi{\arabic{enumi}.}
\tightlist
\item
  Implémenter l'algorithme de PLS. On prendra en entrée \(X\), \(Y\), le
  nombre de composantes et \(\varepsilon\) qui sera le seuil pour
  estimer la convergence (typiquement, on prendra souvent
  \(\varepsilon \approx 10^{-6}\)). L'appel que vous pouvez faire pour
  vérifier le bon fonctionnement est par exemple: PLS(X, Y, 2, 0.01). Il
  est conseillé de d'abord faire marcher l'algorithme pour une seule
  composante puis de l'étendre ensuite à un nombre plus grand de
  composantes à l'aide d'une boucle \texttt{for}. (12 points)
\end{enumerate}

\begin{Shaded}
\begin{Highlighting}[]
\NormalTok{PLS }\OtherTok{\textless{}{-}} \ControlFlowTok{function}\NormalTok{(X,Y,eps) \{}
\NormalTok{  u }\OtherTok{\textless{}{-}} \FunctionTok{matrix}\NormalTok{() }
\NormalTok{  t }\OtherTok{\textless{}{-}} \FunctionTok{matrix}\NormalTok{() }
\NormalTok{  v }\OtherTok{\textless{}{-}} \FunctionTok{matrix}\NormalTok{()}
\NormalTok{  list\_comp }\OtherTok{\textless{}{-}} \FunctionTok{c}\NormalTok{()}

\NormalTok{  w }\OtherTok{\textless{}{-}}\NormalTok{ Y[,}\DecValTok{1}\NormalTok{]}
\NormalTok{  w }\OtherTok{\textless{}{-}} \FunctionTok{as.matrix}\NormalTok{(w)}
\NormalTok{  w\_old }\OtherTok{\textless{}{-}}\NormalTok{ w}
  
\NormalTok{  a }\OtherTok{\textless{}{-}}\NormalTok{ (}\FunctionTok{t}\NormalTok{(X)}\SpecialCharTok{\%*\%}\NormalTok{w)}
\NormalTok{  b }\OtherTok{\textless{}{-}} \FunctionTok{sum}\NormalTok{(w}\SpecialCharTok{\^{}}\DecValTok{2}\NormalTok{)}
\NormalTok{  u }\OtherTok{\textless{}{-}}\NormalTok{ a}\SpecialCharTok{/}\NormalTok{b}
\NormalTok{  u }\OtherTok{\textless{}{-}}\NormalTok{ u}\SpecialCharTok{/}\FunctionTok{norm}\NormalTok{(u)}
\NormalTok{  t }\OtherTok{\textless{}{-}}\NormalTok{ X }\SpecialCharTok{\%*\%}\NormalTok{ u[,}\DecValTok{1}\NormalTok{]}
\NormalTok{  v }\OtherTok{\textless{}{-}}\NormalTok{ (}\FunctionTok{t}\NormalTok{(Y)}\SpecialCharTok{\%*\%}\NormalTok{t)}\SpecialCharTok{/}\NormalTok{(}\FunctionTok{sum}\NormalTok{(t}\SpecialCharTok{\^{}}\DecValTok{2}\NormalTok{))}
\NormalTok{  v }\OtherTok{\textless{}{-}}\NormalTok{ v}\SpecialCharTok{/}\FunctionTok{norm}\NormalTok{(v)}
\NormalTok{  w }\OtherTok{\textless{}{-}}\NormalTok{ Y }\SpecialCharTok{\%*\%}\NormalTok{ v}
  
  \ControlFlowTok{while}\NormalTok{ (}\FunctionTok{abs}\NormalTok{(}\FunctionTok{norm}\NormalTok{(w}\SpecialCharTok{{-}}\NormalTok{w\_old)) }\SpecialCharTok{\textgreater{}}\NormalTok{ eps) \{ }
\NormalTok{    u }\OtherTok{\textless{}{-}}\NormalTok{ (}\FunctionTok{t}\NormalTok{(X)}\SpecialCharTok{\%*\%}\NormalTok{w)}\SpecialCharTok{/}\NormalTok{(}\FunctionTok{sum}\NormalTok{(w}\SpecialCharTok{\^{}}\DecValTok{2}\NormalTok{))}
\NormalTok{    u }\OtherTok{\textless{}{-}}\NormalTok{ u}\SpecialCharTok{/}\FunctionTok{norm}\NormalTok{(u)}
\NormalTok{    t }\OtherTok{\textless{}{-}}\NormalTok{ X }\SpecialCharTok{\%*\%}\NormalTok{ u[,}\DecValTok{1}\NormalTok{]}
\NormalTok{    v }\OtherTok{\textless{}{-}}\NormalTok{ (}\FunctionTok{t}\NormalTok{(Y)}\SpecialCharTok{\%*\%}\NormalTok{t)}\SpecialCharTok{/}\NormalTok{(}\FunctionTok{sum}\NormalTok{(t}\SpecialCharTok{\^{}}\DecValTok{2}\NormalTok{))}
\NormalTok{    v }\OtherTok{\textless{}{-}}\NormalTok{ v}\SpecialCharTok{/}\FunctionTok{norm}\NormalTok{(v)}
\NormalTok{    w\_old }\OtherTok{\textless{}{-}}\NormalTok{ w}
\NormalTok{    w }\OtherTok{\textless{}{-}}\NormalTok{ Y }\SpecialCharTok{\%*\%}\NormalTok{ v}
\NormalTok{  \}}
  
\NormalTok{  c }\OtherTok{\textless{}{-}}\NormalTok{ (}\FunctionTok{t}\NormalTok{(X)}\SpecialCharTok{\%*\%}\NormalTok{t)}\SpecialCharTok{/}\NormalTok{(}\FunctionTok{sum}\NormalTok{(t}\SpecialCharTok{\^{}}\DecValTok{2}\NormalTok{))}
\NormalTok{  e }\OtherTok{\textless{}{-}}\NormalTok{ (}\FunctionTok{t}\NormalTok{(Y)}\SpecialCharTok{\%*\%}\NormalTok{t)}\SpecialCharTok{/}\NormalTok{(}\FunctionTok{sum}\NormalTok{(t}\SpecialCharTok{\^{}}\DecValTok{2}\NormalTok{))}
\NormalTok{  X }\OtherTok{\textless{}{-}}\NormalTok{ X }\SpecialCharTok{{-}}\NormalTok{ t}\SpecialCharTok{\%*\%}\FunctionTok{t}\NormalTok{(c)}
\NormalTok{  Y }\OtherTok{\textless{}{-}}\NormalTok{ Y }\SpecialCharTok{{-}}\NormalTok{ t}\SpecialCharTok{\%*\%}\FunctionTok{t}\NormalTok{(e)}
        
\NormalTok{  list\_comp }\OtherTok{\textless{}{-}} \FunctionTok{append}\NormalTok{(list\_comp, }
                      \FunctionTok{list}\NormalTok{(}\StringTok{"w"}\OtherTok{=}\NormalTok{w,}\StringTok{"t"}\OtherTok{=}\NormalTok{t,}\StringTok{"u"}\OtherTok{=}\NormalTok{u,}\StringTok{"v"}\OtherTok{=}\NormalTok{v,}\StringTok{"c"}\OtherTok{=}\NormalTok{c,}\StringTok{"e"}\OtherTok{=}\NormalTok{e,}\StringTok{"X"}\OtherTok{=}\NormalTok{X,}\StringTok{"Y"}\OtherTok{=}\NormalTok{Y))}
  \FunctionTok{return}\NormalTok{(list\_comp)}
\NormalTok{\} }

\NormalTok{PLS\_FOR }\OtherTok{\textless{}{-}} \ControlFlowTok{function}\NormalTok{(X,Y,}\AttributeTok{nbcomp=}\DecValTok{2}\NormalTok{,eps) \{}
\NormalTok{  X }\OtherTok{\textless{}{-}} \FunctionTok{as.matrix}\NormalTok{(X)}
\NormalTok{  Y }\OtherTok{\textless{}{-}} \FunctionTok{as.matrix}\NormalTok{(Y)}
  \ControlFlowTok{if}\NormalTok{ (}\FunctionTok{mean}\NormalTok{(X) }\SpecialCharTok{!=} \DecValTok{0}\NormalTok{) \{}
\NormalTok{    X }\OtherTok{\textless{}{-}} \FunctionTok{scale}\NormalTok{(X)}
\NormalTok{  \}}
  \ControlFlowTok{if}\NormalTok{ (}\FunctionTok{mean}\NormalTok{(Y) }\SpecialCharTok{!=} \DecValTok{0}\NormalTok{) \{}
\NormalTok{    Y }\OtherTok{\textless{}{-}} \FunctionTok{scale}\NormalTok{(Y)}
\NormalTok{  \}}

\NormalTok{  ws }\OtherTok{\textless{}{-}} \FunctionTok{c}\NormalTok{()}
\NormalTok{  ts }\OtherTok{\textless{}{-}} \FunctionTok{c}\NormalTok{()}
\NormalTok{  us }\OtherTok{\textless{}{-}} \FunctionTok{c}\NormalTok{()}
\NormalTok{  vs }\OtherTok{\textless{}{-}} \FunctionTok{c}\NormalTok{()}
\NormalTok{  cs }\OtherTok{\textless{}{-}} \FunctionTok{c}\NormalTok{()}
\NormalTok{  es }\OtherTok{\textless{}{-}} \FunctionTok{c}\NormalTok{()}

  \ControlFlowTok{for}\NormalTok{ (i }\ControlFlowTok{in} \FunctionTok{seq.int}\NormalTok{(}\DecValTok{1}\NormalTok{,nbcomp)) \{}
\NormalTok{    res }\OtherTok{\textless{}{-}} \FunctionTok{PLS}\NormalTok{(X,Y,}\FloatTok{0.01}\NormalTok{)}
\NormalTok{    X }\OtherTok{\textless{}{-}}\NormalTok{ res}\SpecialCharTok{$}\NormalTok{X}
\NormalTok{    Y }\OtherTok{\textless{}{-}}\NormalTok{ res}\SpecialCharTok{$}\NormalTok{Y}
\NormalTok{    ws }\OtherTok{\textless{}{-}} \FunctionTok{cbind}\NormalTok{(ws,res}\SpecialCharTok{$}\NormalTok{w)}
\NormalTok{    ts }\OtherTok{\textless{}{-}} \FunctionTok{cbind}\NormalTok{(ts, res}\SpecialCharTok{$}\NormalTok{t)}
\NormalTok{    us }\OtherTok{\textless{}{-}} \FunctionTok{cbind}\NormalTok{(us,res}\SpecialCharTok{$}\NormalTok{u)}
\NormalTok{    vs }\OtherTok{\textless{}{-}} \FunctionTok{cbind}\NormalTok{(vs,res}\SpecialCharTok{$}\NormalTok{v)}
\NormalTok{    cs }\OtherTok{\textless{}{-}} \FunctionTok{cbind}\NormalTok{(cs,res}\SpecialCharTok{$}\NormalTok{c)}
\NormalTok{    es }\OtherTok{\textless{}{-}} \FunctionTok{cbind}\NormalTok{(es,res}\SpecialCharTok{$}\NormalTok{e)}
\NormalTok{  \}}
  \FunctionTok{return}\NormalTok{(}\FunctionTok{list}\NormalTok{(}\StringTok{"w"}\OtherTok{=}\NormalTok{ws,}\StringTok{"t"}\OtherTok{=}\NormalTok{ts,}\StringTok{"u"}\OtherTok{=}\NormalTok{us,}\StringTok{"v"}\OtherTok{=}\NormalTok{vs,}\StringTok{"c"}\OtherTok{=}\NormalTok{cs,}\StringTok{"e"}\OtherTok{=}\NormalTok{es,}\StringTok{"X"}\OtherTok{=}\NormalTok{X,}\StringTok{"Y"}\OtherTok{=}\NormalTok{Y))}
\NormalTok{\}}

\NormalTok{algo }\OtherTok{\textless{}{-}} \FunctionTok{PLS\_FOR}\NormalTok{(X, Y, }\DecValTok{2}\NormalTok{, }\FloatTok{0.01}\NormalTok{)}
\end{Highlighting}
\end{Shaded}

\begin{enumerate}
\def\labelenumi{\arabic{enumi}.}
\setcounter{enumi}{1}
\tightlist
\item
  Tracer le graphe des individus (avec deux composantes) dans les blocs
  X et Y avec la librairie \texttt{mixOmics}, puis avec votre
  algorithme. (4 points)
\end{enumerate}

Pour les abscisses, nous avons multiplié par (-1) pour optenir le même
graphique que la fonction PLS de R.

\begin{Shaded}
\begin{Highlighting}[]
\NormalTok{res }\OtherTok{\textless{}{-}} \FunctionTok{pls}\NormalTok{(X,Y,}\AttributeTok{ncomp=}\DecValTok{2}\NormalTok{)}
\FunctionTok{plotIndiv}\NormalTok{(res)}
\end{Highlighting}
\end{Shaded}

\includegraphics{Sujet_TD4_files/figure-latex/unnamed-chunk-4-1.pdf}

\begin{Shaded}
\begin{Highlighting}[]
\FunctionTok{par}\NormalTok{(}\AttributeTok{mfrow=}\FunctionTok{c}\NormalTok{(}\DecValTok{1}\NormalTok{,}\DecValTok{2}\NormalTok{))}
\FunctionTok{par}\NormalTok{(}\AttributeTok{mar=}\FunctionTok{c}\NormalTok{(}\DecValTok{1}\NormalTok{, }\DecValTok{1}\NormalTok{, }\DecValTok{1}\NormalTok{, }\DecValTok{1}\NormalTok{),}\AttributeTok{pty =} \StringTok{"s"}\NormalTok{)}

\NormalTok{ax1 }\OtherTok{\textless{}{-}}\NormalTok{ algo}\SpecialCharTok{$}\NormalTok{t[,}\DecValTok{1}\NormalTok{]}\SpecialCharTok{*}\NormalTok{(}\SpecialCharTok{{-}}\DecValTok{1}\NormalTok{)}
\NormalTok{ax2 }\OtherTok{\textless{}{-}}\NormalTok{ algo}\SpecialCharTok{$}\NormalTok{t[,}\DecValTok{2}\NormalTok{]}
\FunctionTok{plot}\NormalTok{(ax1,ax2, }
     \AttributeTok{xlab=}\StringTok{"Component 1 {-} Latent variable associated to X"}\NormalTok{, }
     \AttributeTok{ylab=}\StringTok{"Component 2 {-} Latent variable associated to X"}\NormalTok{,}
     \AttributeTok{main=}\StringTok{"Latent variable associated to X"}\NormalTok{,}
     \AttributeTok{cex=}\DecValTok{0}\NormalTok{,}
     \AttributeTok{panel.first=}\FunctionTok{grid}\NormalTok{())}\CommentTok{\#X}
\FunctionTok{text}\NormalTok{(ax1,ax2,}\AttributeTok{labels=}\FunctionTok{row.names}\NormalTok{(algo}\SpecialCharTok{$}\NormalTok{X),}\AttributeTok{cex =} \FloatTok{0.9}\NormalTok{, }\AttributeTok{pos =} \DecValTok{4}\NormalTok{, }\AttributeTok{col =} \StringTok{"blue"}\NormalTok{)}

\NormalTok{ax11 }\OtherTok{\textless{}{-}}\NormalTok{ algo}\SpecialCharTok{$}\NormalTok{w[,}\DecValTok{1}\NormalTok{]}\SpecialCharTok{*}\NormalTok{(}\SpecialCharTok{{-}}\DecValTok{1}\NormalTok{)}
\NormalTok{ax21 }\OtherTok{\textless{}{-}}\NormalTok{ algo}\SpecialCharTok{$}\NormalTok{w[,}\DecValTok{2}\NormalTok{]}
\FunctionTok{plot}\NormalTok{(ax11, ax21,}
     \AttributeTok{xlab=}\StringTok{"Component 1 {-} Latent variable associated to Y"}\NormalTok{, }
     \AttributeTok{ylab=}\StringTok{"Component 2 {-} Latent variable associated to Y"}\NormalTok{, }
     \AttributeTok{main =} \StringTok{"Latent variable associated to Y"}\NormalTok{,}
     \AttributeTok{panel.first=}\FunctionTok{grid}\NormalTok{(),}
     \AttributeTok{cex=}\DecValTok{0}\NormalTok{)}\CommentTok{\#Y}
\FunctionTok{text}\NormalTok{(ax11,ax21,}\AttributeTok{labels=}\FunctionTok{row.names}\NormalTok{(algo}\SpecialCharTok{$}\NormalTok{X),}\AttributeTok{cex =} \FloatTok{0.9}\NormalTok{, }\AttributeTok{pos =} \DecValTok{4}\NormalTok{, }\AttributeTok{col =} \StringTok{"blue"}\NormalTok{)}
\end{Highlighting}
\end{Shaded}

\includegraphics{Sujet_TD4_files/figure-latex/unnamed-chunk-4-2.pdf}

\begin{enumerate}
\def\labelenumi{\arabic{enumi}.}
\setcounter{enumi}{2}
\tightlist
\item
  Tracer le cercle des variables (avec deux composantes) avec la
  librairie \texttt{mixOmics}, puis avec votre algorithme.
\end{enumerate}

Ici, nous n'avons pas besoin de multiplier par (-1), les données des
variables sont déjà correctement ordonnées.

\begin{Shaded}
\begin{Highlighting}[]
\FunctionTok{plotVar}\NormalTok{(res)}
\end{Highlighting}
\end{Shaded}

\includegraphics{Sujet_TD4_files/figure-latex/unnamed-chunk-5-1.pdf}

\begin{Shaded}
\begin{Highlighting}[]
\FunctionTok{plot}\NormalTok{(}\SpecialCharTok{{-}}\NormalTok{algo}\SpecialCharTok{$}\NormalTok{c[,}\DecValTok{1}\NormalTok{],algo}\SpecialCharTok{$}\NormalTok{c[,}\DecValTok{2}\NormalTok{],}\AttributeTok{col =} \StringTok{"blue"}\NormalTok{,}
     \AttributeTok{xlab=}\StringTok{"Component 1"}\NormalTok{,}\AttributeTok{ylab=}\StringTok{"Component 2"}\NormalTok{,}\AttributeTok{xlim=}\FunctionTok{c}\NormalTok{(}\SpecialCharTok{{-}}\DecValTok{2}\NormalTok{,}\DecValTok{2}\NormalTok{), }\AttributeTok{ylim=}\FunctionTok{c}\NormalTok{(}\SpecialCharTok{{-}}\DecValTok{2}\NormalTok{,}\DecValTok{2}\NormalTok{), }
     \AttributeTok{main =} \StringTok{"Partial regression coefficients from the regression of X and Y"}\NormalTok{)}
\FunctionTok{points}\NormalTok{(}\SpecialCharTok{{-}}\NormalTok{algo}\SpecialCharTok{$}\NormalTok{e[,}\DecValTok{1}\NormalTok{],algo}\SpecialCharTok{$}\NormalTok{e[,}\DecValTok{2}\NormalTok{], }\AttributeTok{col =} \StringTok{"orange"}\NormalTok{)}
\FunctionTok{draw.ellipse}\NormalTok{(}\DecValTok{0}\NormalTok{,}\DecValTok{0}\NormalTok{,}\FloatTok{1.65}\NormalTok{,}\FloatTok{1.85}\NormalTok{,}\AttributeTok{lty=}\DecValTok{1}\NormalTok{,}\AttributeTok{lwd=}\DecValTok{1}\NormalTok{)}
\FunctionTok{draw.ellipse}\NormalTok{(}\DecValTok{0}\NormalTok{,}\DecValTok{0}\NormalTok{,}\FloatTok{0.825}\NormalTok{,}\FloatTok{0.925}\NormalTok{)}
\FunctionTok{legend}\NormalTok{(}\FloatTok{1.5}\NormalTok{,}\DecValTok{2}\NormalTok{,}\AttributeTok{legend=}\FunctionTok{c}\NormalTok{(}\StringTok{"X"}\NormalTok{, }\StringTok{"Y"}\NormalTok{), }
       \AttributeTok{col=}\FunctionTok{c}\NormalTok{(}\StringTok{"blue"}\NormalTok{,}\StringTok{"orange"}\NormalTok{), }\AttributeTok{pch=}\DecValTok{1}\NormalTok{)}
\end{Highlighting}
\end{Shaded}

\includegraphics{Sujet_TD4_files/figure-latex/unnamed-chunk-5-2.pdf}

\end{document}
